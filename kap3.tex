\chapter{Kombinatorikk og utvalgsmodeller}
\label{kap:kombinatorikk} % Opprinnelig kapittelnr: 3

 Vi lar sannsynlighetsteori hvile et øyeblikk, for å kunne presentere
 noen effektive tellemetoder knyttet til utvalg. Dette  kalles ofte 
{\em kombinatorikk}, og kommer oss til nytte i forbindelse
med såkalte {\em utvalgsmodeller}.

\section{Utvalg}

Følgende argumentasjon forekommer ofte i det følgende:

\begin{center} \framebox[12cm]{\begin{minipage}{11cm}\rule{0cm}{0.5cm}
     \noindent {\bf Multiplikasjonsregelen :} En jobb skal utføres i to
     trinn. Trinn 1 kan utføres på $m_1$ ulike måter. For hver av
     de $m_1$ ulike måtene å utføre trinn 1, er det $m_2$ ulike
     måter å utføre trinn 2. Det er da $m_1 \cdot m_2$ ulike
     måter å utføre hele jobben.\\
\end{minipage}} \end{center}

\noindent Denne regelen kan åpenbart utvides til å gjelde jobber med
mer enn to trinn. \\

\begin{eksempel}{En middag}
Hvor mange menyer er det mulig å komponere når vi på en
restaurant har valget mellom 6 forretter, 10 hovedretter og 4
desserter? For hver av de 6 mulige måter å velge forrett er det
10 mulige måter å velge hovedrett. Det er derfor $6\cdot 10 = 60$
mulige måter å velge forrett og hovedrett, og for hver av disse
60 er det 4 mulige måter å velge dessert. Det er derfor i alt
$6\cdot 10\cdot 4 = 240$ måter å komponere menyen på.
\end{eksempel}

Mange praktiske problemstillinger kan tilbakeføres til følgende
generelle situasjon: Gitt en populasjon (dvs. en samling av
objekter) bestående av $N$ elementer. Vi ønsker å trekke et utvalg
på $s$ elementer fra populasjonen. Hva som menes med ordet utvalg
må imidlertid presiseres nærmere, det fins nemlig fire hovedtyper
av utvalg:
\begin{enumerate}
\item  Ordnet utvalg med tilbakelegging.
\item  Ordnet utvalg uten tilbakelegging.
\item  Uordnet utvalg uten tilbakelegging.
\item  Uordnet utvalg med tilbakelegging.
\end{enumerate}
Et ordnet utvalg er et utvalg hvor elementene er ordnet i 
rekkefølge, mens et uordnet utvalg ikke tar hensyn til
rekkefølgen. Trekning med tilbakelegging betyr at et uttrukket
element blir lagt tilbake i populasjonen før neste
trekning.\\


\begin{eksempel}{Utvalg på to}
Anta at vår populasjon består av de tre bokstavene $a$, $b$, og $c$
(dvs. $N=3$). Anta at vi trekker et utvalg på $s=2$ elementer
uten tilbakelegging. Vi skal symbolisere ordnede utvalg med runde
parenteser (), og uordnede utvalg med krøllparenteser \{\}. Vi
har i alt 6 mulige ordnede utvalg på 2 elementer:
\begin{center}
\begin{tabular}{cccccc}
   $(a,b)$,&$(a,c)$,&$(b,c)$,&$(b,a)$,&$(c,a)$,&$(c,b)$ \\
\end{tabular}
\end{center}
mens vi har bare 3 mulige uordnede utvalg på 2 elementer:
\begin{center}
\begin{tabular}{ccc}
   $\{a,b\}$,&$\{a,c\}$,&$\{b,c\} $ \\
\end{tabular}
\end{center}

\noindent Dersom trekningen foregår med tilbakelegging har vi i alt 9
mulige ordnede utvalg på 2 elementer:
\begin{center}
\begin{tabular}{ccccccccc}
 $(a,a)$,&$(a,b)$,&$(a,c)$,&$(b,a)$,&$(b,b)$,&$(b,c)$,&$(c,a)$,&$(c,b)$,&
     $(c,c)$ \\
\end{tabular}
\end{center}
mens vi har bare 6 mulige uordnede utvalg på 2 elementer:
\begin{center}
\begin{tabular}{cccccc}
  $\{a,a\},$&$\{a,b\},$&$\{a,c\},$&$\{b,b\},$&$\{b,c\},$&$\{c,c\}$ \\
\end{tabular}
\end{center}
Merk altså at $\{a,b\}$ og $\{b,a\}$ regnes som samme utvalg,
mens $(a,b)$ og $(b,a)$ regnes som to ulike utvalg, i $(a,b)$ er
$a$ første element og $b$ annet element, i $(b,a)$ er $b$ første
element og $a$ annet element.
\end{eksempel}

\begin{eksempel}{Utvalg på tre}
Gitt en populasjon på $N = 4$ elementer, $a$, $b$, $c$ og $d$. Vi skal
trekke et utvalg på $s=3$ elementer uten tilbakelegging. Det
foreligger i alt 4 mulige uordnede utvalg
\begin{center}
\begin{tabular}{cccc}
         $\{a,b,c\},$&$\{a,b,d\},$&$\{a,c,d\},$&$\{b,c,d\}$ \\
\end{tabular}
\end{center}
mens det foreligger følgende mulige ordnede utvalg:
\begin{center}
\begin{tabular}{cccc}
            $(a,b,c),$&$(a,b,d),$&$(a,c,d),$&$(b,c,d)$ \\
            $(a,c,b),$&$(a,d,b),$&$(a,d,c),$&$(b,d,c)$ \\
            $(b,a,c),$&$(b,a,d),$&$(c,a,d),$&$(c,b,d)$ \\
            $(b,c,a),$&$(b,d,a),$&$(c,d,a),$&$(c,d,b)$ \\
            $(c,a,b),$&$(d,a,b),$&$(d,a,c),$&$(d,b,c)$ \\
            $(c,b,a),$&$(d,b,a),$&$(d,c,a),$&$(d,c,b)$ \\
\end{tabular}
\end{center}
Vi ser at for et gitt uordnet utvalg på tre elementer fins 6
mulige måter å ordne de tre elementene på, dvs. det er i alt
$4\cdot 6=24$ ulike ordnede utvalg.
\end{eksempel}

La oss stille spørsmålet generelt: Hvor mange ulike utvalg på $s$
elementer fra en populasjon på $N$ elementer fins det? Vi skal
besvare dette spørsmålet for hver av de tre første hovedtyper av
utvalg ovenfor. Uordnet utvalg med tilbakelegging forekommer så sjelden
i praksis at vi utelater det.


\section{Ordnede utvalg. Permutasjoner}
\begin{center} \framebox[10cm]{\begin{minipage}{9cm}\rule{0cm}{0.5cm}
     Antall ulike {\em ordnede} utvalg med tilbakelegging på $s$
     elementer fra en populasjon på $N$ elementer er $N^s$.\\
\end{minipage}} \end{center}
Begrunnelse: Vi tenker oss de $s$ elementene utvalgt etter tur.
Når vi etter hver utvelgelse legger tilbake det valgte
elementet, er det hele tiden $N$ mulige valg av neste element. Vi
har $N$ mulige valg av første element, for hvert av disse $N$
mulige valg, er det $N$ mulige valg av annet element, og det er
derfor $N\cdot N = N^2$ mulige måter å velge de to første
elementene. For hver av disse $N^2$ måtene å velge de to første
elementene, er det $N$ måter å velge det tredje, og det er derfor
$N^2\cdot N=N^3$ mulige måter å velge de tre første elementene.
Slik kan vi fortsette, og tankerekken ender slik: For hver av de
$N^{s-1}$ måter å velge de $s-1$ første elementene, er det $N$
måter å velge det siste elementet, og det er derfor $N^{s-1}
\cdot N = N^s$ mulige utvalg på $s$ elementer.\\


\begin{eksempel}{Tipperekker}
En enkeltrekke består av en markering $H$ (hjemmeseier), $U$
(uavgjort) eller $B$ (borteseier) for hver av de 12 kampene på
tippekupongen. Hvor mange ulike enkeltrekker fins det? En
enkeltrekke kan ses på som et ordnet utvalg med tilbakelegging på
$s = 12$ elementer fra populasjonen $H$, $U$, $B$. Følgelig er det
$3^{12}$ ulike enkeltrekker.
\end{eksempel}

\begin{eksempel}{Registreringsskilt}
Anta at et registreringsskilt for motorkjøretøyer består av to
bokstaver fra alfabetet (fra A til og med Å) og fem sifre (fra 0
til og med 9). Hvor mange registreringsskilt er det mulig å lage
på denne måten? For hver av de $29^2$ måtene å velge ut de to
bokstavene på, er det $10^5$ mulige måter å velge ut de 5 sifrene
på, dvs. i alt $29^2 \cdot 10^5$ mulige registreringsskilt.
Dersom vi ikke godtar skilt hvor alle sifrene er null, blir
antallet $29^2 \cdot (10^5-1)$. I praksis brukes ikke Æ, Ø, Å 
og enkelte andre bokstaver og heller ikke visse sifferkombinasjoner.
\end{eksempel}
\begin{center} \framebox[11cm]{\begin{minipage}{10cm}\rule{0cm}{0.5cm}
     Antall ulike {\em ordnede} utvalg {\em uten} tilbakelegging
     på $s$ elementer fra en populasjon på $N$ elementer er
     $(N)_s$ hvor
     \[ (N)_s=N(N-1)(N-2)\cdots (N-s+1) \]
     Symbolet $(N)_s$ uttales gjerne ``$N$ i $s$ faktoriell''.\\
\end{minipage}} \end{center}
\noindent Begrunnelse :
 \footnote{Enkelte foretrekker å bruke symbolet $N^{(s)}$
 istedenfor $(N)_s$.}
Vi tenker oss de $s$ elementene valgt ut etter tur.
Det er $N$ mulige valg av første element. For hvert av disse $N$
mulige valg av første element er det, siden et element nå er
fjernet, $N-1$ mulige valg av annet element. Det er derfor 
$N \cdot (N-1)$ mulige måter å velge de to første elementene på. For
hver av disse $N(N-1)$ måtene er det, siden to elementer er
fjernet, $N-2$ mulige måter å velge det tredje elementet på. Det
er derfor \\
$N(N-1)(N-2)$ mulige måter å velge de tre første
elementene på. Slik kan vi fortsette. Før det siste elementet
skal velges, er det fjernet $s-1$ elementer, og det er derfor $N-
(s-1)=N-s+1$ mulige måter å velge det siste elementet på.

Symbolet $(N)_N$ kan tolkes som antall ulike {\em permutasjoner}
av $N$ elementer, dvs. antall mulige måter å ordne $N$ elementer
på. Vi ser at $(N)_N=N(N-1)(N-2)\cdots 3\cdot 2\cdot 1$. Denne
størrelse betegnes vanligvis med $N$! og vi uttaler ``$N$
fakultet''. \\


\begin{eksempel}{Styret}
I en forening på 10 medlemmer er det 4 styreverv: formann,
viseformann, sekretær og kasserer. Hvor mange
styresammensetninger finnes det? Vi ser at problemet er det samme
som å bestemme antall ordnede utvalg på $s=4$ elementer fra en
populasjon på $N=10$ elementer (uten tilbakelegging). Vi har
følgelig $(N)_s=(10)_4=10\cdot 9\cdot 8\cdot 7 =5040$ mulige
styrer.
\end{eksempel}

\begin{eksempel}{Talelisten}
En representant for hvert av 8 politiske partier skal innlede i en
debatt. Antall mulige talerekkefølger er åpenbart lik antall
mulige måter å permutere $N=8$ elementer, dvs. $N!=8!=40320$
mulige rekkefølger.
\end{eksempel}

\noindent     Nedenfor følger en tabell over antall ordnede utvalg $(N)_s$
for $s$ og $N$ fra 1 til og med 10.
%\begin{center} \small \addtolength{\tabcolsep}{-0.5\tabcolsep}
\renewcommand{\arraystretch}{1.2}
\begin{table}
\begin{tabular}{|r|rrrrrrrrrr|} \hline 
N&s=1&2&3&4&5&6&7&8&9&10 \\ \hline
1&  1& & & & & & & & &   \\ 
2&  2&2& & & & & & & &  \\
3&  3&6&6& & & & & & &  \\
4&  4&12&24&24& & & & & & \\
5&  5&20&60&120&120& & & & &  \\
6&  6&30&120&360&720&720& & & & \\
7&  7&42&210&840&2\,520&5\,040&5\,040& & &  \\
8&  8&56&336&1\,680&6\,720&20\,160&40\,320&40\,320& &  \\
9&  9&72&504&3\,024&15\,120&60\,480&181\,440&362\,880&362\,880&  \\
10& 10&90&720&5\,040&30\,240&151\,200&604\,800&1\,814\,400&3\,628\,800&3\,628\,800\\[2mm]
\hline
\end{tabular}
\caption{Antall ordnede utvalg $(N)_s$}
\label{tab:antall_ordnede}
\end{table}
%\end{center}

\section{Uordnede utvalg}

\begin{center} \framebox[10cm]{\begin{minipage}{9cm}\rule{0cm}{0.5cm}
     Antall ulike {\em uordnede} utvalg {\em uten} tilbakelegging
     på $s$ elementer fra en poulasjon på $N$ elementer betegner
     vi $\bino{N}{s}$ og uttaler ``$N$ over $s$''. Her blir

    \[   \bino{N}{s}= \frac{(N)_s}{s!} \] 
\mbox{}
\end{minipage}} \end{center}

\noindent Begrunnelse: En annen måte å telle opp antall mulige ordnede
utvalg uten tilbakelegging på $s$ elementer fra en populasjon på
$N$ elementer er følgende: Velg først ut de $s$ elementene som
skal være med i utvalget. Dette kan gjøres på $\bino{N}{s}$ mulige
måter. For hver av disse er det $s$! mulige måter å ordne de $s$
elementene på, altså $\bino{N}{s} \cdot s!$ . Herav følger
 påstanden. \\ \\

Innsetter vi uttrykket for $(N)_s$ fra forrige avsnitt får vi
regneformelen

\[ \bino{N}{s}=\frac{N(N-1)(N-2) \cdots(N-s+1)}{s!} \]

\noindent Dersom vi multipliserer teller og nevner med $(N-s)$! får vi
alternativt

\[ \bino{N}{s}=\frac{N!}{s!(N-s)!} \]

\noindent Av denne formelen ser vi følgende symmetri

\[ \bino{N}{s}=\bino{N}{N-s} \]

\noindent Dette kan vi også innse direkte: Istedenfor å velge ut de $s$
elementene som skal utgjøre utvalget, velger vi ut de $N-s$
elementene som ikke skal være med.
Siden   $\bino{N}{N}=1$   ser vi at dersom vi definerer

\[     \bino{N}{0}=1 \mbox{     og      } 0!=1 \]

\noindent så oppnår vi at de to siste formlene gjelder for $s=0$ og
 $s=N$ også.\\


\begin{eksempel}{Delegasjonen}
I en forening på 10 medlemmer skal 4 velges ut som utsendinger
til et møte. Hvor mange ulike delegasjoner finnes? Her er det
bare spørsmål om hvem som kommer med i utvalget, det er ingen
ordning som i Eksempel 6. Problemet er å finne antall uordnede
utvalg på $s=4$ elementer fra en populasjon på $N=10$ elementer
(uten tilbakelegging). Det søkte antall er

\[ \bino{N}{s}=\bino{10}{4}=\frac{10 \cdot 9 \cdot 8 \cdot 7}
                                        {1 \cdot 2 \cdot 3 \cdot 4}=210 \]
\end{eksempel}

\begin{eksempel}{Poker}
En korthånd i poker kan oppfattes som et uordnet utvalg på 5 kort
fra en kortstokk på 52 kort. Antall mulige korthender blir derfor

\[ \bino{N}{s}=\bino{52}{5}=\frac{52 \cdot 51 \cdot 50 \cdot 49 \cdot 48}
                            {1 \cdot 2 \cdot 3 \cdot 4 \cdot 5}= 2\:598\:960 \]
\end{eksempel}

\noindent     For antall uordnede utvalg gjelder følgende identitet

\[ \bino{N}{s}=\bino{N-1}{s-1}+\bino{N-1}{s}  \mbox{\ \ \  } (0 < s < N)\]

\noindent Begrunnelse: Antall ulike utvalg på $s$ elementer fra en
populasjon på $N$ elementer kan alternativt telles opp slik:
Fiks\'{e}r et av elementene i populasjonen. Antall ulike utvalg hvor
dette elementet er med er $\bino{N-1}{s-1}$, mens antall ulike utvalg
hvor det ikke er med er $\bino{N-1}{s}$ .

\begin{table}
%\begin{table}
\begin{tabular}{|r|rrrrrrrrrrr|} \hline
N&s=0&1&2&3&4&5&6&7&8&9&10 \\ \hline
0&  1& & & & & & & & & &   \\ 
1&  1&1& & & & & & & & &   \\ 
2&  1&2&1& & & & & & & &  \\
3&  1&3&3&1& & & & & & &  \\
4&  1&4&6&4&1& & & & & & \\
5&  1&5&10&10&5&1& & & & &  \\
6&  1&6&15&20&15&6&1& & & & \\
7&  1&7&21&35&35&21&7&1& &  &  \\
8&  1&8&28&56&70&56&28&8&1& &  \\
9&  1&9&36&84&126&126&84&36&9&1&  \\
10& 1&10&45&120&210&252&210&120&45&10&1\\[2mm]
\hline
\end{tabular}
%\end{center}
\caption{Antall uordnede utvalg $\bino{N}{s}$}
\label{tab:antall_uordnede}
\end{table}

Vi ser at Tabell~\ref{tab:antall_uordnede} er i overensstemmelse med denne formelen:
Hvert tall inne i tabellen er lik summen av tallet som står rett
ovenfor og tallet som står ovenfor til venstre. Som øvelse kan
leseren føye en ny linje til tabellen. Denne tabellen kalles ofte
{\em Pascals trekant}. En mer omfangsrik tabell er Tabell~\ref{tab:Binomiske_koeffisienter} i
Appendiks~\ref{app:fordelngstabeller}.

\section{Utvalgsmodeller}

I dette avsnitt skal vi studere et par enkle modeller for
trekning av utvalg som ivaretar forestillingen om en rettferdig
trekningsmåte.

\begin{center} \framebox[11cm]{\begin{minipage}{10cm}\rule{0cm}{0.5cm}
\begin{definisjon}
     Med et {\em tilfeldig ordnet utvalg med tilbakelegging} på
      $s$ elementer fra en populasjon på $N$
     elementer menes et ordnet utvalg trukket på en slik måte at
     alle $N^s$ mulige ordnede utvalg har samme sannsynlighet for
     å bli trukket.
\end{definisjon}
\end{minipage}} \end{center}
I denne definisjonen ligger det implisitt at utfallsrommet for
trekningen består av de $N^s$ mulige ordnede utvalg, og at
modellen for trekningen er uniform, dvs. hvert utvalg har
sannsynlighet $1/N^s$ for å bli trukket. Siden modellen er
uniform kan sannsynligheten for enhver begivenhet beregnes ved å
bruke regelen om ``gunstige på mulige''.\\

\begin{eksempel}{Tipperekker}
Anta at en tipperekke blir trukket ut tilfeldig, dvs. slik at
alle $3^{12}$ mulige tipperekker er like sannsynlige. Hva er da
sannsynligheten for begivenheten $A$= en rekke med bare $H$'er og $B$=
en rekke uten $U$'er. Siden vi har uniform modell får vi

\[P(A)=\frac{g}{m}=\frac{1}{3^{12}},\; P(B)=\frac{g}{m}=\frac{2^{12}}{3^{12}}\]

\noindent Merk at en tipperekke uten $U$'er kan oppfattes som et ordnet
utvalg på 12 elementer med tilbakelegging fra en populasjon på 2
elementer, nemlig $\{H,B\}$. Vurder om dette er en realistisk
modell for 12 rette i tipping.
\end{eksempel}

\begin{eksempel}{Fem terningkast}
Utfallet av 5 kast med en terning kan oppfattes som et ordnet
utvalg med tilbakelegging på $s=5$ elementer fra populasjonen
$\{1,2,3,4,5,6\}$. Vi har da $m=6^5$ mulige utfall. Dersom
terningen er rettferdig, er det rimelig å anta at alle disse
$m=6^5$ mulige utfall er like sannsynlige, dvs. uniform modell.
La $A$ være begivenheten at ingen av kastene ga samme antall
øyne. Vi får

\[ P(A)=\frac{g}{m}=\frac{(6)_5}{6^5}=\frac{5}{54} \]

\noindent fordi antall utfall hvor ingen kast viser samme antall øyne er
lik antall ordnede utvalg uten tilbakelegging på 5 elementer fra
populasjonen på 6 elementer.
\end{eksempel}

\begin{eksempel}{Fødselsmåneder}
Hva er sannsynligheten for begivenheten $A$ at det i en gruppe på
6 personer finnes to eller flere som har fødselsdag i samme
måned. La oss anta at alle $m=12^6$ mulige ordnede utvalg av
fødselsmåneder for de 6 personene er like sannsynlige

\[ P(A)=1-P(\bar{A})=1-\frac{(12)_6}{12^6}=0.777 \]

\noindent Diskuter om den brukte modell er realistisk.
\end{eksempel}

Når vi i det følgende snakker om ordnede og uordnede utvalg, vil
vi, dersom ikke annet nevnes, mene utvalg uten tilbakelegging.
\begin{center} \framebox[11cm]{\begin{minipage}{10cm}\rule{0cm}{0.5cm}
\begin{definisjon}
     Med et {\em tilfeldig ordnet utvalg
     uten tilbakelegging} på $s$ elementer fra en populasjon på
     $N$ elementer, menes et ordnet utvalg trukket på en slik
     måte at alle $(N)_s$ ordnede utvalg har samme sannsynlighet
     for å bli trukket.
\end{definisjon}
\end{minipage}} \end{center}

\noindent Utfallsrommet består her av de $(N)_s$ mulige ordnede utvalg, og
siden vi har antatt uniform modell, har hvert mulig utvalg
sannsynlighet $1/(N)_s$.\\

\begin{eksempel}{Styret}
I en forening med 10 medlemmer blir de 4 styrevervene formann,
viseformann, sekretær og kasserer fordelt ved loddtrekning. Vi
ønsker å finne sannsynlighetene for følgende begivenheter
\begin{center}
\begin{tabular}{ccl}
     $A$&$=$& Foreningens eldste blir formann \\
     $B$&$=$& Foreningens eldste blir sekretær \\
     $C$&$=$& Foreningens eldste blir med i styret \\
     $D$&$=$& De 4 yngste medlemmene utgjør styret.
\end{tabular}
\end{center}
En rettferdig trekning betyr at alle $m=(10)_4=5040$ mulige
styrer er like sannsynlige. Vi får da

\[ P(A)=\frac{g}{m}=\frac{(9)_3}{(10)_4}=\frac{1}{10}, \;\;
                        P(B)=\frac{g}{m}=\frac{(9)_3}{(10)_4}=\frac{1}{10} \]
\[ P(C)=\frac{g}{m}=\frac{4 \cdot (9)_3}{(10)_4}=\frac{4}{10}, \;\;
                        P(D)=\frac{g}{m}=\frac{4!}{(10)_4}=\frac{1}{210} \]
\end{eksempel}                 

\begin{eksempel}{Slalåm}
Et slalåmrenn har 15 deltakere og startrekkefølgen avgjøres ved
loddtrek\-ning. Hva er sannsynligheten for følgende begivenheter
\begin{center}
\begin{tabular}{ccl}
    $ A $&$=$& Favoritten starter sist \\
    $ B $&$=$& Favoritten starter blant de 5 siste \\
    $ C $&$=$& Begge nordmenn starter blant de 5 første.
\end{tabular}
\end{center}
Ved en rettferdig trekning blir alle $m=15$! mulige
startrekkefølger like sannsynlige. Vi får 

 \[ P(A)=\frac{g}{m}=\frac{14!}{15!}=\frac{1}{15} \]
 \[ P(B)=\frac{g}{m}=\frac{5 \cdot 14!}{15!}=\frac{1}{3} \]
 \[ P(C)=\frac{g}{m}=\frac{(5)_2 \cdot 13!}{15!}=\frac{2}{21} \]

\noindent Begrunnelser: Skal favoritten starte sist er det 14! mulige måter
å ordne startrekkefølgen for de andre på. For hver av de 5
aktuelle startnumrene for favoritten, er det 14! mulige ordninger
av andre. For hver av de $(5)_2$ aktuelle startnumrene for de to
nordmennene, er det 13! mulige måter å ordne de andre på.
\end{eksempel}

\begin{center} \framebox[11cm]{\begin{minipage}{10cm}\rule{0cm}{0.5cm}
\begin{definisjon}
     Med et {\em tilfeldig uordnet utvalg
     uten tilbakelegging} på $s$ elementer fra en populasjon på
     $N$ elementer menes et uordnet utvalg trukket på en slik
     måte at alle   $\bino{N}{s}$   mulige uordnede utvalg har samme
     sannsynlighet for å bli trukket.
\end{definisjon}
\end{minipage}} \end{center}
Utfallsrommet består her av alle de $\bino{N}{s}$ mulige uordnede
utvalg og hvert av disse har fått tildelt sannsynlighet
 $1/{\bino{N}{s}}$ (uniform modell).\\

\begin{eksempel}{Poker}
Hva er sannsynligheten for begivenheten $A$ = den utdelte korthånd
består av 5 spar (sparflush). Anta at alle $m=\bino{N}{s}$    mulige
korthender er like sannsynlige.

\[ P(A)=\frac{g}{m}= \frac{\bino{13}{5}}{\bino{52}{5}}=
                                   \frac{1\: 287}{2\: 598\: 960}=0.000495 \]

\noindent fordi antall gunstige korthender for begivenheten $A$ er lik antall
uordnede utvalg på 5 kort fra de 13 sparene i kortstokken.
\end{eksempel}

\begin{eksempel}{Delegasjonen}
I en forening på 10 medlemmer er det 6 menn og 4 kvinner. En
delegasjon på 4 medlemmer skal velges ut ved loddtrekning. Vi
ønsker å beregne sannsynlighetene for følgende begivenheter
\begin{center}
\begin{tabular}{ccl}
    $ A $&$=$& Delegasjonen består av bare kvinner \\
    $ B $&$=$& Delegasjonen består av bare menn \\
    $ C $&$=$& Foreningens eldste er med i delegasjonen \\
    $ D $&$=$& Delegasjonen består av 2 kvinner og 2 menn \\
    $ E $&$=$& Delegasjonen består av 1 kvinne og 3 menn.
\end{tabular}
\end{center}
Vi antar rettferdig trekning, dvs. at alle   $\bino{N}{s}$   mulige
(uordnede) utvalg er like sannsynlige, og får 

\[ P(A)=\frac{g}{m}= \frac{1}{\bino{10}{4}}=\frac{1}{210} \]
\[ P(B)=\frac{g}{m}= \frac{\bino{6}{4}}{\bino{10}{4}}=\frac{15}{210}=
                                 \frac{1}{14} \]
\[ P(C)=\frac{g}{m}= \frac{\bino{9}{3}}{\bino{10}{4}}=\frac{84}{210}=
                                 \frac{4}{10} \]
\[ P(D)=\frac{g}{m}= \frac{\bino{4}{2}\cdot \bino{6}{2}}{\bino{10}{4}}=
                     \frac{6 \cdot 15}{210}=\frac{3}{7} \]
\[ P(E)=\frac{g}{m}= \frac{\bino{4}{1}\cdot \bino{6}{3}}{\bino{10}{4}}=
                   \frac{4 \cdot 20}{210}=\frac{8}{21} \]
\noindent Begrunnelser: En delegasjon av bare menn er et utvalg på 4 fra de
6 mennene i foreningen, det er i alt $\bino{6}{4}$ slike. Antall ulike
delegasjoner med 2 kvinner og 2 menn er gitt ved at for hver av
de $\bino{4}{2}$ måtene å velge ut de 2 kvinnene som skal være med,
er det $\bino{6}{2}$ måter å velge ut de 2 mennene, i alt
$\bino{4}{2}\cdot \bino{6}{2}$ slike delegasjoner etc.
\end{eksempel}

\section{ Noen setninger om tilfeldige utvalg}

\begin{center} \framebox[11cm]{\begin{minipage}{10cm}\rule{0cm}{0.5cm}
     1. Et tilfeldig uordnet utvalg (uten tilbakelegging) kan
     oppnås ved å trekke et tilfeldig ordnet utvalg og se bort
     fra rekkefølgen av elementene i utvalget.\\
\end{minipage}} \end{center}
\noindent Begrunnelse: Det er i alt $s$! ordnede utvalg som gir det samme
uordnede utvalg (antall måter å ordne de $s$ elementene i
utvalget). Ved tilfeldig ordnet trekningsmåte blir derfor
sannsynligheten for et bestemt uordnet utvalg lik

\[\frac{s!}{(N)_s}=\frac{1}{\bino{N}{s}} \]

\noindent dvs. samme sannsynlighet som et bestemt tilfeldig uordnet utvalg.
          
\begin{center} \framebox[11cm]{\begin{minipage}{10cm}\rule{0cm}{0.5cm}
     2. Ved trekning av et tilfeldig ordnet utvalg har hvert av
     de $N$ elementene i populasjonen samme sannsynlighet $1/N$
     for å opptre på en bestemt av de $s$ plassene i utvalget.
     Dette kalles ofte for {\em ekvivalensloven} for et tilfeldig
     ordnet utvalg.\\
\end{minipage}} \end{center}
\noindent Begrunnelse: La $A_i=$ det bestemte elementet er på i'te plass i
utvalget. Siden antall gunstige utvalg for begivenheten $A_i$ er
lik antall måter å besette de $s-1$ andre plassene i utvalget, får
vi

\[ P(A_i)=\frac{g}{m}=\frac{(N-1)_{s-1}}{(N)_s}=\frac{1}{N} \]

\begin{eksempel}{Lat elev}
I en klasse på 20 velger læreren ut 5 elever som etter tur skal
regne hver sin av 5 hjemmeoppgaver på tavlen. En elev har ikke
gjort oppgave nr. 4. Sannsynligheten for at han blir bedt om å
regne denne oppgaven er 1/20. Se også Eksempel 13.
\end{eksempel}

\begin{center} \framebox[11cm]{\begin{minipage}{10cm}\rule{0cm}{0.5cm}
     3. Dersom det trekkes et tilfeldig utvalg (uordnet eller
     ordnet) på $s$ elementer fra en populasjon på $N$ elementer,
     så har hvert element i populasjonen sannsynlighet $s/N$ for
     å bli med i utvalget.\\
\end{minipage}} \end{center}

\noindent Begrunnelse: La $A=$ Det bestemte elementet er med i utvalget.
Problemets natur sammen med setning 1 ovenfor medfører at den
søkte sannsynlighet må være den samme for begge typer utvalg, og
det er derfor likegyldig om vi resonnerer uordnet eller ordnet.
For illustrasjonens skyld tar vi likevel med begge
resonnementene. Tilfeldig uordnet utvalg: Siden antall gunstige
utvalg for begivenheten $A$ er lik antall måter å velge ut de 
$s-1$ andre elementene som skal være med i utvalget, får vi

\[ P(A)=\frac{g}{m}=\frac{\bino{N-1}{s-1}}{\bino{N}{s}}=\frac{s}{N} \]

\noindent Tilfeldig ordnet utvalg: Vi kan skrive

\[ A=A_1 \cup A_2 \cup \cdots \cup A_s \]

\noindent hvor $A_i=$ Det bestemte elementet opptrer på i'te plass i
utvalget. Vi vet fra setning 2 ovenfor at $P(A_i)=1/N$ for alle
i, og addisjonssetningen for disjunkte begivenheter gir derfor

\[ P(A)=P(A_1)+P(A_2)+ \cdots +P(A_s)=
           \frac{1}{N}+\frac{1}{N}+ \cdots +\frac{1}{N}=\frac{s}{N} \]                   

\begin{eksempel}{Lat lærer}
En elev har levert inn 10 oppgaver til retting hvorav en oppgave
er galt løst. Læreren velger tilfeldig 4 oppgaver som han retter.
Sannsynligheten for at den gale oppgaven er med i utvalget er
4/10. Se også Eksempel 13 og Eksempel 16.
\end{eksempel}

\section{Tilfeldige tall}

I avsnittene ovenfor har vi sett problemstillinger der det er
tale om å trekke et tilfeldig utvalg fra en gitt populasjon.
Hvordan man realiserer en slik trekning i praksis, har vi bare
flyktig berørt, bl. a. gir de setningene som ble presentert i
forrige avsnitt en viss teoretisk innsikt som kan brukes til å
realisere trekninger. Et nyttig hjelpemiddel i praksis er
tabeller over såkalte {\em tilfeldige tall}, det er grovt sagt en
følge av en-sifrede tall, der hvert tall er en trekning blant
sifrene 0, 1, 2,..., 9 (forhåpentligvis) med samme sannsynlighet.
Tallene i følgen blir trukket uavhengig av hverandre (om begrepet
uavhengighet se Kapittel 4.5). En modell for et tilfeldig tall er
altså
\begin{center}
\renewcommand{\arraystretch}{1.3}
\begin{tabular}{l|cccccccccc}
Siffer&0&1&2&3&4&5&6&7&8&9 \\ \hline
Sannsynlighet&$\frac{1}{10}$&$\frac{1}{10}$&$\frac{1}{10}$&$\frac{1}{10}$
     &$\frac{1}{10}$&$\frac{1}{10}$&$\frac{1}{10}$&$\frac{1}{10}$
     &$\frac{1}{10}$&$\frac{1}{10}$
\end{tabular}
\end{center}
For å kontruere en tabell over tilfeldige tall trenger vi en
mekanisme som, ut fra rene fysiske betraktninger eller generell
erfaring, gir sifre som alle er like sannsynlige, og som
genererer nye tall uavhengig av de som er generert før. I praksis
gjøres dette på en regnemaskin (dator), men selv lommeregnere
gir idag  muligheter for å
generere slike tall. Det finnes også publisert bøker av
tilfeldige tall generert på dette viset. Det er visse filosofiske
problemer forbundet med hva som egentlig er et tilfeldig tall,
men de systemer som er tilgjengelig i dag kan betraktes som
brukbare for de fleste praktiske formål. La oss presentere et
utsnitt av en tabell over tilfeldige tall generert ved en
regnemaskin:

\begin{table}
%\begin{center}
\begin{tabular}{|ccccccc|} \hline
15380 & 30458 & 24235 & 37387 & 55965 & 05730 & 34338 \\
31196 & 55337 & 12100 & 48218 & 77918 & 96825 & 16175 \\
32060 & 74661 & 85245 & 60211 & 55321 & 87577 & 82319 \\
61696 & 28679 & 48462 & 29023 & 31904 & 08143 & 27340 \\
45595 & 03178 & 00973 & 06210 & 27249 & 31618 & 63034 \\
17382 & 39463 & 85125 & 52023 & 41381 & 72824 & 81201 \\ \hline
\end{tabular}
\caption{Tilfeldige tall}
\label{tab:tilfeldige_tall}
%\end{center} 
\end{table}
                                                
For å lette oversikten har vi gruppert tallene fem og fem. Selv
om tabellen inneholder tilfeldige en-sifrede tall kan den også
brukes til å plukke ut tilfeldige to-sifrede tall, idet to
suksessive sifre kan oppfattes som en rett\-ferdig trekning fra
alle 100 to-sifrede tall 00, 01, 02, ..., 97, 98, 99 (dvs. alle
har samme sannsynlighet 1/100). Likeens kan tre suksessive sifre
oppfattes som et tilfeldig tre-sifret tall, dvs. et tall generert
slik at alle 1000 tre-sifrede tall 000, 001, 002, ..., 997, 998,
999 har samme sannsynlighet 1/1000 osv. Ved bruk av en slik
tabell kan en ta tallene i rekkefølge horisontalt, vertikalt
eller på skrå. Følgen av tall som fremkommer skal, dersom
generatoren er brukbar, kunne oppfattes som tilfeldig uansett. En
ting må vi imidlertid passe på, ved gjentatt bruk av slik tabell,
må vi starte på nytt sted hver gang, og helst velge startsted
tilfeldig, ellers vil en lett kunne komme til å misbruke id\'{e}en om
tilfeldige tall. I praksis har en tabell over tilfeldige tall
minst to bruksmåter: Den ene er å trekke et utvalg fra en endelig
populasjon, den andre er å bruke tabellen til å generere data som
brukes i forbindelse med en gitt modell for et system til å
studere egenskapene ved systemet, det siste kalles {\em
simulering}. La oss her se på den første bruksmåten:\\


\begin{eksempel}{Tilfeldige utvalg}
Vi har en gruppe på $N=74$ personer. Fra denne skal trekkes et
tilfeldig utvalg på $s=5$ personer. Vi kan da nummerere personene
slik 00, 01, ...,73. La oss (tilfeldig) velge å bruke siste linje
i tabellen, vi får da følgende to-sifrede tall
\begin{center}
\begin{tabular}{cccccccccc}
    17&38&23&94&63&85&12&55&20&... \\
\end{tabular}
\end{center}
\noindent Her ser vi at tallene 94 og 85 ikke svarer til noen person og
disse neglisjeres. Vårt utvalg på 5 personer består derfor av
personene med nr. 17, 38, 23, 63, 12. Det er intuitivt opplagt, og
kan vises formelt, at dersom alle 100 to-sifrede tall som
genereres er like sannsynlige så vil alle 74 tillatte to-sifrede
tall som framkommer være like sannsynlige, dvs. at vi har en
rettferdig trekning. Det utvalg vi har fått ovenfor er et
(forhåpentligvis) tilfeldig ordnet utvalg. Et tilfeldig uordnet
utvalg får vi ved å se bort fra rekkefølgen av personene. I dette
eksemplet er det tale om utvalg uten tilbakelegging. Skulle samme
to-sifrede tall dukke opp flere ganger, tar vi det med bare en
gang og går videre til neste to-sifrede tall. Dette endrer ikke på
det forhold at de fem tall vi ender opp med kan anses å være
resultat av en rettferdig trekning.
\end{eksempel}

\section{$\star$Grupperinger}
\small

Problemet med å trekke et utvalg (uordnet uten tilbakelegging) på
$s$ elementer fra en populasjon på $N$ elementer kan også sees på
som å dele populasjonen i to grupper, gruppe nr. 1 med $s$
elementer og gruppe nr. 2 med $N-s$ elementer. Det følger da at
det er

\[ \bino{N}{s}=\frac{N!}{s!(N-s)!} \]

\noindent ulike måter å sette sammen de to gruppene på.
                                   
     La oss se på et tilsvarende problem der vi skal dele en
 populasjon i tre grupper: En populasjon på $N$ elementer skal
deles i tre grupper, gruppe nr.1 med $s_1$ elementer, gruppe nr.2
med $s_2$ elementer og gruppe nr.3 med $N-s_1-s_2$ elementer.
I analogi med skrivemåten for to grupper lar vi $\bino{N}{s_1,s_2}$ betegne
antall mulige måter å fordele $N$ elementer på tre grupper med
henholdsvis $s_1, s_2$ og $N-s_1-s_2$ elementer i gruppe nr. 1, 2
og 3. Vi har følgende formel

\[\bino{N}{s_1,s_2}=\bino{N}{s_1}\cdot \bino{N-s_1}{s_2} \]

\noindent En mer symmetrisk formel er

\[\bino{N}{s_1,s_2}=\frac{N!}{s_1!s_2!(N-s_1-s_2)!} \]

\noindent Begrunnelse: Det er i alt$\bino{N}{s_1}$ mulige valg av de $s_1$
elementer fra populasjonen på $N$ som skal utgjøre gruppe nr. 1.
For hvert av disse er det $\bino{N-s_1}{s_2}$ mulige valg av de $s_2$
elementene som skal utgjøre gruppe nr. 2 (disse må velges blant
de $N-s_1$ elementene som ikke allerede er blitt plassert i
gruppe nr. 1). Følgelig er det

\[ \bino{N}{s_1} \cdot \bino{N-s_1}{s_2} \]

\noindent mulige måter å velge ut henholdsvis $s_1$ elementer til
 gruppe nr. 1 og $s_2$ elementer til gruppe nr. 2. De $N-s_1-s_2$ elementene
som er igjen utgjør så gruppe nr. 3, slik at det søkte antall er
nettopp utrykket ovenfor. Dette uttrykk kan skrives

\[ \bino{N}{s_1}\cdot \bino{N-s_1}{s_2}=\frac{N!}{s_1!(N-s_1)!}
                             \cdot \frac{(N-s_1)!}{s_2!(N-s_1-s_2)!} \]

\noindent Ved å stryke $(N-s_1)$! i teller og nevner får vi den alternative formel.\\

\begin{eksempel}{Gruppearbeid}
I en skoleklasse på 15 elever skal det utføres gruppearbeid.
Elevene skal deles i tre grupper med 6 elever i gruppe nr. 1, 5
elever i gruppe nr. 2 og 4 elever i gruppe nr. 3. Dette svarer
til situasjonen ovenfor med $N=15, s_1=6,s_2=5$ og $N-s_1-s_2=4$.
Antall ulike sammensetninger av de tre gruppene er derfor

\[ \bino{15}{6,5}=\bino{15}{6}\cdot \bino{9}{5}=5005 \cdot 126=630\: 630 \]
\end{eksempel}
\noindent Uttrykket $\bino{N}{s_1,s_2}$ var i første omgang tenkt definert
 for $s_1,s_2$ slik at $s_1, s_2>0$ og $s_1+s_2<N$, men vi ser at det er
meningsfylt å sette 

\[ \bino{N}{s_1,0}= \bino{N}{s_1},\: \: \bino{N}{0,s_2}=\bino{N}{s_2} \]

\noindent og

\[ \bino{N}{s_1,s_2}= \bino{N}{s_1}= \bino{N}{s_2}
                                     \mbox{ når  } s_1+s_2=N \]

\noindent slik at $\bino{N}{s_1,s_2}$ når $s_1+s_2=N$ gjerne kan brukes til å
 betegne antall måter å fordele $N$ elementer i to navngitte grupper,
 med henholdsvis $s_1$ elementer i gruppe nr. 1 og $s_2$ elementer i
gruppe nr. 2. Merk at

\[ \bino{N}{0,0}=\bino{N}{N,0}=\bino{N}{0,N} =1 \]

Problemstillingen ovenfor kan åpenbart generaliseres til å dele
en populasjon i et vilkårlig antall grupper: En populasjon på $N$
elementer skal deles i r navngitte grupper, med henholdsvis $s_1$
elementer i gruppe nr. 1, $s_2$ elementer i gruppe nr. 2,...,
$s_{r-1}$ elementer i gruppe nr. $(r-1)$ og $N-s_1-s_2-\cdots -
s_{r-1}$ elementer i gruppe nr. r. Antall ulike slike
grupperinger er (med symbolikk valgt i analogi med det
foregående)

\[ \bino{N}{s_1,s_2, \cdots s_{r-1}}=\bino{N}{s_1} \bino{N-s_1}{s_2}
    \bino{N-s_1-s_2}{s_3} \cdots  \bino{N-s_1-s_2- \cdots -s_{r-2}}{s_{r-1}}  \]

\noindent En mer attraktiv formel er kanskje

\[ \bino{N}{s_1,s_2, \cdots s_{r-1}}=\frac{N!}{s_1!s_2! \cdots
                                              (N-s_1-s_2- \cdots -s_{r-1})!} \]
\begin{eksempel}{Bridge}
Antall initiale spillesituasjoner i Bridge med 13 kort til hver
av spillerne Nord, Øst, Syd og Vest er

\[ \bino{52}{13,13,13}= \frac{52!}{13!13!13!13!}=\frac{52!}{{(13!)}^4} \]

\noindent som er et tall med 29 sifre.
\end{eksempel}

I visse sammenhenger kan det komme på tale å lage modeller for en
gruppering hvor en ivaretar forestillingen om at grupperingen
foretas på en rettferdig måte. Vi antar da en uniform modell der
alle mulige grupperinger antas å være like sannsynlige. \\

\begin{eksempel}{Gruppearbeid}
La situasjonen være som i Eksempel 20. Vi antar at alle $\bino{15}{6,5}$  
mulige fordelinger av de 15 elevene på de tre gruppene med
henholdsvis 6, 5 og 4 elever i gruppe nr. 1, 2 og 3 er like
sannsynlige. Vi ønsker å finne sannsynligheten til begivenhetene
\begin{center}
\begin{tabular}{ccl}
    $A$&$=$& Per kommer i gruppe nr. 1. \\                           
    $B$&$=$& Kameratene Per, Pål og Espen kommer alle i gruppe nr. 1.\\
    $C$&$=$& Per, Pål og Espen kommer i henholdsvis gruppe nr. 1, 2 og 3.\\
\end{tabular}
\end{center}
Siden vi har en uniform modell, kan disse sannsynlighetene regnes
ut ved å bruke regelen om ``gunstige på mulige''. Vi får 

 \[ P(A)=\frac{g}{m}=\frac{\bino{14}{5,5}}{\bino{15}{6,5}}=
     \frac{\frac{14!}{5!5!4!}}{\frac{15!}{6!5!4!}}=\frac{6}{15} \]
 \[ P(B)=\frac{g}{m}=\frac{\bino{12}{3,5}}{\bino{15}{6,5}}=
     \frac{\frac{12!}{3!5!4!}}{\frac{15!}{6!5!4!}}=
    \frac{6 \cdot 5 \cdot 4}{15 \cdot 14 \cdot 13}=\frac{4}{91} \]
 \[ P(C)=\frac{g}{m}=\frac{\bino{12}{5,4}}{\bino{15}{6,5}}=
     \frac{\frac{12!}{5!4!3!}}{\frac{15!}{6!5!4!}}=
          \frac{6 \cdot 5 \cdot 4}{15 \cdot 14 \cdot 13}=\frac{4}{91} \]
\noindent Begrunnelse: Antall grupperinger hvor Per er i gruppe nr. 1 er
lik antall måter å fordele de 14 andre elevene på de tre gruppene
med henholdsvis 5,5 og 4 elever i gruppe nr. 1,2 og 3. Antall
grupperinger hvor Per, Pål og Espen alle er i gruppe nr. 1 er lik
antall måter å fordele de 12 andre elevene på de tre gruppene med
henholdsvis 3, 5 og 4 elever i gruppe nr. 1, 2 og 3. Antall
grupperinger hvor Per, Pål og Espen kommer i henholdsvis gruppe
nr. 1, 2 og 3 er lik antall måter å fordele de 12 andre elevene
på de tre gruppene med henholdsvis 5, 4 og 3 elever i gruppe nr.
1, 2 og 3. \\ \\
\end{eksempel}
En anvendelse med en litt annen karakter har vi i følgende
eksempel.\\

\begin{eksempel}{Oppstilling}
20 rekrutter, hvorav 7 fra Østlandet, 5 fra Vestlandet, 4 fra
Sørlandet og 4 fra Nord-Norge stiller opp, f. eks. i matkø. Hvor
mange ulike køkonstellasjoner finnes det når vi bare tar omsyn
til hjemstavn og ikke person? Her er det nyttig å nummerere
plassene i køen 1, 2, 3,..., 19, 20. En konstellasjon kan da
oppfattes som en gruppering av tallene 1, 2, 3,..., 19, 20 i fire
navngitte grupper, Ø gruppen, V gruppen, S gruppen og N gruppen,
med henholdsvis 7, 5, 4 og 4 tall i hver gruppe (merk at her er
populasjonen tallene 1, 2,..., 20 og ikke mengden av rekrutter).
Det finnes derfor i alt
                      
\[ m=\bino{20}{7,5,4} \]

\noindent mulige ulike konstellasjoner. Anta at de 12 første rekruttene får
kjøttkaker, mens resten må nøye seg med lappskaus. Antar vi at
alle $m$ mulige konstellasjoner er like sannsynlige (diskuter
hvorvidt dette er realistisk), kan vi beregne sannsynligheter.
Eksempelvis la \\ \\
\begin{tabular}{ccl}
     $A$&$=$& Blant de 12 første finnes 6 Østlendinger, \\
        &   & 3 Vestlendinger, 2 Sørlendinger og 1 fra Nord-Norge. \\
     $B$&$=$& Alle Østlendingene er blant de 12 første. \\ \\
\end{tabular}

\noindent Da blir ifølge regelen om ``gunstige på mulige''

\[ P(A)=\frac{g}{m}=\frac{\bino{12}{6,3,2} \cdot \bino{8}{1,2,2}}
    {\bino{20}{7,5,4}}=\frac{93\:139\:200}{6\:983\:776\:800}=0.013 \]
\[ P(B)=\frac{g}{m}=\frac{\bino{12}{7} \cdot \bino{13}{0,5,4}}
    {\bino{20}{7,5,4}}=\frac{71\:351\:280}{6\:983\:776\:800}=0.010 \]

\noindent Forklaring: Vi skal først gruppere de 12 første plassnumrene
 med 6 til Ø, 3 til V, 2 til S og 1 til N. Dette kan gjøres på
$\bino{12}{6,3,2}$ måter. For hver av disse kan vi gruppere de 8 siste
 plassnumrene med 1 til Ø, 2 til V, 2 til S og 3 til N på
 $\bino{8}{1,2,2}$  måter, ialt $\bino{12}{6,3,2}\cdot \bino{8}{1,2,2}$
 måter som er gunstige for begivenheten $A$. For  
begivenheten $B$ kan de 7 plassnumre blant de 12 første som skal
være Ø velges ut på $\bino{12}{7}$ måter. For hver av disse
 er det $\bino{13}{0,5,4}$  måter å gruppere de 13 ledige numre
 slik at 0 gis til Ø, 5 til V, 4 til S og 4 til N.
En alternativ framgangsmåte i dette eksemplet ville være å
identifisere de 20 rekruttene ved navn (eller nr.) og tenke seg
at alle 20! ordninger er like sannsynlige. Ved opptellig i
telleren må en da ta hensyn til at ulike ordninger kan gi samme
konstellasjon.
\end{eksempel}
\normalsize

\section{Oppgaver}
\small
\begin{enumerate}
\item   Handelsreisende Hansen skal reise hjemmefra a til b  og så
     til c, og deretter tilbake til a. \\
     Fra a til b er det 4 reisemåter: fly, buss, båt, tog.\\
     Fra b til c er det 2 reisemåter: buss, båt.\\
     Fra c til a er det 3 reisemåter: fly, buss, båt.\\
     Hvor mange reisemåter foreligger i alt?

\item  Kursen på en aksje observeres på etterfølgende hverdager og
     noteres enten oppgang (+), status quo (0) eller nedgang ($-$).
     Hvor mange utfall finnes det dersom det observeres i 20
     dager? Enn i 30 dager?

\item  Et produkt produseres i serier på 12 enheter. Hver enhet kan
     klassifiseres som intakt (i) eller defekt (d).\\
     (a)  Hvor mange ulike produksjonsserier finnes det?\\
     (b)  Hvor mange serier er slik at ingen intakt kommer etter
          defekt?\\
     Anta at intakte enheter sorteres i tre kvaliteter a, b og c.\\
     (c)  Hvor mange ulike produksjonsserier finnes nå?\\
     (d)  Hvor mange serier er slik at ingen er defekte?\\
     (e)  Hvor mange serier er slik at de 6 siste     
     produksjonsnumre er defekte?\\

\item  En visesangerinne har 10 sanger på repertoaret. Ved en konsert
     skal hun synge 6 av disse. Hvor mange ulike programvalg
     finnes det \\
     (a)  når vi tar hensyn til rekkefølgen av sangene,\\
     (b)  når vi ikke tar hensyn til rekkefølgen av sangene.

\item I en skoleklasse bestående av 10 jenter og 12 gutter skal
     det oppføres et skuespill som har 3 jenteroller og 4
     gutteroller. Hvor mange rollelister finnes det?

\item Hvor mange ulike signal kan vi sende ved å sette flagg i
     rekkefølge dersom\\
     (a)  vi har 4 flagg av ulik farge,\\
     (b)  vi har 5 flagg av ulik farge,\\
     (c)  vi har 2 røde flagg og 4 flagg til av ulik farge.

\item På hvor mange måter kan 7 personer plasseres ved et rundt
     bord dersom\\
     (a)  vi tar hensyn til personenes plassering i forhold til  
        rommet ellers,\\
     (b)  vi ikke tar hensyn til dette,\\
     (c)  den eneste kvinne i laget skal sitte nærmest døren.

\item n land har diplomatiske forhold.\\
     (a)  Hvor mange diplomatiske forhold?\\
     (b)  Hvor mange ambassadører?

\item  I en skoleklasse bestående at 10 jenter og 12 gutter skal 4
     elever utføre et bestemt gruppearbeid. Hvor mange ulike
     sammensetninger av denne gruppen finnes det når\\
     (a)  det velges blant alle elevene,\\
     (b)  alle 4 skal være jenter,\\
     (c)  alle 4 skal være gutter,\\
     (d)  det skal være 2 gutter og 2 jenter med.

\item  Ti personer skal avgjøre hvilket av to produkter x eller y
     de liker best.\\
     (a)  Hvor mange utfall er det for dette eksperimentet når vi
          holder rede på hvem som valgte hva?\\
     (b)  Hvor mange av disse utfall er slik at 5 personer     
     foretrekker x og 5 foretrekker y?\\
     (c)  Hvor mange utfall er slik at 6 personer foretrekker x  
        og 4 foretrekker y?

\item  Ti personer skal avgjøre hvilket av tre produkter x, y og z
     de liker best.\\
     (a)  Hvor mange utfall finnes det i alt?\\
     (b)  Hvor mange utfall er slik at z aldri blir foretrukket?\\
     (c)  Hvor mange utfall er slik at x blir foretrukket oftere
          enn y og z tilsammen?\\
     (d)  Hvor mange utfall er slik at 5 personer foretrekker x,
          3 y og 2 z?\\

Ved løsningen av oppgavene nedenfor bør man først fastlegge
utfallsrom og sannsynlighetsmodell før de enkelte punkter
besvares.


\item  Hvor mange tall med seks sifre finnes det når hvert siffer
     kan være et av tallene 0 til 9 (eksempelvis betyr 004902
     tallet 4902). Hva menes med et tilfeldig seks-sifret tall?\\
     Finn sannsynligheten for at et tilfeldig seks-sifret tall \\
     (a)  ikke starter med null,\\
     (b)  ikke inneholder nuller,\\
     (c)  hverken inneholder nuller eller niere,\\
     (d)  inneholder bare odde sifre,\\
     (e)  har verdi mindre enn 5000,\\
     (f)  inneholder seks forskjellige sifre.\\
     (g)  har tverrsum 3.

\item  Se på situasjonen i Oppgave~10. Gitt at produktene i
     virkeligheten er identiske (bare forskjellig emballasje).
     Finn sannsynlighetene for begivenhetene nedenfor under
     forutsetning av at de ti personene velger ut tilfeldig det
     produkt de ``liker best''.\\
     (a)  5 liker prudukt x og 5 liker produkt y,\\
     (b)  produkt x likes best av minst 6 personer,\\
     (c)  et av produktene likes best av minst 7 personer.

\item  Se på situasjonen i Oppgave~11. Anta at de ti personene
     velger sitt foretrukne produkt tilfeldig. Finn
     sannsynligheten for at\\
     (a)  et av produktene blir foretrukket av minst 5,\\
     (b)  5 personer foretrekker x, 3 personer y og 2 personer z.
              
\item  Til en vinsmakekonkurranse foreligger i alt 20
     rødvinsmerker, hvorav 8 er Bordeaux viner. Hver deltaker får
     i alt 5 smaksprøver og skal for hver besvare spørsmål om
     opprinnelse, årgang etc. Anta at utvalget og rekkefølgen av
     vinene for hver deltaker skjer ved en (rettferdig)
     loddtrek\-ning. Finn sannsynligheten for at en deltaker får
     smake:\\
     (a)  den eldste vinen,\\
     (b)  de to dyreste vinene,\\
     (c)  bare Bordeaux viner,\\
     (d)  Bordeaux vin først,\\
     (e)  Bordeaux vin først og sist.\\
     (f)  aldri to Bordeaux viner etter hverandre.

\item  Seks trommeslagere plasseres i en rekke på en tilfeldig
     måte. Hva er sannsynligheten for at\\
     (a)  Per har ytterplass,\\
     (b)  både Per og Pål har ytterplass,\\
     (c)  hverken Per eller Pål har ytterplass,\\
     (d)  Per og Pål står ved siden av hverandre.

\item  I et musikkorps er det 12 trompetister som er plassert 6 og
     6 i de to siste rekkene. Anta at plasseringen avgjøres ved
     loddtrekning. Hva er sannsynligheten for at Hans og Grete
     står\\
     (a)  bak hverandre,\\
     (b)  i samme rekke,\\
     (c)  ved siden av hverandre,\\
     (d)  begge i ytterplass.

\item  En konkurranse i et ukeblad består i at man skal ``matche''
     fire filmstjerner med fire barnebilder. Under forutsetning
     av at matchingen skjer tilfeldig finn sannsynligheten for\\
     (a)  ingen rette,\\
     (b)  en rett,\\
     (c)  to rette,\\
     (d)  tre rette,\\
     (e)  fire rette.

\item  Et produksjonsparti på 12 enheter inneholder 4 defekte
     enheter. Tre enheter velges ut tilfeldig for kontroll. Hva
     er sannsynligheten for at\\
     (a)  alle er intakte,\\
     (b)  en er defekt,\\
     (c)  minst en er defekt,\\
     (d)  høyst to er defekte.

\item  Et produksjonsparti på 24 enheter inneholder 6 defekte. Hvor
     stort utvalg må trekkes for at sannsynligheten for å få
     minst en defekt er minst\\
     (a)  1/2,  (b)  4/5,   (c)  9/10.\\

\item  I en bedrift med 30 ansatte, hvorav 24 er arbeidere og 6
     funksjonærer, skal velges en komit\'{e} på 4 til å forberede
     julebordet. Dersom utvelgelsen foregår tilfeldig, hva er
     sannsynligheten for at utvalget består av\\
     (a)  ingen funksjonærer,\\
     (b)  3 arbeidere og en funksjonær,\\
     (c)  minst en funksjonær.

\item  I en avdeling i en bedrift er ansatt 10 kvinner og 10 menn.
     Fraværstallene for siste år blir ordnet etter stigende
     rekkefølge og kjønn blir notert f. eks. MKMM ... MM (i dette
     tilfellet er den som har minst fravær mann, nest minst
     kvinne etc.). Anta at fraværstilbøyeligheten for menn og
     kvinner i virkeligheten er den samme, og finn
     sannsynligheten for \\
     (a)  akkurat 5 kvinner og 5 menn blant de 10 med mest     
          fravær,\\
     (b)  minst 6 kvinner blant de 10 med mest fravær,\\
     (c)  minst 4 kvinner blant de 5 med mest fravær.

\item  I en bedrift er det 20 ansatte. Bedriftsklubben har luftet
     tanken om at arbeidstiden i sommerhalvåret skal starte kl.
     7.30 mot kl. 8.00 nå. Det viste seg at 8 var tilhenger av
     det tidligere alternativ (T), mens 12 var tilhenger av det
     sene (S). Man har samtidig notert seg alder og reisetid for
     de ansatte. Gjør de forutsetninger som trengs (diskuter om
     disse er realistiske) for å kunne beregne sannsynligheten
     for at\\
     (a)  ingen av de 3 yngste favoriserer det tidlige          
          alternativ,\\
     (b)  høyst en blant de 5 yngste favoriserer det tidlige     
          alternativ,\\
     (c)  blant de 6 med lengst reisetid var det flertall for det
          tidlige alternativ.

\item I en bedrift med 30 ansatte hvorav 24 er arbeidere (16 menn
     og 8 kvinner) og 6 funksjonærer (4 menn og 2 kvinner) skal
     det dannes et utvalg på 5 medlemmer hvorav 3 skal velges fra
     gruppen arbeidere og 2 fra funksjonær gruppen. Utvelgelsen
     skjer ved loddtrekning. Finn sannsynligheten for at utvalget
     vil bestå av:\\
     (a)  bare menn,\\
     (b)  minst en kvinne,\\
     (c)  minst en kvinnelig arbeider,\\
     (d)  minst en kvinne og en mann fra hver gruppe.

\item  To familier skal flytte inn i samme blokk som Per. Per vet
     at det er et barn i hver familie som skal begynne på samme
     skole som ham, og er derfor spent på kjønn og klassetrinn (1
     til 6). Foreslå egnet utfallsrom, anta uniform modell og
     finn sannsynligheten for\\
     (a)  minst en gutt,\\
     (b)  minst en førsteklassing,\\
     (c)  minst en gutt på trinn 3 eller lavere.\\
     Vil samme modell være realistisk i samme grad dersom det
     dreier seg om en familie med to skolebarn?

\item  En undersøkelse blant alle 200 studentene i et hybelhus ga som
       resultat

\begin{center}
\begin{tabular}{l|cc|c}
           & Har TV    & Har ikke TV & Sum  \\ \hline
Har PC     &    50     &  30         &  80   \\
Har ikke PC&    80     &  40         & 120   \\ \hline
Sum        &   130     &  70         & 200   \\ \hline
\end{tabular}
\end{center}
\begin{itemize}
\item[(a)] Finn sannsynlighetene for at en tilfeldig utvalgt student har \\
           (i) PC (ii) TV (iii) både PC og TV (iv) PC gitt TV.
\item[(b)] Finn sannsynlighetene for at 2 tilfeldig utvalgte studenter har \\
           (i) begge PC (ii) bare en PC (iii) en PC og en TV 
           (iv) begge PC gitt minst en har PC.
\item[(c)] Det siste punktet under (a) og (b) krever muligens en ekstra
           antakelse. Forklar!
\end{itemize}

\item  Hva menes med en tilfeldig korthånd på 13 kort i bridge.
     Finn sannsynligheten for at en tilfeldig korthånd inneholder
     nøyaktig\\
     (a)  bare røde kort,\\
     (b)  4 honnører,\\
     (c)  ingen honnører,\\
     (d)  ett ess,\\
     (e)  to ess.

\item  En pokerhånd på fem kort trekkes ut tilfeldig fra en
     kortstokk. Hva er sannsynligheten for begivenhetene\\
     (a)  et par (to kort med samme verdi og tre kort med andre  
          verdier),\\
     (b)  to par (to ganger to kort med samme verdi og et kort   
          med annen verdi),\\
     (c)  3 like (tre kort med samme verdi og to kort med andre  
          verdier),\\
     (d)  4 like (fire kort med samme verdi og et kort med annen
          verdi),\\
     (e)  flush (alle fem kort av samme slag enten spar, hjerter,
          ruter eller kløver).

\item  En kortstokk på 52 kort blandes godt, og kortene legges ut
     ett etter ett fra toppen. Finn sannsynligheten for at første
     ess kommer\\
     (a)  som femte kort,\\
     (b)  før kort nummer ni.

\item  $\star$ Betrakt produktet $(a+b)^n=(a+b)(a+b)\cdots (a+b)$ hvor $n$
     er ikke negativt heltall. Bruk kombinatorisk argument til
     å vise {\em Newtons binominalformel} :
\begin{eqnarray*}
      (a+b)^n&=&b^n+\bino{n}{1}ab^{n-1}+\bino{n}{2}a^2b^{n-2}+ \cdots
                               +\bino{n}{n-1}a^{n-1}b+a^n  \\
             &=&\sum_{k=0}^{n} \bino{n}{k}a^kb^{n-k} 
\end{eqnarray*}

\item  $\star$ La situasjonen være som i Eksempel 22. Finn sannsynligheten
     for begivenhetene \\
\begin{tabular}{ccl}
    $E$ & = &  Per, Pål og Espen kommer i samme gruppe,\\
    $F$ & = & Per, Pål og Espen kommer i forskjellig gruppe,\\
    $G$ & = &  Per og Pål er i samme gruppe, mens Espen er i en annen
          gruppe,\\
    $H$ & = & Per har minst en kamerat med seg i sin gruppe.
\end{tabular}                                        

\item  $\star$ I Eksempel 23 finn sannsynligheten for at\\
    (a)  minst fem fra Østlandet står blant de 12 første i køen,\\
     (b)  det er like mange fra hver landsdel blant de 12 første
          i køen.

\item  $\star$ I et Bridgespill hvor kortene stokkes godt, finn
     sannsynligheten for at\\
     (a)  Sparene fordeler seg med i spar til Nord, j spar til   
          Øst, k spar til Syd og l spar til Vest ($i+j+k+l=13$).\\
     (b)  Essene fordeler seg med i ess til Nord, j ess til Øst,
          k ess til Syd og l ess til Vest ($i+j+k+l=4$).\\
Hvilken av ``sitsene'' 4000, 3100, 2200, 2110 og 1111 er mest
sannsynlig (uansett spiller).

\item Undersøk din foretrukne programvare om antallet av ulike typer utvalg 
      lar seg lett beregne, i første rekke $(N)_s$ og $ \bino{N}{s}$.
      Disse er ofte tilgjengelige på lommeregnere som egne taster, med
      betegnelsen P for ``permutations" og C for ``combinations".
      Undersøk din lommeregner.
\item Hvordan vil du med din programvare simulere tilfeldig fordeling av
      kort\-hender med 13 kort til hver av de 4 spillerne ved et bridgebord?
\end{enumerate}
\normalsize
