\chapter{Variansanalyse}
\label{kap:variansanalyse} % Opprinnelig kapittelnr: 11

\section{Innledning}

I Kapittel 8.1 studerte vi analyse av data som i vid forstand kunne oppfattes
som gjentatt observasjon av samme ukjente størrelse.  I praksis foreligger
ofte observasjoner der vi har supplerende opplysninger, f.eks. at
observasjonene kan henføres til gitte grupper. Vi kan da være
interessert i eventuelle gruppeforskjeller. Slike problemstillinger
forekommer ofte i forbindelse med kontrollerte eksperimenter (se Kapittel 10),
men også i andre sammenhenger.

Vi skal derfor ta opp generaliseringer
av målemodellen som går under navnet {\em variansanalysemodeller}.
Dette er modeller som forklarer observasjonene ut fra variasjon i en
eller flere kategorivariable, her kalt {\em faktorer}.
Til hver faktor hører to eller flere kategorier (``nivåer''), og
kategorikombinasjonen bestemmer gruppetilhørigheten til observasjonen.
I modellene gjøres antakelser om hvordan faktorene bidrar til den forventede
respons. Vi vil kunne være interessert i hvilken kombinasjon av nivåer
på de ulike faktorer som gir størst forventet respons, eventuelt
om enkelte av faktorene er uten betydning.  Eksempler på dette
kan være:
\begin{itemize}
\item  Innsatsfaktorer og produksjonsresultat.
\item  Markedsføringstiltak og salg.
\item Kostholdsfaktorer og ytelse (evt. reduksjon av risiko for skade/død).
\end{itemize}

Vi vil gi et kort innblikk i aktuelle modeller og analysemetoder knyttet til 
konkrete eksempler. Eksakte pålitelighetsgarantier forutsetter at
respon\-sene er uavhengige og normalfordelte, og er knyttet til $t$-fordelingen,
som vi ble kjent med i Kapittel 8, og en ny fordeling kalt $F$-fordelingen.
Dersom dataene er samlet inn ved et randomisert eksperiment (se Kapittel 10),
vil garantiene gjelde tilnærmet selv uten normalitetsantakelse. 
Uttømmende begrunnelser for disse metodene krever matematiske kunnskaper
ut over det vi har forutsatt til nå, men en grundigere behandling er
imidlertid lett tilgjengelig ellers i litteraturen.

\section{Konstant-effekt og to-utvalgsmodellen}
Vi er ofte interessert i å studere virkningen og eventuelt forskjeller
i responsen ved to ulike framgangsmåter, eksempelvis to 
behandlingsmåter for individer, to produksjonsmåter for artikler
etc.  Vi skaffer oss derfor observasjoner for hver av framgangsmåtene.
Hvilken modell som er aktuell vil avhenge av omstendighetene, vi vil her
studere to vesensforskjellige situa\-sjoner, som best illustreres ved
eksempler.\\

\begin{eksempel}{Spesialverktøy}
En vanskelig arbeidsoperasjon kan utføres uten eller med
spesialverktøy. 
Vi ønsker å anslå den tidsforskjell som forventes ved de to
metodene.\footnote{Denne situasjonen ble også brukt som
gjennomgangseksempel i Kapittel 10 om kontrollerte eksperimenter.}
Anta at bedriften har latt $n$ = 10 arbeidere utføre
arbeidsoperasjonen en gang med hver metode.  La $X_i$ og $Y_i$ være
forbrukt tid for $i$'te arbeider ved henholdsvis uten og med
spesialverktøy $(i = 1, 2, \ldots, n).$  Vi vil anta at $X_1, X_2,
\ldots, X_n, Y_1, Y_2, \ldots, Y_n$ er uavhengige stokastiske variable
med forventninger henholdsvis

\[ EX_i = {\alpha}_i +\delta \mbox{\ \ \ } EY_i = {\alpha}_i
                                    \mbox{\ \ \ }  (i = 1, 2, \ldots, n)\]
\noindent og varianser

\[ varX_i = {\sigma}_1^2 \mbox{\ \ \ }  varY_i = {\sigma_2}^2 
                                    \mbox{\ \ \ } (i = 1, 2, \ldots, n)\]
I denne modellen blir $\delta$ å oppfatte som tillegget (evt. fradraget)
i forventet tid som skyldes at spesialverktøy ikke brukes, og det vil
være av interesse å estimere $\delta$, eventuelt teste hypoteser om 
$\delta$ på grunnlag av observasjonene.  Merk at modellen tillater at 
de ulike arbeiderne har ulikt arbeidstempo (dvs. ulike $\alpha_i$), men 
det kreves altså at tillegget (fradraget) $\delta$ er det samme for
alle arbeiderne.  Videre ser vi at variasjonen i arbeidstempo tillates å
være ulik for de to metodene, men for hver metode kreves det at den
er samme for alle arbeiderne.  Disse antakelsene kan selvsagt diskuteres.
Den mest interessante er kanskje at tillegget $\delta$ antas konstant,
og det har gitt modellen navnet {\em konstant-effekt modellen}.
\end{eksempel}

Vi vil nå studere inferensproblemer for modellen presentert i 
eksemplet.  Betrakt forskjellen i respons for hver arbeider

\[  Z_i = X_i - Y_i  \mbox{\ \ \ }   (i = 1, 2, \ldots, n).      \]
Ifølge antakelsene i modellen blir

\begin{eqnarray*}
EZ_i&=& E(X_i-Y_i)=EX_i-EY_i={\alpha}_i + \delta - {\alpha}_i = \delta \\
varZ_i&=& var(X_i-Y_i)=varX_i+varY_i={\sigma}_1^2 + {\sigma}_2^2 
\end{eqnarray*}
Vi ser nå at $Z_1, Z_2, \ldots, Z_n$ er uavhengige stokastiske
variable med samme forventning $\delta$ og med samme varians
${\sigma}^2={\sigma}_1^2 + {\sigma}_2^2$, dvs.
oppfyller betingelsene i den enkle målemodellen fra Kapittel 8.1,
og estimering av $\delta$ er derfor redusert til et velkjent problem
basert på observasjonene $Z_1, Z_2, \ldots, Z_n$.  En aktuell
estimator for $\delta$ er 

\[ \hat{\delta}=\bar{Z}=\frac{1}{n} \sum_{i=1}^{n} Z_i=\bar{X}-\bar{Y}.  \]
Denne er forventningsrett med varians ${\sigma}^2/n$.
Vanligvis er ${\sigma}^2$ ukjent, men en forventningsrett estimator
er som før

\[  S^2=\frac{1}{n-1}  \sum_{i=1}^{n} {(Z_i-\bar{Z})}^2     \]
Som estimator for standardavviket $\sigma ({\hat{\delta}})=\sigma /\sqrt{n}$,
bruker vi $S({\hat{\delta}})=S/\sqrt{n}$,
Estimert $\delta$ med tilhørende standardavvik blir derfor

\[   \bar{Z} \pm \frac{S}{\sqrt{n}}        \]
\addtocounter{eksecount}{-1}
\begin{eksempel}{Spesialverktøy (fortsatt)}
La oss illustrere dette med data for Eksempel 1.  Anta at 
observasjonene (i sekunder) er gitt ved 
\begin{center}
\begin{tabular}{ccccccccccc}
$X_i$: &  48  &  53  &  52  &  57  &  43  &  63  &  59  &  51  &  40  &  61 \\
$Y_i$: &  45  &  42  &  58  &  50  &  41  &  47  &  53  &  46  &  45  &  53 
\end{tabular}
\end{center}
Differensene i forbrukt tid blir
\begin{center}
\begin{tabular}{ccccccccccc}
$Z_i$: & 3  &  11  & $-6$  &  7  &  2  &  16  &  6  &  5 & $-5$ & 8 
\end{tabular}
\end{center}
Utregningen gir $\bar{Z}=4.7$ og $S=6.7$, og for anslått forventet
reduksjon i forbrukt tid med spesialverktøy rapporterer vi $4.7 \pm 2.1$ .
Sett i forhold til det estimerte standardavvik, ser det ut til at reduksjonen
er signifikant, dvs. neppe skyldes tilfeldigheter.  Om reduksjonen er av
praktisk betydning for bedriften er en annen sak.
\end{eksempel}

Eksakte sannsynlighetsutsagn om estimeringsfeilen i konstanteffekt
mo\-del\-len kan ikke gjøres uten ytterligere forutsetninger.  Dersom vi
er villig til å anta at observasjonene er normalfordelte, kan man
ta utgangspunkt i den stokastiske variable

\[  T=\frac{\bar{Z}-\delta}{S/\sqrt{n}}          \]
som da er $t$-fordelt med $n - 1$ frihetsgrader (se Kapittel 8.4).  
Ved hjelp av en tabell over arealer under $t$-kurver kan vi derfor 
lage sannsynlighetsutsagn.  Eksempelvis med 9 frihetsgrader er
sannsynligheten for estimeringsfeil på høyst 2.26 ganger estimert
standardavvik lik 0.95 (se Tabell~\ref{tab:T_kurver_Fraktil}).

Det kan også være aktuelt å teste hypoteser om $\delta$,
f.eks. teste hypotesen $\delta = {\delta}_0$, der ${\delta}_0$ er
et spesifisert tall.  Hypotesen om ingen forskjell mellom de to metodene
svarer til å sette ${\delta}_0$ = 0.  En mulig testobservator er 
$T$, der vi for $\delta$ bruker den spesifiserte ${\delta}_0$.  Under
forutsetning av normalfordelte observasjoner vil denne $T$, dersom 
nullhypotesen er riktig, være $t$-fordelt med $n - 1$ frihetsgrader.
Forkastingsområdet vil avhenge av om problemstillingen er ensidig
eller to-sidig.  I eksemplet vil det første være tilfelle dersom
vi på forhånd vet at spesialverktøy i hvert fall ikke øker
forventet forbrukt tid.  Ved hjelp av $t$-tabellen kan vi derfor etablere
tester med ønsket signifikansnivå, eventuelt beregne $P$-verdier.

En viktig grunn til at konstant-effekt modellen er rimelig i Eksempel 1,
er at responsen for de to framgangsmåtene blir observert for samme
individ.  Vi vil nå se på en problemstilling der dette ikke er 
tilfelle.\\

\begin{eksempel}{Spesialverktøy}
Anta at problemstillingen er som i Eksempel 1, men at bedriften isteden
har latt $n_1$ = 10 arbeidere utføre arbeidsoperasjonen etter metode
1 (uten spesialverktøy) og $n_2$ = 11 andre arbeidere etter metode 2
(med spesialverktøy).  La oss nummerere arbeiderene innen hver
gruppe, og la $X_j$ være forbrukt tid for $j$'te arbeider i gruppe
nr.1 ($j = 1, 2, \ldots, n_1$) og $Y_j$ være forbrukt tid for 
$j$'te arbeider i gruppe nr. 2 ($j = 1, 2, \ldots, n_2$). 
Anta at observasjonene (i sekunder) er gitt ved:
\begin{center}
\begin{tabular}{cccccccccccc}
$X_j$:& 45  &  48  &  61  &  52  &  48  &  63  &  52  &  54  &  50  &  58 &\\
$Y_j$:& 47  &  59  &  43  &  50  &  45  &  45  &  49  &  41  &  47  &  52 & 50
\end{tabular}
\end{center}
Vi vil basere dataanalysen på følgende modell:
\begin{center} \framebox[10cm]{\begin{minipage}{9cm}\rule{0cm}{0.5cm}
{\bf To-utvalgsmodellen :} $X_1, X_2, \ldots, X_{n_1},
  Y_1, Y_2, \ldots, Y_{n_2}$
 er uavhengige stokastiske variable med forventninger henholdsvis
\[   EX_j = {\mu}_1 \mbox{\ \ \ \ \ }   EY_j = {\mu}_2     \]
og varianser henholdsvis
\[      var X_j = {\sigma}_1^2  \mbox{\ \ \ \ \ } var Y_j = {\sigma}_2^2   \]
\mbox{}
\end{minipage}} \end{center}
Vi ser at to-utvalgsmodellen er sammensatt av to enkle 
målemodeller, en for hver gruppe, med tilleggsantakelsen at
observasjonene i de to gruppene er uavhengige.
Sammenligner vi modellen for dette eksemplet med den i Eksempel 1,
ser vi at her oppfattes observasjonene
som målinger av et felles forventet prestasjonsnivå for hver
metode.  De individuelle variasjoner fra arbeider til arbeider er nå
å oppfatte som del av de tilfeldige variasjonene omkring dette
nivået.  En vil derfor vente at variansene ${\sigma}_1^2$ og
${\sigma}_2^2$ vil være betydelig større her enn i Eksempel 1.
Vi vil være interessert i å estimere forventet
tidsforskjell mellom de to metodene, i denne modellen uttrykt ved
$\delta = {\mu}_1 - {\mu}_2$.  Siden $\bar{X}$ og $\bar{Y}$ er
forventningsrette estimatorer for henholdsvis ${\mu}_1$ og ${\mu}_2$, ser vi at

\[ \hat{\delta}=\bar{X}-\bar{Y}=\frac{1}{n_1} \sum_{j=1}^{n_1} X_j-
                                  \frac{1}{n_2} \sum_{j=1}^{n_2} Y_j \]
er forventningsrett estimator for $\delta = {\mu}_1 - {\mu}_2$.
Estimatoren har varians

\[ var\hat{\delta}=\frac{{\sigma}_1^2}{n_1}+\frac{{\sigma}_2^2}{n_2}
                  =(\frac{1}{n_1}+\frac{1}{n_2}){\sigma}^2 \]
Den siste likheten forutsetter at variansen i de to gruppene er like,
dvs. ${\sigma}_1^2={\sigma}_2^2={\sigma}^2$, en antakelse som ofte
gjøres i praksis, delvis fordi teorien blir enklere.
I dette tilfellet kan nemlig

\[  S^2=\frac{1}{n_1+n_2-2}( \sum_{j=1}^{n_1}{(X_j-\bar{X})}^2 +
                           \sum_{j=1}^{n_2}{(Y_j-\bar{Y})}^2)      \]
brukes som forventningsrett estimator for fellesvariansen ${\sigma}^2$,
og standardavviket  til estimatoren estimeres med $S(\hat{\delta})$, der
$S$ erstatter $\sigma$ i formelen for $\sigma (\hat{\delta})$.
Estimert $\delta$ med tilhørende standardavvik blir derfor

\[  \bar{X}-\bar{Y} \pm S \cdot \sqrt{\frac{1}{n_1}+\frac{1}{n_2}} \]
Med de gitte observasjonene får vi 
$\bar{X}$ = 53.1, $\bar{Y}$ = 48.0 og $S$ = 5.40, slik at vår rapport
 om differensen mellom forventningene $\delta$ blir $5.10 \pm 2.36$.
Sammenholdes den estimerte reduksjon i forbrukt tid med de estimerte
standardavvik, ser det ut til at reduksjonen er signifikant, dvs. neppe kan
skyldes tilfeldigheter.

Vi kan heller ikke i denne modellen gi eksakte sannsynlighetsutsagn
om estimeringsfeilen uten ytterligere forutsetninger.  Dersom vi er
villige til å anta at observasjonene er normalfordelte, kan man
ta utgangspunkt i den stokastiske variable

\[ T=\frac{\hat{\delta}-{\delta}}{S \cdot
                         \sqrt{\frac{1}{n_1}+\frac{1}{n_2}}} \]
som da er $t$-fordelt med $n_1 + n_2 - 2$ frihetsgrader.  Ved hjelp av
en $t$-tabell kan vi derfor lage sannsynlighetsutsagn.  Eksempelvis
med 19 frihetsgrader, er sannsynligheten for en estimeringsfeil på
høyst 2.09 ganger estimert standardavvik lik 0.95 (se Tabell~\ref{tab:T_kurver_Fraktil}).

Også for denne modellen er det aktuelt å teste 
hypotesen ${\delta} = {\delta}_0$, der ${\delta}_0$ er et spesifisert tall,
eksempelvis null. 
 En mulig testobservator er $T$, der vi for $\delta$ bruker den
spesifiserte ${\delta}_0$.  Under forutsetning av normalfordelte 
observasjoner vil denne $T$, dersom nullhypotesen er riktig, være
$t$-fordelt med $n_1 + n_2 - 2$ frihetsgrader, og ved hjelp av
$t$-tabellen kan vi derfor etablere tester med ønsket signifikansnivå,
eventuelt beregne $P$-verdier. \\

\noindent {\bf Merknad.} Teorien ovenfor forutsetter at vi er villig 
til å anta at variansene ${\sigma}_1^2$ og ${\sigma}_2^2$ er like.
Uten denne antakelsen må vi estimere variansene i hver gruppe for seg:

\[  S_1^2=\frac{1}{n_1-1} \sum_{j=1}^{n_1}{(X_j-\bar{X})}^2 \mbox{\ \ \ \ \ }
    S_2^2=\frac{1}{n_2-1} \sum_{j=1}^{n_2}{(Y_j-\bar{Y})}^2 \]
og deretter beregne en estimator for $\sigma(\hat{\delta})$ der
${\sigma}_1$ og ${\sigma}_2$ erstattes med henholdsvis $S_1$ og $S_2$.
Teorien er nå langt mer komplisert, men gode tilnærmingsmetoder
er implementert i programvare, slik at brukeren ikke behøver å
gjøre antakelsen om like varianser, dersom denne er urealistisk.


\begin{center} \framebox[10cm]{\begin{minipage}{9cm}\rule{0cm}{0.5cm}
\tt
  >> READ 'eks11.2' X Y \\
  >> TWOSAMPLE X Y; POOLED S. \\
\begin{tabular}{rrrrr}
    &   N  &   Mean  &  StDev & SE Mean \\
  X &  10  &  53.10  &   5.92 &     1.9 \\
  Y &  11  &  48.00  &   4.90 &     1.5
\end{tabular} \\
 Pooled Standard Deviation =  5.40  \\
 Standard Error Difference of Means = 2.36 \\
 95 \% Conf.Int. for EX - EY : (0.2, 10.0) \\
 T-test EX = EY (VS. NEQ): T=2.16 DF=19 P=0.044 \\
\mbox{}
 \end{minipage}} \end{center} 
En typisk utskrift fra en programpakke er gitt ovenfor, og gir
i hovedsak  de tidligere omtalte størrelser.
De individuelle standardavvik er ikke så mye forskjellige at
antakelsen om felles verdi er urimelig. Sist i utskriften er resultatet
av $T$-testen for $H_0$ : $\delta =0$ mot $H_A$ : $\delta \neq 0$.
Dersom vi krever 5\% signifikansnivå, kan vi forkaste hypotesen om
lik forventet tid ($P < 0.05)$.
\end{eksempel}
For begge modellene ovenfor kan det vises, dersom vi antar at
observasjonene er normalfordelte, at de metoder som er
foreslått er optimale i en viss forstand for inferens om
${\delta}$.  I mange situa\-sjo\-ner vil det imidlertid være urealistisk
å anta normalitet.  Det gjelder spesielt i situasjoner der det er 
en viss mulighet for ``ville" observasjoner.  I eksemplene ovenfor kan det
tenkes at slike observasjoner oppstår ved et arbeidsuhell hos en
arbeider, som ikke er forårsaket av den arbeidsmetode som brukes.  
Det er likevel ikke tilrådelig å luke ut eventuelt avvikende 
observasjoner på rent subjektiv grunnlag, kan hende skyldes avviket
likevel metoden.  Det finnes imidlertid alternative metoder for
sammenligning av to framgangsmåter, med gunstige egenskaper, som
bl.a. unngår at ``ville" observasjoner får avgjørende
innflytelse på de konklusjoner som trekkes.

For konstant-effekt modellen i Eksempel 1, vil medianen av differensene
være en alternativ estimator for $\delta$.  En alternativ 
testmetode vil være Wilcoxon-testen for parrede observasjoner.  For
modellen med to forventninger i Eksempel 2, vil differensen mellom
medianene være en alternativ estimator for ${\sigma}$, mens
Wilcoxon testen for to utvalg kan være en alternativ testmetode.
De nevnte testmetoder er basert på ranger, og er beskrevet i
Kapittel 10.4.  De forutsetter ikke normalitet, og kan rettferdiggjøres
i mer generelle situasjoner enn de som er beskrevet i Kapittel 10.4.  
Selv når observasjonene er normalfordelte, så taper en lite i
forhold til de ``optimale" metodene, mens gevinsten kan være betydelig
for store avvik fra normalitet.

\section{En-faktor modellen}
En møter ofte situasjoner der en ønsker å sammenligne mer enn to
grupper, definert ved flere kategorier for en faktor, som vi nedenfor
betegner $A$.\\

\begin{eksempel}{Produksjonsmetoder}
En bedrift vurderer i alt $r=3$ ulike produksjonsmetoder for et produkt.
Faktoren $A$ er her produksjonsmetode. 
For å sammenligne metodene m.h.t. kvaliteten av det ferdige produkt,
produseres 5 enheter med hver metode og kvaliteten av disse tallfestes. 
Resultatet ble:

%\begin{center}
\begin{tabular}{lccccc}
Produksjonsmetode nr.~1 : & 4.7 & 3.5 & 3.3 & 4.2 & 3.6 \\
Produksjonsmetode nr.~2 : & 3.2 & 4.2 & 3.3 & 3.9 & 3.0 \\
Produksjonsmetode nr.~3 : & 3.1 & 2.9 & 2.2 & 3.0 & 2.8
\end{tabular}
%\end{center}
\end{eksempel}% 

Vi skal bruke følgende generelle notasjon : Det produseres $n_i$ artikler 
med metode nr. $i$ ($i = 1, 2, \ldots, r$), og $Y_{ij}$ er målt
kvalitet av den $j$'te produserte artikkel med metode nr.$i$. 
($j = 1, 2,\ldots, n_i; i = 1, 2, \ldots, r$).
Vi vil basere dataanalysen på en modell der vi antar at alle disse
stokastiske variable er uavhengige, og der forventningen og 
variansen er den samme for observasjonene med samme metode, men
muligens ulik fra metode til metode, dvs.
\[ EY_{ij} = {\mu}_i \mbox{\ \ \ \ } varY_{ij} = {\sigma}_i^2 \mbox{\ \ }
                                              j = 1, 2, \ldots, n_i. \]
Parameteren ${\mu}_i$ er altså forventet kvalitet med metode nr.$i$.
Ofte kan det være rimelig å anta at variansen er den samme for 
de ulike metodene, kall i så fall den felles verdi for ${\sigma}^2$.

Som estimator for kvalitetsnivåene for de ulike metodene, kan vi 
bruke de respektive gjennomsnitt

\[   \bar{Y}_i=\frac{1}{n_i}\sum_{j=1}^{n_i}Y_{ij} \mbox{\ \ \ }
                                                 j=1,2,\ldots ,n_i, \]
som er forventningsrette med varianser henholdsvis ${\sigma}_i^2/n_i$.
Variansene ${\sigma}_i^2$ for hver metode kan vi estimere på vanlig
 måte, kall estimatorene $S_i^2$. Har vi antatt en felles ${\sigma}^2$,
kan vi utnytte alle observasjonene ved estimeringen. En forventningsrett
estimator for ${\sigma}^2$ er da

\[ S^2=\frac{1}{n-r}\sum_{i=1}^{r}\sum_{j=1}^{n_i}{(Y_{ij}-\bar{Y}_i)}^2
                                 =\frac{1}{n-r}\sum_{i=1}^{r}(n_i-1)S_i^2 \]
der $n = \sum_{i=1}^r n_i$ er det totale antall observasjoner. Vi kan rapportere
 forventet kvalitet for hver metode slik

\[     \bar{Y}_i \pm \frac{S}{\sqrt{n_i}} \mbox{\ \ \ } i=1,2,\ldots ,r.  \]
Vi kan være interessert i å vurdere eventuelle
forskjeller i forventet kvalitet.  I en situasjon med mer enn to 
forventninger, kan vi sammenligne to og to, og rapportere
forskjeller som i Eksempel 2. For metode nr.$i$ og nr.$k$ :

\[ \bar{Y}_i -\bar{Y}_k\pm S\sqrt{\frac{1}{n_i}+\frac{1}{n_k}} \]
Eksakte konfidensintervaller kan i begge tilfeller lages med sikkerhetsfaktor
hentet fra $t$-tabellen med $n-r$ frihetsgrader (forutsetter normalfordelte
observasjoner).

Ved sammenligning av en rekke forventninger kan det være lurt å
ta utgangspunkt i hypotesen at alle forventningene er like, dvs.

\[    H_A: {\mu}_1 = {\mu}_2 = \cdots = {\mu}_r.   \]
og teste den mot alternativet at minst to forventninger er ulike.
Hvis denne hypotesen ikke forkastes, er det liten grunn til å legge
vekt på gruppeforskjeller, de kan like godt skyldes tilfeldigheter.
Vi forutsetter at alle observasjonene har samme varians  ${\sigma}^2$.
Uten denne antakelsen er det vanskelig å trekke 
generelle konklusjoner.  Dersom nullhypotesen $H_A$ var riktig, kunne
vi bruke gjennomsnittet av alle observasjonene

\[ \bar{Y}=\frac{1}{n}\sum_{i=1}^{r}\sum_{j=1}^{n_i}Y_{ij} =
                \frac{1}{n}\sum_{i=1}^{r}n_i\bar{Y}_i \]
til å anslå den felles forventning.  Dersom gruppegjennomsnittene\\
$\bar{Y}_1, \bar{Y}_2, \ldots, \bar{Y}_r$ avviker lite fra det totale
gjennomsnitt $\bar{Y}$, er det ingen grunn til å forkaste nullhypotesen,
mens store avvik tyder på at nullhypotesen ikke er riktig.  Et mål
som sammenfatter graden av avvik er

\[     Q_A=\sum_{i=1}^{r}n_i{(\bar{Y}_i-\bar{Y})}^2 \]
Her er $n_i$ brukt som vekter i summen, bl.a. fordi et bestemt avvik
basert på et stort antall observasjoner bør tillegges større
vekt enn et tilsvarende basert på et lite antall observasjoner.
Dersom $Q_A$ er stor gir dette grunn til å forkaste nullhypotesen.
Stor i forhold til hva? Vi må åpenbart se $Q_A$ i forhold til den
 naturlige tilfeldige variasjon anslått med $S^2$. 
 Betrakt kvadratsummene

\[ Q_0=\sum_{i=1}^{r}\sum_{j=1}^{n_i}{(Y_{ij}-\bar{Y_i})}^2 \mbox{\ \ \ \ }
    Q=\sum_{i=1}^{r}\sum_{j=1}^{n_i}{(Y_{ij}-\bar{Y})}^2 \]
som måler observasjonenes variasjon omkring sine respektive
gruppegjennomsnitt og omkring det totale gjennomsnitt.  Det kan vises at

\[           Q = Q_A + Q_0             \]
som tolkes slik:
\begin{center}
\begin{tabular}{rcl}
Total variasjon & = & variasjon forklart ved faktor $A$  \\
                & + & uforklart (tilfeldig) variasjon.   
\end{tabular}
\end{center}
Vi merker oss at
\[ S^2=\frac{Q_0}{n-r}. \]
Her er $n-r$ det såkalte frihetsgradtall knyttet til $Q_0$.
For å gjøre $Q_A$ sammenlignbar med $S^2$, må vi også
dividere $Q_A$  med sitt frihetsgradtall, som er $r-1$.
En egnet testobservator er derfor forholdet
\footnote{Motivasjon av $F$ : Ved å bruke forhold og ikke
differens oppnår vi at testobservatoren ikke avhenger av eventuelle
endringer av måleenheten, f.eks. fra gram til kilo, centimeter til
meter.  Divisjon med frihetsgradtallene er rimelig fordi 
 $EQ_A/(r-1) \geq {\sigma}^2$, med likhet hvis og bare hvis nullhypotesen
 er riktig, mens $EQ_0/(n-r)= {\sigma}^2$ uansett om nullhypotesen er riktig
 eller ikke, noe som samtidig indikerer rimeligheten av å forkaste $H_A$
når $F$ er stor.}

\[ F=\frac{Q_A/(r-1)}{Q_0/(n-r)} \]
Dersom forholdet $F$ er stort gir dette grunn til å forkaste 
nullhypotesen om at alle forventningene er like.  Dette er en såkalt
{\em $F$-observator}. Under forutsetning av
normalfordelte observasjoner er sannsynlighetsfordelingen til $F$, dersom
nullhypotesen er riktig, en såkalt {\em Fisherfordeling} eller kortere
en {\em $F$-fordeling}.  Denne fordeling dukker også opp i en rekke
andre sammenhenger, den er grundig studert av statistikere, og det er
utarbeidet tabeller som gir arealer under $F$-kurver, 
se Tabell~\ref{tab:F_kurver_Fraktiler_5_prosent} og~\ref{tab:F_kurver_Fraktiler_1_prosent} 
i Appendiks~\ref{app:fordelngstabeller}.
Ved hjelp av slike tabeller kan vi lage tester tilpasset et gitt
 signifikansnivå, eventuelt beregne $P$-verdier. I den foreliggende
situasjon dreier det seg om $F$-fordelingen med frihetsgradtall
$(r - 1, n - r)$. La oss se på utskrift fra en analyse av dataene ovenfor.
\footnote{ Dataene leses fra en fil med kolonnefelter, ett for responsen Y,
og ett for faktoren A, der kodetall 1,2,3 brukes for kategoriene.} 
\begin{center} \framebox[11cm]{\begin{minipage}{10cm}\rule{0cm}{0.5cm}
\tt
 >> READ 'eks11.3' Y A\\
 >> ANOVA Y A ; MEANS A. \\
 \begin{tabular}{lrrrrr}
\multicolumn{3}{l}{ Analysis of Variance Table}& & & \\
 Source   &  DF  &      SS  &      MS  &    F   &     P \\
 Factor A &   2  &   2.929  &   1.465  & 6.15   & 0.015 \\
 Error    &  12  &   2.860  &   0.238  &        &        \\
 Total    &  14  &   5.789  &          &        &
\end{tabular}
\begin{center}\addtolength{\tabcolsep}{-0.5\tabcolsep}
 \begin{tabular}{rrrcl}
      A & N &   Mean & StDev&   95\% Conf.Int.Means (Common S=0.488)  \\
      1 & 5 &  3.860 & 0.577&  \hspace{3.05cm} (-------*-------)  \\
      2 & 5 &  3.520 & 0.507&  \hspace{2.05cm} (-------*-------)  \\  
      3 & 5 &  2.800 & 0.354&  (-------*-------)                    \\
        &   &        &       & -+---------+---------+---------+- \\
\multicolumn{4}{l}{} &2.40\hspace{1.1cm}3.00\hspace{1.1cm}3.60\hspace{1.1cm}4.20\\ \mbox{}
\end{tabular}
\end{center}
\end{minipage}} \end{center}

\noindent Utskriften har i hovedsak gitt oss størrelsene nevnt ovenfor,
og starter med den såkalte {\em variansanalyse-tabellen} (ANOVA-tabellen).
I denne har vi kvadratsummer (SS=Sum of Squares), frihetsgradtall
(DF=Degrees of Freedom) og forholdene (MS=SS/DF=Mean Sum of Squares) og beregnet
 $F=6.15$ med tilhørende $P=0.015$. Vi ser at vi kan forkaste hypotesen
om like forventede kvaliteter med 5\% signifikansnivå, men ikke med
1\% signifikansnivå. 

Vi ser forøvrig at antakelsen om felles varians ikke er urimelig, 
se også Oppgave~18. Dersom hypotesen
om like forventninger forkastes, innbyr utskriften til å sammenligne
konfidensintervallene parvis, med sikte på rangering av metodene.
Metode nr.1 peker seg ut som bedre enn metode nr.3, mens metode nr.2 hverken
 kan påstås bedre enn nr.3 eller dårligere enn nr.1. Det er
ingen grunn til å ekskludere metode nr.2 fra den videre vurdering,
spesielt ikke dersom den er billigere enn metode nr.1. Den observerte forskjell
kan godt skyldes tilfeldigheter, og man må i alle fall vurdere om den
estimerte forskjell har praktisk betydning. I denne forbindelse ville det
kanskje være bedre å estimere alle forventede forskjeller
 med tilhørende feilmarginer, etter formelen ovenfor. 

Det er imidlertid et problem knyttet til alle slike parvise betraktninger,
idet vi ikke har noen totalgaranti mht. risiko for feilkonklusjoner for de\\
$r(r-1)/2$ sammenligninger som gjøres (her 3).
Det finnes teorier for {\em multiple sammenligninger},
som kan gi slike garantier, som medfører bruk av
noe større sikkerhetsfaktorer enn $t$-tabellen foreskriver.
En rekke forslag til metode for multippel sammenligning tilbys (Fisher, Tukey,
Dunnett etc.),
som avviker noe mht. hvilke garantier som gis. Slike metoder gis ofte
som opsjoner i aktuell programvare for variansanalyse.


\section{To-faktor modeller}

Vi utvider nå diskusjonen til observasjoner gruppert ved to faktorer,
nedenfor kalt $A$ og $B$.\\

\begin{eksempel}{Produksjonsmetoder og råstoff}
I Eksempel 3 studerte vi virkningen av ulike produksjonsmetoder på
kvali\-teten av et produkt.  Anta at det også er ulike muligheter
m.h.t. bruk av råstoff.  Vi har da to faktorer, produksjonsmetode ($A$)
og råstoff ($B$).  Anta at det er $r$=3 produksjonsmetoder og $s$=4 
ulike råstoffkvaliteter, vi har da i alt $r\cdot s$ = 12 mulige
faktorkombinasjoner.  La oss i første omgang tenke oss at vi
produserer en artikkel for hver faktorkombinasjon, og måler 
kvaliteten av hver av disse.  La $Y_{ij}$ være kvaliteten av 
artikkelen produsert med metode nr.$i$ med råstoff 
nr.$j$, $(i=1, 2, \ldots, r; j=1, 2, \ldots, s)$.  Anta at resultatene
ble som følger:

%\begin{center}
\begin{tabular}{|r|rrrr|r|} \hline
$Y_{ij}$  &   $j$=1    &    2    &    3    &    4    &   Gj.snitt \\ \hline    
$i$= 1    &     5.4    &   4.3   &   3.7   &   3.2   &       4.15 \\
     2    &     6.3    &   6.0   &   5.7   &   4.5   &       5.63 \\ 
     3    &     5.3    &   5.8   &   4.8   &   4.0   &       4.98 \\ \hline
Gj.snitt  &     5.67   &   5.37  &   4.73  &   3.90  &       4.92 \\ \hline
\end{tabular}
%\end{center}
\end{eksempel}% Innsatt av HJS; krever avklaring
Vi har her beregnet den gjennomsnittlige kvalitet som er observert
for de ulike metoder (linjevis) og de ulike råstoffer (kolonnevis),
samt det totale gjennomsnitt, som her ble 4.92.  Det kan se ut som 
metode nr.2 og råstoff nr.1 er den beste kombinasjonen.  Vi trenger
imidlertid er nærmere vurdering av usikkerheten ved en slik 
konklusjon.

Vi vil lage en modell for eksperimentet ovenfor som tar omsyn til at
de ulike produksjonsmetodene kan gi ulik kvalitet for ulike typer 
råstoff, dvs. at forventet kvalitet $EY_{ij}$ er en funksjon både
av $i$ og $j$.  La oss skrive $EY_{ij} = \mu_{ij}$.  Nå kan det
være rimelig å anta at $\mu_{ij}$ kan splittes opp i to
komponenter, en som skyldes produksjonsmåte og en som skyldes
råstoff, og at disse to faktorene virker additivt, dvs. at vi kan skrive

\[ EY_{ij}=\mu +{\alpha}_i+{\beta}_j \mbox{\ \ \ }
                   (i = 1, 2, \ldots , r; j = 1, 2, \ldots ,s)\]
Her er $\mu$ et gitt kvalitetsnivå, mens ${\alpha}_i$ kan 
oppfattes som et tillegg (evt. fradrag) i kvalitet som skyldes 
produksjonsmåte nr.$i$, og som gjør seg gjel\-dende uansett råstoff,
og ${\beta}_j$ kan oppfattes som et tillegg i forventet kvalitet som 
skyldes råstoff nr.$j$, og som gjør seg gjeldende uansett 
produksjonsmåte.  Vi kan uten tap av generalitet anta 
$\sum_i{\alpha}_i = 0$ og $\sum_j{\beta}_j = 0$.  Kvalitets/nivået
$\mu$ blir da å oppfatte som et gjennomsnitt tatt over de ulike 
metoder og ulike råstoff.  I tillegg til antakelsene ovenfor vil
vi anta at alle $Y_{ij}$'ene er uavhengige med samme varians ${\sigma}^2$.

I modellen er $\mu, {\alpha}_1, {\alpha}_2, \ldots, {\alpha}_r,
{\beta}_1, {\beta}_2, \ldots, {\beta}_s$ og ${\sigma}^2$ ukjente
parametre.  Vi ønsker å estimere disse parametrene, samt teste
hypoteser om dem.  La oss innføre følgende notasjon for søyle- og
radgjennomsnitt:

\[ \bar{Y}_{\cdot j}=\frac{1}{r}\sum_{i=1}^{r}Y_{ij},  \mbox{\ \ \ }
              \bar{Y}_{i\cdot}=\frac{1}{s}\sum_{j=1}^{s}Y_{ij} \]
og for det totale gjennomsnitt
\[ \bar{Y}=\frac{1}{rs}\sum_{i=1}^{r}\sum_{j=1}^{s}Y_{ij}=
            \frac{1}{r}\sum_{i=1}^{r}\bar{Y}_{i\cdot}=
            \frac{1}{s}\sum_{j=1}^{s}\bar{Y}_{\cdot j} \]
Prikkene angir altså hvilken indeks som er summert bort.  Følgende
estimatorer virker rimelige for sine respektive parametre

\begin{eqnarray*}
  \hat{\mu}     & = & \bar{Y} \\
{\hat{\alpha}}_i & = & \bar{Y}_{i\cdot}-\bar{Y} \mbox{\ \ }(i=1,2,\ldots,r)\\
{\hat{\beta}}_j  & = & \bar{Y}_{\cdot j}-\bar{Y} \mbox{\ \ }(j=1,2,\ldots,s)
\end{eqnarray*}
Som estimator for forventet kvalitet $\mu_{ij}$ brukes da 

\[ {\hat{\mu}}_{ij}=\hat{\mu}+{\hat{\alpha}_i}+{\hat{\beta}_j}=
  \bar{Y}+({\bar{Y}}_{i\cdot}-\bar{Y})+({\bar{Y}}_{\cdot j}-\bar{Y}) \]
Det er lett å vise at estimatorene ovenfor alle er forventningsrette
for sine respektive parametre.  Deres varianser kan også finnes, men de
utelates her (Oppgave~15).  Størrelsen ${\hat{\mu}}_{ij}$
kan tenkes brukt som estimator for forventningen $\mu_{ij}$.  I lys
av situasjonen i eksemplet tolkes denne slik:  Vi tar utgangspunkt i
det totale gjennomsnitt $\bar{Y}$, til dette legges to 
korreksjonsfaktorer, den ene korrigerer for at produksjonsmetode nr.$i$
er brukt, den andre korrigerer for at råstoff nr. $j$ er brukt.
Differensen

\[ Y_{ij}-{\hat{\mu}}_{ij}=Y_{ij}-{\bar{Y}}_{i\cdot }-
                          {\bar{Y}}_{\cdot j}+\bar{Y} \]
kan tolkes som det avvik som ikke lar seg forklare ved den antatte
modell, og som tilskrives tilfeldigheter.  Betrakt 
kvadratsummen av alle slike avvik

\[  Q_0=\sum_{i=1}^{r}\sum_{j=1}^s
          {(Y_{ij}-{\bar{Y}}_{i\cdot}-{\bar{Y}}_{\cdot j}+\bar{Y})}^2 \]
Det kan vises at $EQ_0 = (r-1)\cdot (s-1){\sigma}^2$, slik at

\[ S^2=\frac{1}{(r-1)(s-1)}Q_0           \]
er en forventningsrett estimator for ${\sigma}^2$.  Som i 
tilfellet med en-faktormodellen i forrige avsnitt kan vi splitte 
kvadratsummen for den totale variasjon

\[  Q=\sum_{i=1}^{r}\sum_{j=1}^s {(Y_{ij}-\bar{Y})}^2 \]
i flere kvadratsummer.  Her blir 

\[    Q = Q_A + Q_B + Q_0    \]
der $Q_0$ er definert ovenfor og 

\[  Q_A=s\cdot \sum_{i=1}^{r}{(\bar{Y}_{i\cdot}-\bar{Y})}^2 \mbox{\ \ \ \ \  }
         Q_B=r\cdot \sum_{j=1}^{s}{(\bar{Y}_{\cdot j}-\bar{Y})}^2   \]
Dette tolkes slik
\begin{center}
\begin{tabular}{rcl}
Total variasjon & = & variasjon forklart ved faktor $A$  \\
                & + & variasjon forklart ved faktor $B$  \\
                & + & uforklart (tilfeldig) variasjon.   
\end{tabular}
\end{center}
Dette indikerer metoder til å teste interessante hypoteser.  
Hypotesen $H_A: {\alpha}_1 = {\alpha}_2 = \cdots = {\alpha}_r = 0$ 
svarer til at faktor $A$ overhodet ikke innvirker på forventningen,
mens $H_B: {\beta}_1 = {\beta}_2 = \cdots = {\beta}_s = 0$ svarer til
at faktor $B$ ikke innvirker på forventningen.  Den første 
hypotesen forkastes dersom $Q_A$ er stor i forhold til $Q_0$, den siste 
dersom $Q_B$ er stor i forhold til $Q_0$. Egnede testobservatorer til å
teste hypotesene $H_A$ og $H_B$ er henholdsvis 

\[ F_A=\frac{Q_A/(r-1)}{Q_0/(r-1)(s-1)} \mbox{\ \ \ }
                         F_B=\frac{Q_B/(s-1)}{Q_0/(r-1)(s-1)} \]
som, dersom observasjonene er normalfordelte og de respektive 
nullhypoteser holder, er $F$-fordelt med frihetsgradtall henholdsvis\\
$((r-1),(r-1)\cdot (s-1))$ og $((s-1),(r-1)\cdot (s-1))$.

Ved analysen av de data som er gitt ovenfor, fikk vi følgende 
variansanalysetabell: 
\begin{center} \framebox[11cm]{\begin{minipage}{10cm}\rule{0cm}{0.5cm}
\tt
 >> READ 'eks11.4' Y A B\\
 >> ANOVA Y A B \\
 \begin{tabular}{lrrrrr}
\multicolumn{3}{l}{ Analysis of Variance Table}& & & \\
 Source   &  DF  &      SS  &      MS  &    F   &     P \\
 Factor A &   2  &   4.372  &   2.186  & 15.87  & 0.004 \\
 Factor B &   3  &   5.497  &   1.832  & 13.27  & 0.005 \\
 Error    &   6  &   0.828  &   0.138  &        &        \\
 Total    &  11  &  10.697  &          &        &         \\
\mbox{}
\end{tabular}
\end{minipage}} \end{center}

\noindent Vi ser at begge faktorene $A$ og $B$ er signifikante, og vi kunne
i likhet med utskriften i Eksempel 3 be om mer detaljerte opplysninger om
hva forskjellen mellom gruppene besto i med sikte på å rangere
de ulike produksjonsmetoder og de ulike råstoffer m.h.t. kvalitet.
Dette leder til problemstillinger av typen multiple sammenligninger
som ble omtalt i forrige avsnitt.

Modellen ovenfor brukes også i situasjoner der vi bare er interessert
i den ene faktoren, f.eks. produksjonsmetode, mens den annen faktor
er en forstyrrende faktor som det lønner seg å ta omsyn til i
analysen. Eksempelvis dersom råstoff varierer fra parti til parti,
og vi har bare nok til å produsere tre enheter pr. parti.

Det sentrale poeng ved modellen som er presentert ovenfor,
er antakelsen om additive effekter.  Det er imidlertid mulig at en 
faktor i kombinasjon med en annen gir et bidrag til responsen ut over
det rent additive, dette kalles {\em samspill}.  Best kjent er kanskje
dette fenomen i farmasi, hvor to legemidler (faktorer) som i ulike 
doseringer hver for seg har gunstig virkning, har en helt annen, kanskje
verdiløs eller ødeleggende virkning dersom de brukes sammen.  I
medisin kalles dette synergistiske effekter.

Muligheten for samspill må ikke overses. Det kan være en
kilde til feilkonklusjoner, og kan representere uutnyttede muligheter,
se f.eks Kapittel 15.4 om kvalitetsforbedringer.

Dersom samspill ikke kan utelukkes, kreves
det en annen modell, og en annen analysemetode enn den vi har
presentert ovenfor.  Innenfor en slik modell kan man bl. a. teste om 
samspill ikke er til stede, dette vil imidlertid kreve mer enn en 
observasjon pr. faktorkombinasjon. La oss gi et kort innblikk i modeller
med samspill.

Anta at vi i Eksempel 4 har $m$ observasjoner for hver faktorkombinasjon,
og la $Y_{ijk}$ være kvaliteten av artikkel nr.$k$ produsert etter
metode nr.$i$ med råstoff nr.$j$. En modell med samspill mellom faktorene
 kan skrives slik

\[ EY_{ijk}={\mu}_{ij}=\mu +{\alpha}_i+{\beta}_j+{\phi}_{ij} \]
der ${\phi}_{ij}$ representerer samspillet som en korreksjon til
hovedeffektene ${\alpha}_i$ og ${\beta}_j$ til de to faktorene.
Den naturlige estimator for ${\mu}_{ij}$ er nå gjennomsnittet 
 $\bar{Y}_{ij}$ av de m observasjonene for faktorkombinasjonen 
 $(i,j)$. De øvrige parametrene estimeres i henhold til 
oppstillingen
\begin{center} \addtolength{\tabcolsep}{-0.4\tabcolsep}
\begin{tabular}{ccccccccc}
 $\hat{\mu}_{ij}$&=&$\hat{\mu}$&+&$\hat{\alpha}_i$&+
                            &$\hat{\beta}_j$&+&$\hat{\phi}_{ij}$ \\
 $\bar{Y}_{ij}$  &=&$\bar{Y}$&+&$(\bar{Y}_{i\cdot}-\bar{Y})$&+
     &$(\bar{Y}_{\cdot j}-\bar{Y})$&+
   &$(\bar{Y}_{ij}-\bar{Y}_{i\cdot}-\bar{Y}_{\cdot j}+\bar{Y})$
\end{tabular}
\end{center}
der prikker angir hvilken indeks som er summert bort i tillegg til $k$.
I likhet med situasjonen uten samspill, kan vi splitte opp total variasjon slik

\[ Q=Q_A+Q_B+Q_{AB}+Q_{0} \]
der det ekstra leddet $Q_{AB}$ er variasjon forklart ved samspill mellom de 
to faktorene. På tilsvarende måte som ovenfor kan vi nå motivere

\[ F_{AB}=\frac{Q_{AB}/(r-1)(s-1)}{Q_0/rs(m-1)} \]
som testobservator for $H_{AB}$ : ${\phi}_{ij}=0$, dvs. at det ikke er
 samspill mellom de to faktorene. Dersom $H_{AB}$ er riktig, er $F_{AB}$
 $F$-fordelt med frihetsgradtall $((r-1)(s-1),rs(m-1))$.

Som ovenfor gis de nødvendige beregninger for å gjennomføre 
testen i form av en ANOVA-tabell, se Oppgave~17.


\section{Varianskomponent-modeller}
Den teori som er utviklet ovenfor kan generaliseres til modeller for
mer kompliserte situasjoner med mer enn to faktorer og ulike former
for samspill.  Felles for disse modellene er at responsen blir
dekomponert etter de faktorer man mener kan forklare denne, først
tar en med de såkalte egenvirkningene av hver faktor, deretter tas
med samspill mellom ulike faktorer i den grad man mener at disse kan
gjøre seg gjeldende.  Innenfor den valgte modell kan en så 
stille spørsmål om disse egenvirkningene og samspillene som en
kan få besvart ved å samle inn observasjoner.  De nødvendige
observasjoner vil kunne avhenge av modellen.

Modeller av dette slaget kalles gjerne {\em dekomponeringsmodeller}, og
ved dekomponering er det aktuelt med to typer komponenter, såkalte
{\em forventningskomponenter og varianskomponenter}.  Vi har ovenfor
bare sett eksempler på det første.  Vi betraktet der de ulike 
faktorkombinasjoner som gitte.  Det kan imidlertid tenkes problemstillinger
der faktorkombinasjoner blir bestemt som del av eksperimentet.  En slik
faktor er det mer rimelig å representere med en varianskomponent.
La oss illustrere de to typer komponenter med et eksempel. \\


\begin{eksempel}{Produksjonsmetoder og arbeidere}
Anta at ulike produksjonsmetoder (faktor $A$) blir prøvd ut med 
arbeidere med ulike ferdigheter (faktor $B$).  I noen situasjoner vil
de arbeidere som deltar være gitt, f. eks. alle arbeidere i
bedriften, eventuelt de som er villig til å delta.  Vi vil anta at
det ikke er samspill mellom metode og arbeider. Dette er ikke alltid 
realistisk, det kan f. eks. tenkes at en god metode av enkelte møtes 
med skepsis fordi den er ny, og disse gjør en dårlig innsats av
den grunn.  Vi vil derfor bruke to-faktor modellen beskrevet i avsnitt
11.4.  Denne svarer til at responsen $Y_{ij}$ for $i$'te metode og 
$j$'te arbeider dekomponeres slik
\[     Y_{ij} = \mu + {\alpha}_i + {\beta}_j + \mbox{tilfeldig variasjon}  \]
der ${\alpha}_i$ og ${\beta}_j$ er konstanter, og tilfeldig variasjon
er stokastiske variable med forventning null og samme varians 
${\sigma}^2$.  Inferensen gjelder nå de deltakende personer og en
bør være forsiktig med å trekke konklusjoner utover disse.
Imidlertid kan det tenkes at vi nettopp ønsker å trekke generelle 
konklusjoner om ferdigheten til arbeidere for de ulike 
produksjonsmetodene på grunnlag av observerte ferdigheter for et
utvalg av arbeidere.  Skal dette gjøres, bør de arbeidere som får
prøve seg representere et tilfeldig utvalg fra den populasjon av
arbeidere man ønsker å trekke konklusjoner om.  I denne situasjon
kan det være aktuelt å bruke en litt annen dekomponering enn
ovenfor, nemlig

\[   Y_{ij} = \mu + {\alpha}_i + B_j + \mbox{tilfeldig variasjon}  \]
der vi har erstattet konstanten ${\beta}_j$ med en stokastisk 
variabel $B_j$.  Denne skal reflektere det forhold at arbeider nr. $j$
har et bidrag til responsen, utover rent tilfeldige variasjoner, som 
er stokastisk bestemt, nemlig ved loddtrekningen av denne arbeideren
fra populasjonen til å delta i eksperimentet.  Det kan være 
rimelig å anta at alle $B_j$'ene er uavhengige med samme 
forventning, som uten tap av generalitet kan antas lik null, og med
samme varians ${\sigma}_0^2$.  Denne variansen gir uttrykk for
variasjon i ferdigheter blant arbeiderne i populasjonen, uansett
metode.  I denne modellen vil faktor $A$ være en forventningskomponent
og faktor $B$ en varianskomponent.  Hva den siste angår, vil vi
være interessert i å teste om  variansen ${\sigma}_0^2$ er
null. Dette betyr at systematisk individuell variasjon mellom
arbeidere i popu\-lasjonen ikke er til stede, og at observerte forskjeller
mellom arbeidere i sin helhet kan forklares som tilfeldig variasjon.

Dataanalyse på grunnlag av en modell med varianskomponenter vil
i grove trekk kunne gjennomføres som i de tilsvarende modeller med
forventningskomponenter dersom modellen er uten samspill.  Generelt
vil imidlertid dataanalysen kunne arte seg forskjellig i de to
situasjonene, forskjellen består i hva vi veier kvadratsummen for effekten
opp mot, dvs. nevneren i $F$-observatoren. I tilfellet med to faktorer er
oppbyggingen av $F$-observatorene og frihetsgradtallene gitt ved:

\begin{center} \small \addtolength{\tabcolsep}{-0.3\tabcolsep}
\begin{tabular}{|c|c|c|c|c|c|} \cline{4-6}
 \multicolumn{3}{c|}{} &\multicolumn{3}{c|}{Kvadratsum i nevner}\\ \cline{1-3}
Effekt&Frihetsgrad&Kvadratsum&Alle faste&$A$ fast $B$ tilf.&Begge tilf.\\ \hline
 $A$&$(r-1)$&$Q_A$&$Q_0$&$Q_{AB}$&$Q_{AB}$ \\
 $B$&$(s-1)$&$Q_B$&$Q_0$&$Q_{0}$&$Q_{AB}$ \\
 $AB$&$(r-1)(s-1)$&$Q_{AB}$&$Q_0$&$Q_0$&$Q_0$ \\ \hline
\end{tabular}
\end{center}
Når vi ønsker å undersøke virkningen av en faktor alene 
eller i kombinasjon med andre faktorer, vil det eksperiment vi utfører
sammen med vår forhåndsviten bestemme modellen.  På den 
annen side vil kunnskap om egenskapene ved ulike modeller være til
hjelp ved valg av eksperiment.  Innen faget eksperimentplanlegging
står dekomponeringsmodeller sentralt i diskusjonen når det
dreier seg om observasjoner av typen gradert respons. 
 
Teorien ovenfor er i hovedsak utviklet for situasjoner der observasjonene
er tatt i henhold til en plan med et valgt antall observasjoner pr.
faktorkombinasjon. I samfunnsfag studeres ofte utvalgsdata fra en
populasjon, der en i ettertid grupperer observasjonene, f.eks. etter
kjønn og andre karakteristika.  En er da interessert i å
se på gruppeforskjeller mht. målte variable, og variansanalyse
etter forventningskomponentmodellen peker seg ut som en mulig analysemåte. 
Antall observasjoner av hvert slag er imidlertid nå tilfeldig bestemt, og
dette kan være et problem ved analysen.

For det første kan forutsetningen om uavhengige observasjoner være 
tvilsom, og en variansanalyse er da knapt mer enn en eksplorativ metode.
For det andre kan det være et praktisk problem ved
at tilgjengelige varians\-ana\-lyse\-pro\-gram\-mer forutsetter en balansert
 situasjon, dvs. like mange observasjoner pr. faktorkombinasjon. 
Et alternativ er da å omforme variansanalysemodellen til
regresjonsmodell ved bruk av egnede indikatorvariable, og bruke teorien i
Kapittel 12. 

Det fins fordelingsfrie alternativer til variansanalyse med en og to faktorer,
dvs. som ikke forutsetter normalfordelte observasjoner. Mest kjent er
Kruskal-Wallis test og Friedmans test som er såkalte rangmetoder,
der man erstatter observasjonene med deres rang etter størrelse.
Slike metoder er også aktuelle dersom observasjonene
selv er på en rangeringsskala, slik tilfellet ofte er i
meningsmålinger, bl.a. i markedsforskning.
\end{eksempel}% Innsatt av HJS; krever avklaring

\newpage
\section{Oppgaver}
\small
\begin{enumerate}
\item  Et forskerteam mener å ha laget et gjødningsprodukt $(A)$
tilpasset hveteproduksjon som gir høyere avkastning enn det produkt 
$(B)$ som fagfolk til nå har anbefalt for formålet.  En
frittstående institusjon har fått i oppdrag å undersøke
om det er grunn til å satse på de nye produktet.  Undersøkelsen
er lagt opp slik:  Et rektangulært jordstykke deles opp i 9 blokker
med 2 like store ruter i hver blokk:
\begin{center}
\setcounter{telle}{0}
\begin{picture}(150,35)
\put(30,0){\line(1,0){90}} 
\put(30,10){\line(1,0){90}} \put(30,20){\line(1,0){90}}
\multiput(30,0)(10,0){10}{\line(0,1){20}}
\put(53,35){\makebox{Blokknr.}}
\multiput(32,25)(10,0){9}{\addtocounter{telle}{1} \thetelle}
\end{picture}
\end{center}
Jordstykket sås deretter til med hvete.  Innen hver blokk velges
så en av rutene som gjødsles med $A$, mens den andre ruten 
gjødsles med $B$.  Ved innhøstingen veies opp produsert
hvetekvantum i hver av de 18 rutene.  Resultatet i kg pr. rute ble:
\begin{center} \addtolength{\tabcolsep}{-0.1\tabcolsep}
\begin{tabular}{lccccccccc}
Blokk nr.:    &    1  &  2  &  3  &  4  &  5  &  6  &  7  &  8  &  9 \\
Gjødsel $A$: &   27.2& 27.9& 25.8& 26.6& 25.4& 25.1& 24.0& 23.2& 22.5 \\
Gjødsel $B$: &   26.3& 26.8& 24.5& 25.9& 26.0& 24.6& 24.2& 22.8& 22.2
\end{tabular}
\end{center}
\begin{itemize}
\item[(a)]  Bruk en konstant-effekt modell og estimer forventet forskjell
     i avling med $A$ og $B$ og rapporter resultatet.  
\item[(b)]  Vurder sannsynligheten for feilestimering av ulik 
størrelsesorden.  Tyder det observerte resultat på en reell 
forskjell mellom $A$ og $B$?   
\item[(c)]  Utfør en $t$-test for hypotesen om at $A$ og $B$ er like
     gode mot alternativet at $A$ er bedre enn $B$.  Hva blir
     konklusjonen dersom 5\% signifikansnivå er valgt?  
\end{itemize}

\item Det er foreslått to ulike produksjonsmåter for et
produkt som skal tåle en viss belastning, metode nr. 1 faller noe
dyrere enn metode nr. 2, men man mener at nr. 1 gir et bedre resultat.
Det prøveproduseres $n_1 = n_2 = 6$ artikler etter de to metodene.
Disse ble utsatt for en økende belastning og tålte en belastning
i kg på henholdsvis
\begin{center}
\begin{tabular}{lcccccc}
Metode 1:  &   13.2  &  11.7  &  12.3  &  12.9  &  13.4  &  11.2 \\
Metode 2:  &   10.8  &  11.5  &  12.7  &  12.5  &  11.0  &  12.9
\end{tabular}
\end{center}
\begin{itemize}
\item[(a)] Gjør rede for valg av modell.
\item[(b)] Estimer forskjellen i forventet kvalitet og rapporter 
           resultatet.
\item[(c)] Vurder sannsynligheten for feilestimering av ulik 
           størrelsesorden.  Tyder det observerte resultat på en
           reell forskjell mellom de to metodene.
\item[(d)]  Utfør en $t$-test for hypotesen om at de to metodene er 
           like gode mot alternativet at metode 1 er bedre.  Hva blir
           konklusjonen dersom signifikansnivået er valgt lik 5\%?
\end{itemize}

\item 
La situasjonen være som beskrevet i Eksempel 1, der vi studerte 
tidsforbruket ved arbeidsoperasjon med og uten spesialverktøy.
\begin{itemize}
\item[(a)] Lag konfidensintervall konfidensnivå 95\% for forskjellen
           i forventet tid uten og med spesialverktøy.
\item[(b)] Gjennomfør en $t$-test med signifikansnivå 5\%  for
        hypotesen om at forskjellen i forventet tid er null mot alternativet
          at spesialverktøy medfører reduksjon i forventet tid.
\item[(c)] Kjenner du alternative testmetoder som kan brukes i denne
           situasjonen?
\end{itemize}

\item
La situasjonen være som i Eksempel 2, der vi studerte
tidsforbuket ved arbeidsoperasjon med og uten spesialverktøy.
\begin{itemize}
\item[(a)] Lag konfidensintervall konfidensnivå 95\% for forskjellen
           i forventet tid uten og med spesialverktøy.
\item[(b)] Gjennomfør en $t$-test med signifikansnivå 5\%  for
        hypotesen om at forskjellen i forventet tid er null mot alternativet
          at spesialverktøy medfører reduksjon i forventet tid.
\item[(c)] Verifiser informasjonen i utskriften i teksten. Kan $P$-verdien i
           (b) finnes i utskriften.
\item[(d)] Kjenner du en alternativ testmetode som kan brukes i den
           foreliggende situasjon?
\end{itemize}

\item
La situasjonen være som i Oppgave~\ref*{kap:kontrollerte}.10.

\begin{itemize}
\item[(a)] Vurder om en konstant-effekt modell kan brukes her, i så
           fall
\item[(b)] estimer forventet forskjell i toppfart med og uten Luriol
           og rapporter resultatet.
\item[(c)] Utfør en $t$-test for å teste hypotesen om at Luriol
           er uten effekt mot alternativet at Luriol øker toppfarten.
           Hva blir konklusjonen dersom 5\% signifikansnivå er valgt.
\end{itemize}

\item
Samme eksamen avholdes på to læresteder.  Det deltok i alt
$n_1$ = 10 studenter fra sted nr. 1 og $n_2$ = 8 studenter fra sted
nr. 2.  Karakterene ble gitt av de samme sensorer på en skala fra
0 til 9, der 9 er beste karakter.  Vi ønsker å teste om det er 
noen forskjell mht. forventet prestasjonsnivå hos studentene på
de to lærestedene.  Resultatet ble
\begin{center}
\begin{tabular}{ccccccccccc}
Sted nr. 1: &  2  &  4  &  0  &  7  &  6  &  5  &  5  &  4  &  3  &  5 \\
Sted nr. 2: &  5  &  6  &  6  &  4  &  3  &  7  &  5  &  2  &     &
\end{tabular}
\end{center}
\begin{itemize}
\item[(a)] Diskuter valg av testmetode.
\item[(b)] Er forskjellen signifikant på 5\%?
\end{itemize}

\item
For å teste levetiden av to typer batterier ($A$ og $B$) for bruk
i lommeradioer er valgt ut 12 batterier av hver type fra et parti
ferske batterier fra de respektive leverandører.  24 radioer er til
disposisjon og det velges for sikkerhets skyld tilfeldig hvilke radioer
som blir forsynt med henholdsvis $A$ og $B$ batterier.  Undersøkelsen
ga følgende resultat mht. levetid i timer:
\begin{center}\scriptsize \addtolength{\tabcolsep}{-0.3\tabcolsep}
\begin{tabular}{lrrrrrrrrrrrr} 
Type A: & 7.2 & 11.4 & 14.3 & 13.4 & 9.3 & 14.1 & 16.3 & 9.5 & 2.7 & 6.6
        & 18.2 & 11.5 \\
Type B: & 15.8 & 11.3 & 16.2 & 16.0 & 18.4 & 13.9 & 18.5 & 9.2 & 17.4 
        & 18.0 & 13.2 & 15.3
\end{tabular}
\end{center}
\begin{itemize}
\item[(a)] Tyder tallmaterialet på at forutsetningene for bruk av
           $t$-testen for to utvalg er oppfylt?
\item[(b)] Foreslå en alternativ testmetode som bør kunne brukes.\\
           Hint: Se Kapittel 10.4.
\end{itemize}

\item
En bedrift har markedsført en ny type deodorantsåpe $A$, og i
reklamen hevdes det at den hindrer svettelukt mer effektivt enn visse
andre såper på markedet.  En forbrukerorganisasjon har bedt
om belegg for denne påstanden, og får tilsendt et kortfattet 
utdrag av en rapport.  Denne beskriver bl. a. et eksperiment med 
72 deltakere der, over en testperiode på 6 dager, 36 deltakere 
brukte $A$, mens resten brukte den til nå ledende såpe $B$.
Hver dag, umiddelbart før ``den daglige vask" ble det foretatt en
duftprøve (av duft som var 24 timer gammel).  Duften ble målt
på en skala fra 0 (dårligst) til 8 (best).  Rapporten oppgir
at gjennomsnittlig duftskår i testperioden var henholdsvis 5.89 for
de med såpe $A$ og 5.74 for de med såpe $B$.  Konklusjonen som
oppgis er at, med et signifikansnivå på 1\%, så er $A$
signifikant bedre enn $B$.  Anta at du er bedt om å komme med en 
kritisk vurdering av disse opplysningene.  

\begin{itemize}
\item[(a)] Lag modell for eksperimentet og finn ut hva standardavviket
           til hver enkelt luktprøve må ha vært, og vurder om
           dette synes rimelig.
\item[(b)] Anta at det ikke er grunn til å tvile på
           opplysningene.  Diskuter hvorvidt den problemformulering som
           er brukt er relevant i for\-bru\-ker\-sam\-men\-heng.
\end{itemize}

\item
Et firma vurderer tre nye prosedyrer for å stoppe en prosess som gå
ut av kontroll. Det simuleres uhell 15 ganger, og en operatør prøver 5
 ganger med hver metode. Tiden til prosessen stoppes ble målt i sekunder,
med resultat
\begin{center}
\begin{tabular}{cccccc}
   Prosedyre nr. 1:  &  78  &  63  &  85  &  52  &  75 \\
   Prosedyre nr. 2:  &  70  &  68  &  58  &  63  &  55 \\
   Prosedyre nr. 3:  &  82  &  78  &  67  &  59  &  87 
\end{tabular}
\end{center}
\begin{itemize}
\item[(a)] Diskuter bruk av enfaktor-modellen i denne situasjonen.
\item[(b)] Estimer forventet tid for de tre prosedyrene, og ranger dem
         dersom det synes rimelig.
\item[(c)] Er det grunn til å forkaste hypotesen om at det ikke er
           noen forskjell mellom de tre prosedyrene mht. forventet tid?
             Bruk 5\% signifikansnivå.
\item[(d)] Foreslå en alternativ analysemåte som tar omsyn til
           rekkefølgen av observasjonene for hver prosedyre. \\
           Kan det fortsatt være elementer som er forstyrrende for
           analysen?
\end{itemize}

\item
Fire typer piggdekk prøves ved bremseprøver på en
glattkjøringsbane. Bilen har samme dekktype på alle fire hjul, og
etter 6 prøver med samme type, skiftes over til neste type.
Bilen kjøres en gitt hastighet og bremselengden måles i meter.
Resultatet ble
\begin{center}
\begin{tabular}{ccccccc}
Dekktype nr. 1:  &  32  &  35  &  30  &  41  &  32  &  37 \\
Dekktype nr. 2:  &  47  &  31  &  33  &  42  &  49  &  38 \\
Dekktype nr. 3:  &  27  &  30  &  32  &  24  &  37  &  38 \\
Dekktype nr. 4:  &  26  &  18  &  15  &  29  &  35  &  22
\end{tabular}
\end{center}
\begin{itemize}
\item[(a)] Er det grunn til å forkaste hypotesen om at forventet
           bremselengde er den samme for de fire dekktypene?
            Bruk 5\% signifikansnivå.
\item[(b)] Hvis det er grunnlag for det, sammenlign dekktypene parvis,
           og eventuelt ranger dem.
\end{itemize}
Anta isteden at det ble brukt to biler og tre prøver for hver dekktype
før dekkskift 
\begin{itemize}
\item[(c)] Endre dette noe på konklusjonene ovenfor?\\
          (De tre første observasjonene er for bil nr.1, de tre siste for bil nr.2)
\item[(d)] Er det ting du ville gjort annerledes, evt. andre faktorer som
            burde vært med i eksperimentopplegget?
\end{itemize}

\item Et forbrukerkontor vurderer 4 vaskemidler, og har gjennomført 
prøvevask av ensartet tilgriset tøy, og anbefalt dosering av vaskemidlet.
Resultatet av vasken vurderes på en skala etter gitte kriterier.
Det er brukt 3 ulike maskiner i testen, og for at ikke en enkelt maskins effektivitet
skal påvirke resultatet, er det utført en vask for hver kombinasjon
vaskemiddel/maskin. Resultater ble
\begin{center}
\begin{tabular}{l|rrrr}
\multicolumn{1}{c}{} &\multicolumn{4}{c}{Vaskemiddel} \\
        &  1 &  2 &  3 &  4  \\ \hline
Maskin 1& 15 & 17 & 18 & 12  \\
Maskin 2& 13 & 16 & 20 &  7  \\
Maskin 3& 21 & 22 & 25 & 19  \\ \hline
\end{tabular}
\end{center}
\begin{itemize}
\item[(a)] Utfør en variansanalyse med to faktorer. 
\item[(b)] Test hypotesen om at vaskemidlene er like gode.
\item[(c)] Ranger vaskemidlene ( høy skår best).
\item[(d)] Hva ville konklusjonen ha blitt med variansanalyse med en faktor,
           dvs. vi ser bort fra hvilken maskin som ble brukt.
\item[(e)] Hva slags analyse ville du foretatt dersom det forelå to
           observasjoner pr kombinasjon vaskemiddel/maskin for
           (i) samme dosering (ii) to ulike doseringer.
\end{itemize}

\item
La situasjonen være som i Oppgave~9 men anta isteden at fire
operatører A, B, C og D  får prøve alle tre prosedyrene og at 
det foreligger en observasjon for hver kombinasjon menneske/prosedyre.
Anta at resultatet ble
\begin{center}
\begin{tabular}{c|cccc}
              &   A    &   B    &   C    &   D \\ \hline
Prosedyre nr. 1  &   76   &   55   &   70   &   65 \\
Prosedyre nr. 2  &   70   &   58   &   62   &   52 \\
Prosedyre nr. 3  &   88   &   72   &   75   &   61 \\ \hline
\end{tabular}
\end{center}
\begin{itemize}
\item[(a)] Diskuter muligheten for å bruke en to-faktor modell
           uten samspill.
\item[(b)] Estimer parametrene i denne modellen, og lag en ANOVA-tabell.
\item[(c)] Test hypotesen om at prosedyrene er like mht. forventet tid.
\end{itemize}
Anta vi har 5 observasjoner for hver kombinasjon prosedyre og 
operatør.
\begin{itemize}
\item[(d)] Hvilke nye muligheter for analyse åpner seg nå?
\item[(e)] Diskuter sider ved problemet som ikke er belyst ved forventet tid.
\end{itemize}

\item Forklar at det i konstant-effekt modellen er nok å anta at vi har
      $n$  uavhengige observasjonspar med samme kovarians.
\item Situasjonen i Eksempel 2 kan også studeres ut fra teorien for
      variansanalyse med en faktor. Vis at resultatet av analysen blir det
      samme ved
\begin{itemize}
\item[(a)] å utføre analysen.
\item[(b)] å påvise at $F=T^2$, og deretter sammenligne $F$-fordelingen
           med $(1,n-2)$ frihetsgrader med $t$-fordelingen med $n-2$
          frihetsgrader.
\end{itemize}
\item Vis at estimatorene i to-faktormodellene er forventningsrette og beregn
      deres varianser.
\item Anta at vi i toutvalgs-modellen i Eksempel 2 kjente variansene
      ${\sigma}_1^2$ og ${\sigma}_2^2$ i de to gruppene,
\begin{itemize}
\item[(a)] Hvordan ville du lage konfidensintervaller og utføre testingen
      dersom ${\sigma}_1^2$ og ${\sigma}_2^2$ er (i) like (ii) forskjellige ?
\item[(b)] Vis at dersom vi bare har råd til å ta $n$ observasjoner
          i alt, så er det optimalt å velge 
     \[ n_1=n\frac{{\sigma}_1}{{\sigma}_1+{\sigma}_2} \mbox{\ \ \ \ }
                 n_2=n\frac{{\sigma}_2}{{\sigma}_1+{\sigma}_2}      \]
	Hint : Ulikhetene i Oppgave~\ref*{kap:stokastiske}.40 kan nyttes.
\item[(c)] Undersøk tilgjengelig programvare mht. muligheten for å lage
       konfidensintervaller og utføre testingen dersom variansene er
      ukjente, men antatt forskjellige.
\end{itemize}
\item
La situasjonen være som i Eksempel 4, men anta at vi har to observasjoner
pr. kombinasjon produksjonsmetode og råstoff med resultat:
\begin{center}
\begin{tabular}{|r|cccc|} \cline{2-5}
\multicolumn{1}{c|}{}   &\multicolumn{4}{c|}{Råstoff nr.}  \\
\multicolumn{1}{c|}{}   &    1    &    2    &    3    &    4   \\ \hline 
Metode 1  & 5.4 4.8 & 4.3 4.9 & 3.7 3.8 & 3.2 3.7  \\
       2  & 6.3 6.0 & 6.0 5.8 & 5.7 5.2 & 4.5 4.9  \\
       3  & 5.3 5.5 & 5.8 5.4 & 4.8 4.5 & 4.0 4.8  \\ \hline
\end{tabular}
\end{center}
\begin{itemize}
\item[(a)] Tolk utskriften nedenfor. Er det samspill mellom faktorene?
\item[(b)] Hvordan vil du gjøre analysen dersom vi antar fravær av samspill?
\item[(c)] Forsøk å finne programvare som gjør variansanalyse, og
           som gir supplerende informasjon: multiple sammenligninger osv.
\end{itemize}
\begin{center} \framebox[10cm]{\begin{minipage}{9cm}\rule{0cm}{0.5cm}
\tt
 >> READ 'eks11.4b' Y A B\\
 >> ANOVA Y A B A*B\\
 \begin{tabular}{lrrrrr}
\multicolumn{3}{l}{ Analysis of Variance Table}& & & \\
 Source    &  DF  &      SS  &      MS  &    F   &     P \\
 Factor A  &   2  &   7.106  &   3.553  & 34.80  & 0.000 \\
 Factor B  &   3  &   7.385  &   2.462  & 24.11  & 0.000 \\
 Factor A*B&   6  &   0.514  &   0.086  &  0.84  & 0.563 \\
 Error     &  12  &   1.225  &   0.102  &        &        \\
 Total     &  23  &  16.230  &          &        &        \\
\mbox{}
\end{tabular}
\end{minipage}} \end{center}
\item  Nedenfor er en utskrift fra en sjekk om like varianser i ulike grupper
       for tallmaterialet i Eksempel 3. Hva forteller denne?
\begin{center} \framebox[10cm]{\begin{minipage}{9cm}\rule{0cm}{0.5cm}
\tt

>> Read 'Eks11.3' Y A\\
>> Variance check Y A; Confidence 95.0. \\
Homogeneity of Variance\\
Confidence intervals standard deviations (Bonferroni)\\
 \begin{tabular}{ccccl}
  Lower  &   Sigma  &   Upper  & n & Factor Levels\\[0.1cm]
0.311863 & 0.577061 &  2.22105 & 5 & 1 \\
0.273974 & 0.506952 &  1.95120 & 5 & 2 \\
0.191072 & 0.353553 &  1.36079 & 5 & 3
\end{tabular} 
Bartlett's Test (assumes normality): P-value : 0.655\\
Levene's Test (also for non-normal): P-value : 0.673\\[0.2cm]
\end{minipage}} \end{center}



\end{enumerate}
\normalsize
