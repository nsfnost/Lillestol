
\setcounter{page}{291}
\mbox{} \vspace{2cm}
\begin{center}
{\huge Innledning}
\end{center}
\vspace{1cm}

Annen del av denne boka inneholder supplerende stoff som kan leses etter
interesse og behov.  De enkelte kapitlene er, så langt det er 
mulig, bygd opp slik at de kan leses uavhengig av hverandre.  Vi vil
i første rekke ta sikte på å gi en innføring i en del
statistiske problemområder av betydelig praktisk interesse.  I et
avsluttende kapittel vil vi ta opp analyse av beslutningsproblemer
under usikkerhet.

I Kapittel 7 og 8 presenterte vi en del av de sentrale begreper i
forbindelse med statistisk inferens.  Som illustrasjoner studerte vi
noen inferensproblemer for en del enkle men svært viktige modeller.
Dette stoff utgjør det som enhver bruker av statistiske metoder bør
kjenne til.  Leseren skulle nå være i stand til å 
gjennomføre enklere statistiske resonnementer på egen hånd.
Mange problemområder krever imidlertid mer raffinerte modeller.  
Disse vil i svært mange tilfeller være generaliseringer i
ulike retninger av de modellene som er presentert ovenfor.  De ervervede
kunnskaper kan derfor betraktes som et utgangspunkt for videre studier
i metoder som har relevans for ens eget interesseområde.  De vil
også være et brukbart grunnlag for kommunikasjon med en 
fagstatistiker.  

De emner som blir tatt opp i de følgende kapitler tas ikke opp i full
bredde.  For å kunne gjøre dette måtte vi ha utvidet 
teorikunnskapene ytter\-ligere, noe som antakelig ville stille leseren 
på en tålmodighetsprøve.  Vi vil nøye oss med smakebiter
på aktuelle problemstillinger og analyse\-metoder med den hensikt å
gjøre leseren oppmerksom på disse.  For en grundigere veiledning
må vi vise til spesiallitteratur om de ulike emner.  Det er også
en rekke emner som faller utenfor den rammen vi har valgt her.

La oss avslutte denne innledningen med noen generelle 
betraktninger : \\
Formålet med en statistisk undersøkelse vil som
regel være å etablere kunn\-skaper om sammenhenger og 
årsaksforhold, og vi ønsker helst å kunne trekke generelle
konklusjoner, dvs. konklusjoner som har verdi ut over dataene selv.
Kanskje ønsker vi å etablere ny teori eller kanskje ønsker
vi å få bekreftet eller avkreftet allerede eksisterende teori.

En effektiv analyse av statistiske data, utover det rent beskrivende,
vil være avhengig av den forhåndsviten vi har om hvordan
dataene er framkommet, dvs. vi bør støtte oss til en modell.  Da
først vil en være i stand til å vurdere påliteligheten
av de analysemetoder som brukes og de konklusjoner som trekkes ved 
hjelp av disse.  En blir da fort klar over at det ikke er like\-gyldig
hvordan våre data er innsamlet.  De bør samles inn på en
slik måte at det ikke medfører altfor vanskelige tolkningsproblemer,
bl.a bør man i størst mulig grad eliminere mulige feilkilder.
Muligheten for å gjøre dette vil variere med det anvendelsesområde 
det er tale om, ytterpunktene er på den ene side laboratorieforsøk
der data innsamles under veldefinerte forhold, på den annen
side data som har fanget vår interesse, og der omstendighetene ved
datainnsamlingen ikke er kjent i detalj.

Det er vanskelig, for ikke å si umulig, å lage en enhetlig teori som
kan fange opp alle problemstillinger innen disse ytterpunkter, dvs.
lage en ``kokebok" med standardmetoder. Hvert praktisk problem kan ha
sine særtrekk som gjør at det ikke uten videre kan presses inn i den
ramme som anvises av en lærebok.

En effektiv dataanalyse vil som regel være avhengig av det teorigrunnlag
som fins på vedkommende fagområde, spesielt vanskelig er situasjonen 
på områder der teorigrunnlaget er usikkert, diskutabelt eller mangler
helt. Ofte kan data være forenlig med flere ulike teorier, og statistikk
alene kan derfor ikke gi svar uten reservasjoner. ``Facts from figures" kan
være uoppnåelig.

De fleste statistiske inferensmetoder som er tilgjengelige i litteraturen 
er utviklet til bruk under veldefinerte forutsetninger, og ansvar for å
tenke over om disse med rimelighet kan sies å være oppfylt ligger
hos brukeren. Det syndes nok mye her, spesielt ved bruk av de 
``standardmetoder" som er lett tilgjengelige i statistisk programvare. 
Generelt råd : Bruk ikke metoder basert på forutsetninger som
du ikke forstår rekkevidden av, søk heller
assistanse av en fagstatistiker, helst allerede før data er innsamlet.
