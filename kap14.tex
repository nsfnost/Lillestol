\chapter{Utvalgsundersøkelser}
\label{kap:utvalgsundersokelser} % Opprinnelig kapittelnr: 14

\section{Innledning}

En utvalgsundersøkelse er karakterisert ved følgende:  Vi har en 
populasjon bestående av $N$ elementer (objekter eller individer), og vi
ønsker informasjon om disse elementene.  Dersom $N$ er stor vil det ofte 
være uhensikts\-messig å observere alle elementene i populasjonen,
det kan f.eks. være for dyrt, tidkrevende eller praktisk umulig.  Isteden
trekkes et utvalg på et mindre antall elementer $n$, og konklusjoner
om populasjonen som helhet trekkes på grunnlag av dette utvalget.

Den enkleste problemstilling av denne typen har vi når de $N$ elementene
i populasjonen kan beskrives med verdier, henholdsvis $v_1, v_2, \ldots,
v_N$, og vi ønsker å trekke slutninger om gjennomsnittet $\bar{v}$
på grunnlag av et utvalg av elementer fra populasjonen.  For det tilfellet
at utvalget er trukket tilfeldig vil lotterimodellen beskrevet i Kapittel 8.2
kunne brukes til å vurdere usikkerheten vedrørende de konklusjoner som
trekkes.

I praksis vil en ofte ikke kunne realisere et tilfeldig utvalg.  Dette kan ha
forskjellige årsaker, den mest vanlige er at det vil bli for kostbart og
tidkrevende å gjennomføre.  Statistikere har da foreslått en rekke
alternative framgangsmåter som kan komme til anvendelse under ulike 
omstendigheter.  Felles for disse er at de bygger på modeller som er
generaliseringer i ulike retninger av lotterimodellen.  En framgangsmåte
for å trekke et utvalg kalles av fagfolkene ofte for en {\em utvalgsplan}
eller {\em design}.  En design karakteriseres som god i relasjon til det 
foreliggende problem, dersom den lar seg praktisk gjennomføre innenfor
rammen av våre ressurser, og samtidig leder til tillitvekkende
konklusjoner.

Vi vil nedenfor se på noen aktuelle utvalgsplaner og analysemetoder.  Vi
vil bl.a. vise hvordan tilleggsinformasjon om populasjonen kan utnyttes til
å oppnå mer pålitelige konklusjoner.

I forbindelse med utvalgsundersøkelser er det også aktuelt å 
trekke slutninger om proporsjoner, f.eks. brøkdelen av velgere som 
favoriserer et bestemt politisk parti.  Dette kan ses på som et
spesialtilfelle av teorien i dette kapitlet, hvor hvert element i 
populasjonen  har $v$-verdi som enten er 0 eller 1, eksempelvis 0 dersom 
velgeren ikke favoriserer partiet og 1 dersom velgeren favoriserer partiet.
Da er $\bar{v}$ brøkdelen av tilhengere i populasjonen.  Imidlertid vil
estimering av proporsjoner også romme en del aspekter som ikke er dekket
av teorien nedenfor.


\section{Stratifisering}
La oss først forfølge en tankegang som kalles {\em stratifisering}:

Anta at vi skal estimere gjennomsnittet $\bar{v}$ av de $N$ ukjente verdiene
$v_1, v_2, \ldots, v_N$ i populasjonen.  Istedenfor å velge ut $n$
elementer tilfeldig fra hele populasjonen tenker vi oss at denne er delt opp
i et visst antall delpopulasjoner, kalt {\em strata}, med henholdsvis $N_1,
N_2, \ldots, N_k$ elementer i hver, slik at $N_1 + N_2 + \cdots + N_k = N$,
og at vi trekker et tilfeldig utvalg på henholdsvis $n_1, n_2, \ldots,
n_k$ elementer fra hvert stratum slik at $n_1 + n_2 + \cdots + N_k = n$.  En
viktig grunn for å foreta stratifisering har vi dersom det er visse typer
elementer i populasjonen som vi ønsker skal være representert i 
utvalget, en annen grunn følger av teorien nedenfor.

Vi merker oss at vi kan skrive

\[ \bar{v}=\frac{N_1}{N}{\bar{v}}_1+\frac{N_2}{N}{\bar{v}}_2+ \cdots
                          +  \frac{N_k}{N}{\bar{v}}_k \]
\noindent der ${\bar v}_i$ er gjennomsnittet av $v$-verdiene for elementene
i stratum nr. $i$.  En rimelig estimator for ${\bar v}_i$ vil være
gjennomsnittet ${\bar Y}_i$ av $v$-verdiene fra de utvalgte elementene fra
stratum nr. $i$, dvs.

\[  {\bar{Y}}_i=\frac{1}{n_i}=\sum_{j=1}^{n_i}Y_{ij}     \]

\noindent der $Y_{ij}$ betegner $j$'te observasjon fra stratum nr. $i$.
Som estimator for gjennomsnittet av $v$-verdiene i hele populasjonen
$\bar{v}$ er det rimelig å bruke

\[ \tilde{Y}=\frac{N_1}{N}{\bar{Y}}_1+\frac{N_2}{N}{\bar{Y}}_2+ \cdots
                            \frac{N_k}{N}{\bar{Y}}_k .\]

\noindent Innen hvert stratum velges et tilfeldig utvalg, og vi kan her bruke 
resultatene som er utviklet for lotterimodellen, nemlig
\[ E{\bar{Y}}_i={\bar{v}}_i      \]
\[ var{\bar{Y}}_i=\frac{N_i-n_i}{N_i-1}\cdot \frac{{\sigma}_i^2}{n_i}
                                     \approx \frac{{\sigma}_i^2}{n_i} \]
\noindent der tilnæmelsen er god når $n_i$ er liten i forhold til
$N_i$.  Her er ${\sigma}_i^2$ variansen svarende til en enkelt observasjon fra
stratum nr. $i$.  Av dette følger at

\[   E\tilde{Y} = \bar{v} \mbox{\ \ dvs. estimatoren er forventningsrett.} \]

\noindent Vi antar at trekningene for de ulike strata skjer uavhengig av
hverandre. Herav følger at

\[ var\tilde{Y} \approx \frac{N_1^2}{N^2} \cdot \frac{{\sigma}_1^2}{n_1}+
         \frac{N_2^2}{N^2} \cdot \frac{{\sigma}_2^2}{n_1}+ \cdots
         \frac{N_k^2}{N^2} \cdot \frac{{\sigma}_k^2}{n_k}   \]

\noindent Det er ofte rimelig å trekke ut et antall fra hvert stratum
proporsjonalt med antall elementer totalt i stratumet, dvs. velge
\footnote{Vi må i praksis velge $n_1, n_2, \ldots, n_k$ som de
nærmeste positive heltall med sum $n$.}

\[ n_1=n \cdot \frac{N_1}{N},\;\;\; n_2=n \cdot \frac{N_2}{N}, \ldots ,\;\;\;
                               n_k=n \cdot \frac{N_k}{N}        \]

\noindent Dette kalles {\em proporsjonal utvelging}.  I denne situasjon
forenkles varians\-formelen til

\[ var\tilde{Y} \approx \frac{1}{n}(\frac{N_1}{N} \cdot {{\sigma}_1^2}+
                           \frac{N_2}{N} \cdot {{\sigma}_2^2}+\cdots +
                                      \frac{N_k}{N} \cdot {{\sigma}_k^2}) \]

\noindent La oss sammenligne estimatoren $\tilde{Y}$ i tilfellet proporsjonal
utvelging med estimatoren $\bar{Y}$ basert på vanlig tilfeldig utvalg
på $n$ elementer fra hele po\-pu\-la\-sjonen.  Vi husker at $\bar{Y}$ også var
forventningsrett og med varians

\[   var\bar{Y} \approx \frac{{\sigma}^2}{n}        \]

\noindent For det tilfellet at gjennomsnittet og variansen er den samme for
hvert stratum, ${\sigma}_1^2 = {\sigma}_2^2 = \cdots = {\sigma}_k^2 =
{\sigma}^2$, vil høyresiden i de to siste varians\-formlene være like.
I alle andre tilfeller vil høyresiden for $var\tilde{Y}$ være mindre
enn høyresiden for $var\bar{Y}$. Mer konkret viser det seg at

\[ var\bar{Y} \approx var\tilde{Y}+\frac{N_1}{N}{({\bar{v}}_1-\bar{v})}^2     
                       +\frac{N_2}{N}{({\bar{v}}_2-\bar{v})}^2 + \cdots    
               +\frac{N_k}{N}{({\bar{v}}_k-\bar{v})}^2     \]
\noindent som viser at reduksjonen i varians blir større ettersom
${\bar v}_1, {\bar v}_2, \ldots, {\bar v}_k$ avviker mer og mer fra det totale
gjennomsnittet $\bar{v}$.

Det er interessant å legge merke til at ved proporsjonal utvelging er
estimatoren $\tilde{Y}$ også lik gjennomsnittet av de observerte verdier, 
og det som gir variansreduksjonen er altså at utvalgsplanen er en annen.

I praksis kan resultatene ovenfor utnyttes slik:  På forhånd deles
po\-pu\-la\-sjonen i strata slik at elementer innen hvert stratum er mest
mulig homogene, eller ekvivalent, at gjennomsnittene av $v$-verdiene fra
stratum til stratum er avvikende. Siden vi ikke kjenner eksakte $v$-verdier
til elementene i populasjonen må dette gjøres etter beste skjønn, 
f.eks. ut fra visse kjennetegn som er knyttet til disse elementene. Reduksjonen
av varian\-sen ved bruk av $\tilde{Y}$ som estimator istedenfor $\bar{Y}$ vil
avhenge av i hvilken grad vi lykkes å utføre denne stratifiseringen
i henhold til disse ønsker, men poenget er at vi har intet å tape ved
å forsøke en stratifisert utvelging.  Noen eksempler:

\begin{itemize}
\item  Ved en undersøkelse av forbruksvaner hos en gruppe av befolkningen,
       kan vi stratifisere m.h.t. kjønn, bosted og/eller inntektsnivå.
\item  Ved undersøkelse av gjennomsnittsverdi av kjøp i et varehus, kan
       vi stratifisere m.h.t. hvilken avdeling de enkelte kjøp fant sted.
\item  Ved undersøkelse av det gjennomsnittlige antall kuer hos 
       gårdbrukere over hele landet, kan vi stratifisere etter fylke, 
       gårdens størrelse, om brukeren har biinntekt eller ikke.
\item  I revisjon kan en stratifisere etter bokført verdi.
\end{itemize} 

\noindent I disse situasjonene har vi en viss tro på at $v$-verdiene innen
hvert stratum er mer homogene enn i populasjonen som helhet.  La oss regne
på et konkret enkelt eksempel, i en realistisk anvendelse vil selvsagt
populasjonen være mye større.\\

\begin{eksempel}{Røking}
På en arbeidsplass er det 10 røkere, 6 menn og 4 kvinner.  Vi ønsker
å anslå gjennomsnittlig antall sigaretter som er røkt pr. røker
en bestemt dag på grunnlag av et utvalg av $n = 5$ røkere.  Anta at vi
deler populasjonen i to strata etter kjønn, idet vi av erfaring vet at
kvinnene røker gjennomgående noe mindre enn mennene.  Anta at 
populasjonen består av følgende tall som er ukjente for oss.
\begin{center}
\begin{tabular}{lcccccc}
Stratum 1: &   8   &   10   &   5   &   7   &   4   &   9 \\
Stratum 2: &   4   &    2   &   4   &   6   &       &
\end{tabular}
\end{center}
\noindent Ved proporsjonal utvelging trekkes tilfeldig 3 menn og 2 kvinner.
En tabell over tilfeldige tall lar oss observere person nr. 1, 3 og 6 fra
stratum nr. 1, og person nr. 1 og 4 fra stratum nr. 2.  Vi får 
\begin{eqnarray*}
 \bar{Y}_1&=&(8+5+9)/3=7.3 \\
 \bar{Y}_2&=&(4+6)/2=5.0
\end{eqnarray*}
\noindent slik at

\[ \tilde{Y}=\frac{6}{10}\bar{Y}_1+\frac{4}{10}\bar{Y}_2=
             \frac{6}{10}7.3+\frac{4}{10}5.0=6.4 \]
 
\noindent Anta at vi isteden velger de $n = 5$ tilfeldig fra hele populasjonen.
En tabell over tilfeldige tall lar oss da observere person 2, 3, 4 og 6 blant
mennene og person nr. 2 blant kvinnene, slik at vi får

\[ \bar{Y}=(10+5+7+9+2)/5=6.6            \]

\noindent På grunn av teorien ovenfor vil vi foretrekke å bruke den
første framgangsmåten.  Vi ser for øvrig at gjennomsnittet i
populasjonen virkelig er 5.9, noe som vi selvfølgelig ikke vet ut fra det
utvalg som er foretatt.\\
\end{eksempel}

Når en skal vurdere usikkerheten ved et beregnet estimat, ønsker vi
å bruke standardavviket til estimatoren.  Igjen må dette estimeres.
Vi kan estimere variansen innen hvert stratum på vanlig måte, dvs.
vi beregner

\[  S_i^2=\frac{1}{n_i}\sum_{j=1}^{n_i}{(Y_{ij}-\bar{Y}_i)}^2      \]

\noindent hvor $Y_{ij}$ er den $j$'te observasjon fra stratum nr. 1.  Med utgangspunkt
i formelen for var $\tilde{Y}$ ovenfor kan vi få en estimator for denne
variansen ved å erstatte ${\sigma}_i^2$  med $S_i^2$.  Vi får da
estimatoren

\[  S^2 =\frac{N_1^2}{N^2} \cdot \frac{S_1^2}{n_1}+
         \frac{N_2^2}{N^2} \cdot \frac{S_2^2}{n_1}+ \cdots
         \frac{N_k^2}{N^2} \cdot \frac{S_k^2}{n_k}   \]

\noindent som i tilfellet med proporsjonal utvelging reduseres til

\[ S^2 = \frac{1}{n}(\frac{N_1}{N} \cdot {S_1^2}+
                           \frac{N_2}{N} \cdot {S_2^2}+\cdots +
                                      \frac{N_k}{N} \cdot {S_k^2}) \]

\noindent Som estimator for standardavviket brukes da $S$.

Man kan spørre om det er mulig å finne en estimator som reduserer
variansen ytterligere i forhold til proporsjonal utvelging.  Det viser seg
at $var\tilde{Y}$ er minst mulig når utvalgsstørrelsen $n_i$ velges,
ikke proporsjonal med $N_i$, men proporsjonal med $N_i{\sigma}_i$.  Dette 
kalles {\em optimal utvelging}.  Slik utvelging krever imidlertid kunnskap
om ${\sigma}_i$, noe som sjelden er tilgjengelig.  Dersom vi vet at 
variabiliteten innen et bestemt stratum er stor i forhold til andre strata,
kan det være lønnsomt å øke utvalgsstørrelsen fra dette
i forhold til vanlig proporsjonal utvelging.  Imidlertid må det ekstreme
forhold til før en tjener noe vesentlig på å gjøre dette.

I noen situasjoner kan kostnadene ved å velge ut elementer variere fra
stratum til stratum.  Det finnes også teorier for bestemmelse av
utvalgs\-størrelser hvor en tar hensyn til dette.

Vi har for å forenkle teorien ovenfor tatt utgangspunkt i formler hvor
korreksjonsfaktorer av typen $(N - n)/(N - 1)$ er utelatt og de funne
resultater gjelder da strengt tatt bare når utvalgsstørrelsene innen
hvert stratum er lite i forhold til det totale antall i hvert stratum.  Dette
vil også være tilfellet i mange anvendelser.  Imidlertid vil den
konklusjon at det kan være lønnsomt å stratifisere ha mer generell
gyldighet, og i situasjoner der utvalgsstørrelsene forholdsvis ikke er 
små, kan det være aktuelt å ta med korreksjonsfaktorer av typen
ovenfor.



\section{Klyngeutvalg}

I enkelte situasjoner er det ikke hensiktsmessig å gjennomføre et
rent tilfeldig utvalg, ei heller et stratifisert utvalg.  Dette kan skyldes
praktiske omstendigheter og/eller høye kostnader.  Eksempelvis dersom en
ønsker å gjennomføre en forbrukerundersøkelse som gjelder 
husstandene i en større by, vil det ikke være hensiktsmessig å la
en intervjuer besøke utvalgte husstander spredt over et stort område.
Isteden tenker man seg at byen deles opp i mindre områder, f.eks. i
kvartaler.  Det velges så først tilfeldig et visst antall kvartaler.
Fra hvert av de utvalgte kvartaler velges så tilfeldig et visst antall
husstander som intervjues.  Dette kalles et {\em klyngeutvalg}, her er det 
tale om et to-trinns klyngeutvalg.  På første trinn trekkes et utvalg
av såkalte {\em primære utvalgsenheter}.  Fra hver av de utvalgte
primærenheter velges et antall {\em sekundære utvalgsenheter}, og
disse danner grunnlag for den statistiske undersøkelsen.

Det viser seg at estimater basert på klyngeutvalg generelt ikke er så
pålitelige som tilsvarende estimator basert på et rent tilfeldig
utvalg med den samme utvalgsstørrelse.  Imidlertid kan et klyngeutvalg
vise seg å være mindre tidkrevende og billigere, slik at vi med
gitte ressurser kan gjennomføre undersøkelsen med en betydelig 
større utvalgsstørrelse enn med et rent tilfeldig utvalg.  Eksempelvis
vil både tid og penger være spart ved å intervjue familier som
bor i visse konsentrerte områder enn at alle er spredt utover et større
område.

Vi vil ikke her bygge opp noen teori for klyngeutvalg.  En slik teori vil 
avhenge av hvordan de primære utvalgsenhetene velges ut, enten at hver 
primær utvalgsenhet har samme sannsynlighet for å bli valgt ut, eller
at sannsynligheten for å velge en bestemt primær utvalgsenhet er
proporsjonal med antall sekundære utvalgsenheter som denne består av.
Den siste framgangsmåten sikrer at alle sekundære utvalgsenheter har 
samme sannsynlighet for å bli valgt ut, eksempelvis at alle husstander i
byen har samme sjanse for å komme med i undersøkelsen.  Det viser seg
at teorien i det siste tilfellet også blir betydelig enklere enn i det 
første.

Tanken om klyngeutvalg kan også utvides til utvalg som tas i mer enn to
trinn, og teorier for tre-trinns utvalg er tilgjengelig i litteraturen.  I
praksis kan det også være aktuelt å bruke klyngeutvalg i
kombinasjon med stratifisering.  Kan hende er bebyggelsen i byen ulik fra
distrikt til distrikt, noen bydeler med hovedsaklig eldre leiegårder,
andre med nyere blokkbebyggelse og småhus, igjen andre med større 
frittliggende eneboliger.  Det kan tenkes at forbruksmønsteret er noe ulikt
i de tre typer boligområder, og for å få holdepunkter i denne 
retning er det ønskelig at hver type er skikkelig representert i utvalget.
Vi kan da betrakte de tre typer boligområder som tre strata og innen hvert
stratum foretas så et klyngeutvalg etter de retningslinjer som er 
beskrevet ovenfor.

Grunnen til at man på ulike trinn i en utvalgsundersøkelse anbefaler
å velge utvalgsenheter tilfeldig er at det sikrer en viss rettferdighet.
Dette kan bidra til å fjerne mulige feilkilder som ikke er åpenbare
for de som gjennomfører undersøkelsen.  Videre danner en slik 
framgangsmåte grunnlag for en teori som setter oss i stand til å
vurdere påliteligheten av de konklusjoner som trekkes av undersøkelsen.
I praksis hender det imidlertid ofte at man er nødt til å gå på
akkord med målsettingen om at utvalg skal velges tilfeldig.  Som regel
skyldes dette omstendigheter av praktisk og/eller økonomisk natur.  En 
bør imidlertid alltid forstå konsekvensene av den utvalgsmåte som
brukes, om den kan innebære mulige feilkilder, samt hvilke 
usikkerhetsmarginer en må regne med i eventuelle konklusjoner.  I 
litteraturen er beskrevet en rekke mulige framgangsmåter for
gjennomføring av utvalgs\-under\-søkelser ut fra totalvurdering av ideelle
målsettinger og praktisk gjennomførbarhet.  Den utvalgsplan som bør
velges i en konkret situasjon vil avhenge av omstendighetene.


\section{Estimeringsmetoder}

I de foregående avsnitt har vi diskutert ulike utvalgsplaner.  Med en gitt
plan vil det, når observasjoner foreligger til analyse, være mulig
å velge mellom ulike analysemetoder.  Vi har ovenfor studert de enklest
tenkelige estimatorer, og vil her se på noen alternative estimatorer som
er aktuelle i situasjoner der det foreligger tilleggsinformasjon om hver 
enkelt utvalgsenhet.  Det viser seg at disse estimatorene under visse 
omstendigheter, kan gi betydelig presisjonsgevinst i forhold til de vanlige 
estimatorene.

Anta at til hvert av de $N$ elementene i populasjonen er knyttet to verdier,
en $v$-verdi og en $w$-verdi, slik at populasjonen består av $N$ tallpar

\[ (v_1,w_1),(v_2,w_2),\ldots ,(v_N,w_N) \]

\noindent Anta at summen av $w$-verdiene er kjent, dvs. $\bar{w}$ er kjent. Vi
ønsker fortsatt å estimere gjennomsnittet $\bar{v}$ av $v$-verdiene
på grunnlag av et utvalg på $n$ elementer fra populasjonen.  Vi
observerer da både $v$-verdien og $w$-verdien for hvert element i 
utvalget, og spørsmålet er om vi ikke også kan utnytte $w$-verdiene
når vi estimerer $\bar{v}$.  Dette synes i hvert fall rimelig i 
situasjoner der vi vet at det er en viss samvariasjon mellom $v$ og $w$.  La
oss kort skissere et par eksempler hvor dette er aktuelt:

\begin{itemize}
\item  En kommune ønsker å anslå den gjennomsnittlige 
       silokapasiteten for alle gårdsbruk i kommunen på grunnlag av 
       et utvalg gårder.  Man kjenner det totale dyrkede areal på
       alle gårdene, både de som er med og ikke er med i utvalget.

\item  En revisor ønsker å anslå gjennomsnittlig innestående
       på en rekke konti på et bestemt tidspunkt på grunnlag av
       et utvalg konti.  Hun kjenner innestående på alle konti på
       samme tidspunkt året før.
\end{itemize}

Vi vil nedenfor anta at vi trekker et vanlig tilfeldig utvalg.  La oss 
innføre følgende notasjon:  $Y_i$ betegner fortsatt $v$-verdien til 
det $i$'te element i utvalget, mens den tilhørende $w$-verdien betegner
vi med $X_i$.  Vårt observasjonsmateriale består derfor av $n$
observasjonspar

\[ (Y_1, X_1), (Y_2, X_2), \ldots ,(Y_n, X_n) \]

\noindent samt gjennomsnittet av $w$-verdiene i hele populasjonen $\bar{w}$.
 En aktuell estimator for $\bar{v}$ er nå den såkalte
 {\em rateestimatoren}, gitt ved 

\[  R=\bar{Y} \cdot \frac{\bar{w}}{\bar{X}}       \]

\noindent Denne estimatoren kan i lys av det første eksemplet ovenfor
 motiveres slik:  Dersom tilfeldigvis de utvalgte gårdene
gjennomgående har
mindre dyrket areal enn gjennomsnittet $\bar{w}$ vil en kunne vente at 
også silokapasiteten for disse i gjennomsnitt er lavere enn $\bar{v}$,
dvs. at vi underestimerer $\bar{v}$.  Tilsvarende vil vi kunne overestimere
$\bar{v}$ dersom de utvalgte gårder gjennomgående har dyrket areal
over gjennomsnittet.  Opplysning om i hvilken retning en feilestimering
trolig går, får vi ved å sammenligne det observerte 
gjennomsnittlige dyrkede areal $\bar{X}$ med det totale gjennomsnitt $\bar
{w}$, som er antatt kjent.  Den vanlige estimatoren  $\bar{Y}$ kan
da korrigeres ved å multiplisere med faktoren $\bar{w}/\bar{X}$.

Det kan vises at rateestimatoren er tilnærmet forventningsrett med
varians

\[ varR \approx \frac{N-n}{N-1} \cdot \frac{1}{n} \cdot \frac{1}{N}
                       \sum_{i=1}^n{(v_i-\bar{v} \frac{w_i}{\bar{w}})}^2 \]

\noindent Tilnærmelsen er best når $n$ og $N$ er stor.  På den
 annen side husker vi at

\[ var\bar{Y} \approx \frac{N-n}{N-1} \cdot \frac{1}{n} \cdot \frac{1}{N}
                       \sum_{i=1}^n{(v_i-\bar{v})}^2 \]

\noindent Variansen til $R$ vil være betydelig mindre enn variansen til
 $\bar{Y}$ straks $v$-verdiene er tilnærmet proporsjonal med de
 tilhørende $w$-verdier.  Merk at for eksakt proporsjonalitet vil det
 tilnærmede uttrykk for var$R$ reduseres til null.

En annen aktuell estimator er den såkalte {\em regresjonsestimatoren}

\[ V=\bar{Y}+\hat{\beta} \cdot (\bar{w}-\bar{X})       \]

\noindent der koeffisienten $\hat{\beta}$ er bestemt ved minste kvadraters
 metode anvendt på observasjonene i utvalget, dvs.

\[  \hat{\beta}=\frac{\sum_{i=1}^{n}(X_i-\bar{X})Y_i}
                          {\sum_{i=1}^{n}{(X_i-\bar{X})}^2} \]

\noindent For å forstå denne estimatoren trengs kanskje et
tilbakeblikk på teorien i Kapittel 8.3.  Tankegangen er imidlertid den
samme som med rateestimatoren.  De observerte $w$-verdier i utvalget kan
avdekke eventuelle skjevheter i utvalget som kan korrigeres.  Her korrigeres
ut fra forestillingen om en mulig lineær samvariasjon mellom $w$-verdier
og $v$-verdier.  Positiv $\hat{\beta}$ vil indikere positiv samvariasjon.  
Dersom vi observerer $\bar{X}$ mindre enn $\bar{w}$, vil estimatoren 
$\bar{Y}$ antakelig underestimere $\bar{v}$.  Vi ser at i dette tilfellet
vil estimatoren $V$ korrigere oppover, omvendt dersom $\hat{\beta}$ blir
negativ.  $V$ vil derfor trolig gi et bedre estimat enn $\bar{Y}$.  Teori
indikerer at regresjonsestimatoren gjennomgående har mindre varians enn
rateestimatoren i hvert fall for større utvalg.  Rateestimatorer eller
regresjonsestimatorer er ofte aktuelle i situasjoner der $v$-verdiene
representerer en størrelse knyttet til elementene i populasjonen på
et bestemt tidspunkt, mens $w$-verdiene representerer samme størrelse
for disse elementene på et tidligere tidspunkt og disse antas kjent for
alle elementene i populasjonen.

Når det gjelder estimering av varians for rateestimatoren og 
regresjonsestimatoren må vi vise til spesiallitteratur. \\

\begin{eksempel}{Konti}
La oss tenke oss $N = 10$ konti hvor innestående pr. 31.desember i to 
etterfølgende år var følgende (i tusen kroner):
\begin{center}
\begin{tabular}{ccccccccccc}
 $w_i$ : &   2  &  6  &  1  &  6  &  7  &  4  &  1  &  3  &  3  &  7 \\
 $v_i$ : &   3  &  9  &  2  &  5  &  9  &  7  &  1  &  5  &  4  &  7
\end{tabular}
\end{center}
Vi ser at $\bar{w} = 4$, og vi antar at dette er kjent.  Den andre tallrekken
antar vi er ukjent, og at vi ønsker å estimere $\bar{v}$ på
grunnlag av et utvalg på $n = 3$ konti.  En tabell over tilfeldige tall
lar oss observere konto nr. 1, 3 og 9.  Vi får da

\[ \bar{Y}=(3+2+4)/3=3.0.        \]

\noindent Vi ser at $\bar{X} = 2$, og sammenlignet med $\bar{w} = 4$ gir dette 
berettiget mistanke om at $\bar{v}$ er underestimert.  Rateestimatet blir

\[  R=3.0 \cdot \frac{4}{2}=6.0       \]

\noindent mens regresjonsestimatet blir

\[  V=3.0+1.0 \cdot (4-2)=5.0,      \]

\noindent og hvilket som bør velges er et skjønnsspørsmål.
I realistiske anvendelser vil selvsagt $N$ være mye større, dette
 må bare oppfattes som et regneeksempel med enkle tall.
\end{eksempel}

Det finnes også en rekke andre forslag til estimatorer som tar omsyn
til tilleggsinformasjon.  Under de fleste omstendigheter vil disse ikke 
være dårligere enn den vanlige estimatoren $\bar{Y}$, som ikke tar
omsyn til tilleggsinformasjon, og under spesielle omstendigheter vil de
være betydelig bedre.  Valg av estimator vil derfor være delvis
avhengig av de forestillinger vi gjør oss om den tilleggsinformasjon som
foreligger.  Det finnes versjoner av disse estimatorene også for andre
utvalgsplaner enn rent tilfeldig utvalg.

En type utvalg av betydelig interesse er såkalt proporsjonal utvelgelse,
der enhetene i utvalget trekkes med sannsynligheter proporsjonale med
sine $w$-verdier, som antas kjent. Tanken bak dette er at store $v$-verdier
har større innflytelse på resultatet enn små, og derfor viktigere
å få med i utvalget. Dersom det er korrelasjon mellom $v$ og $w$, vil
vil fremgangsmåten nettopp sikre dette, og på en slik måte at
egenskapene til estimeringsmetoder f.eks. rateestimatoren kan utledes.
Dette er et alternativ til å bruke $v$-verdiene til å stratifisere i
to eller flere grupper. 

Vi har ovenfor omtalt noen sentrale id\'{e}er fra teorien for 
utvalgsundersøkelser.  En rekke andre id\'{e}er har vi måtte la ligge,
og interesserte må søke spesiallitteratur.  Dette gjelder spesielt
de praktiske sider ved gjennomføringen av utvalgsundersøkelser,
intervjupraksis osv.


\section{Estimering av populasjonsstørrelse}

En problemstilling av en noe annen karakter enn ovenfor, er å anslå
størrelsen av en populasjon.  Dette har en viss aktualitet
 i forbindelse med økologi og ressursproblemer,
f.eks. ved å anslå størrelsen av en laksestamme eller størrelsen
av en viltbestand.  Slike undersøkelser utføres gjerne ved å merke
et visst antall individer fra populasjonen, slippe disse løs igjen, og 
senere ta et utvalg fra populasjonen, observere andelen av merkede individer,
og trekke konklusjoner om populasjonens størrelse på grunnlag av dette.

La $N$ være antall individer i populasjonen og $M$ være antall 
merkede individer, slik at brøkdelen av merkede individer i populasjonen
er $a = M/N$.  Det trekkes så et utvalg individer, og anta at 
brøkdelen merkede individer i utvalget er $\hat{a}$.  Det synes da rimelig
å anslå populasjonens størrelse $N$ ved

\[    \hat{N} = \frac{M}{\hat{a}}     \]

\noindent Nå kan en slik undersøkelse gjennomføres på ulike
måter, vi vil her kort omtale to måter:  Enten gjenfanges et bestemt
antall individer $n$ fra populasjonen og antall merkede individer $Y$ observeres,
eller man fanger individer inntil et bestemt antall merkede individer $m$
er oppnådd og observerer det antall individer $X$ som det ble nødvendig
å fange i alt.  Det første kalles {\em direkte utvelging}, det siste
{\em invers utvelging}.

La oss først betrakte direkte utvelging:  Brøkdelen av merkede 
individer i utvalget er da $\hat{a} = Y/n$, slik at estimatoren for
populasjonsstørrelsen $N$ kan skrives

\[    \hat{N} = \frac{M \cdot n}{Y}     \]

\noindent For at dette skal ha noen mening må selvsagt $Y>0$ og vi antar
at både $M$ og $n$ er valgt tilstrekkelig store til at dette i praksis er
oppfylt. Estimatoren $\hat{N}$ er ikke forventningsrett, den stokastiske $Y$ i 
nevneren kompliserer det hele.  Det kan imidlertid vises at 

\[   E\hat{N} \approx N+(N-M)/M \approx N \]

\noindent Den siste tilnærmelsen er brukbar dersom $M$ og $n$ ikke velges
altfor liten, i så fall vil altså $\hat{N}$ være tilnærmet
forventningsrett.  Videre kan det vises at variansen til $\hat{N}$ er
tilnærmet lik

\[   var\hat{N} \approx \frac{M^2}{a^4} \cdot \frac{N-n}{N-1} \cdot
         \frac{a(1-a)}{n} \approx \frac{M^2}{n} \cdot  \frac{(1-a)}{a^3} \]

\noindent Den siste tilnærmelsen gjelder når $n$ er liten i forhold
til $N$. Dette uttrykket avhenger av $a$ som er ukjent, men et brukbart
estimat for $a$ er $\hat{a}$.  Variansen til $\hat{N}$ kan derfor estimeres
ved å erstatte $a$ med $\hat{a}$ i formelen ovenfor.  Vi er dermed i stand 
til i en viss grad å vurdere usikkerheten ved den esti\-mering av
populasjonens størrelse som er foretatt.  Teorien ovenfor bygger på en
viktig forutsetning, nemlig at de merkede og de gjenfangede individene
representerer et tilfeldig utvalg fra populasjonen.\\

\begin{eksempel}{Laks}
En lakseoppdretter har laks i en dam og ønsker å anslå det totale
antall laks i dammen.  Han fanger inn $M = 30$ laks som merkes og slippes ut
igjen.  Dagen etter fanger han $n = 20$ laks og observerer 4 merkede blant
disse.  Han estimerer derfor det totale antall laks i dammen til $\hat{N} =
30\cdot 20/4 = 150$.  Et anslag for variansen til estimatoren blir ved
innsetting av $\hat{a} = 4/20 = 1/5$ lik 4500.  Dersom vi rapporterer
estimat standardavvik blir vår rapport for populasjonens størrelse

\[  150 \; \; \pm \; \; 67   \]

\noindent altså ikke altfor god sikkerhet.
\end{eksempel}

La oss isteden betrakte invers utvelging.  Brøkdelen av merkede individer
blant de gjenfangede individer er da $\hat{a} = m/X$ slik at estimatoren
for populasjonsstørrelsen kan skrives

\[  \hat{N}=\frac{M \cdot X}{m}    \]

\noindent Dersom hvert nytt gjenfanget individ er tilfeldig utvalgt fra
populasjonen, viser det seg at denne estimatoren er tilnærmet
forventningsrett, dvs.

\[  E\hat{N} \approx N      \]

\noindent Den tilhørende varians er tilnærmet gitt ved

\[ var\hat{N} \approx \frac{M^2}{m} \cdot \frac{1-a}{a^2}    \]

\noindent hvor tilnærmelsen er brukbar dersom $m$ ikke er for stor i
forhold til $M$ \footnote{Vi kan som tilnærmelse tenke oss at vi
slipper løs
igjen hvert merket individ som fanges, og ventetiden til hvert nytt merket 
individ er da geometrisk fordelt, se Kapittel 6.7, hvorav resultatet 
følger.}  Siden $a$ i dette uttrykket er ukjent må man i praksis
estimere denne variansen ved å erstatte $a$ med estimatet $\hat{a}$.\\

\begin{eksempel}{Laks}
Anta at lakseoppdretten i Eksempel 3 isteden har bestemt seg for å
velge ut laks inntil $m = 5$ merkede er tatt opp, og det viste seg at han
måtte fange 22 laks i alt for å oppnå dette.  Han estimerer
derfor det totale antall laks i dammen til $\hat{N} = 30 \cdot 22/5 =
132$.  Et anslag for variansen til estimatoren blir ved innsetting av
$\hat{a} =5/22$ lik 2692.  Dersom vi rapporterer estimert standardavvik
blir vår rapport for populasjonens størrelse

\[   132 \;\; \pm \;\; 52                           \]

\noindent For en nærmere redegjørelse av de relative fortrinn av de to
metodene samt fastlegging av utvalgsstørrelser må vi vise til 
spesiallitteratur.
\end{eksempel}


\section{Oppgaver}
\small
\begin{enumerate}
\item  En forening har $N$ = 700 medlemmer i sitt kartotek, hver med sitt
tresifrede medlemsnummer.  Hvordan vil du gå fram for å trekke et
tilfeldig utvalg på $n$= 20 medlemmer?

\item  Man ønsker å anslå gjennomsnittlig studiegjeld $\bar{v}$
for studentene på et kull av $N$ = 250 studenter på grunnlag av et
tilfeldig utvalg på $n$ = 30 studenter.  Det viste seg at gjennomsnittet
i utvalget (i tusen kroner) var 54, mens den empiriske varians var 400.

\begin{itemize}
\item[(a)]  Estimer $\bar{v}$ og rapporter resultatet.
\item[(b)]  Ved et annet studiested vet man ut fra en fullstendig 
undersøkelse at gjennomsnittlig gjeld var 48.  Gir resultatet av
utvalgsundersøkelsen ovenfor grunnlag for å påstå at gjelden
er høyere blant studentene der enn her?
\item[(c)]  Kunne vi på noe vis oppnå et bedre estimat for 
gjennomsnittlig gjeld basert på et utvalg?
\end{itemize}

\item
\begin{itemize}
\item[(a)] Påvis at en lotterisituasjon der de $N$ elementene i populasjonen
har verdier enten 1 eller 0 er det samme som en hypergeometrisk situasjon.
\item[(b)]  La $M$ være 1'ere i populasjonen og påvis at

\[ \bar{v}=\frac{M}{N}  \mbox{\ \ \ } {\sigma}^2=\frac{M}{N}(1-\frac{M}{N}) \]

\item[(c)]  La $Y$ være antall 1'ere i et utvalg på $n$ elementer og 
påvis at $\bar{Y} = Y/n$.  Bruk dette til å gi alternativ
utledning av forventningen og variansen til estimatoren $\hat{a} = Y/n$
for estimering av $a = M/N$ i en hypergeometrisk situasjon.
\item[(d)]  Finn en forventningsrett estimator for variansen til $\hat{a}$.
\end{itemize}

\item  Blant et studentkull på $N$ = 250 intervjues et tilfeldig utvalg
på $n$ = 50 studenter.  Det viste seg at 10 av disse hadde deltidsjobb.

\begin{itemize}
\item[(a)]  Estimer brøkdelen av studenter på dette kullet som har
deltidsjobb og rapporter resultatet.
\item[(b)]  Samme undersøkelse ble foretatt for et årskull av eldre 
studenter og her hadde 20 av de 50 utvalgte deltidsjobb.
Estimer og rapporter.
\item[(c)]  Gir resultatene grunnlag for å påstå at de eldre 
studentene er mer tilbøyelig til å jobbe deltid enn de yngre?
\end{itemize}

\item  I en situasjon med $k$=4 strata er antallet innen hvert stratum og
stratumstandardavvikene gitt ved henholdsvis
\begin{center}
\begin{tabular}{crrrr}
 $N_i$ :   &     400   &   300   &   200   &   100 \\
 ${\sigma}_i$ : & 10   &    20   &    20   &    30
\end{tabular}
\end{center}
\begin{itemize}
\item[(a)] Hvor mange observasjoner $n$ må velges ut i alt dersom vi bruker
proporsjonal utvelging og ønsker at estimatoren skal ha et standardavvik
lik $\Delta$ dersom
 (i) $\Delta =1$ \ \  (ii) $\Delta= 2$ \ \  (iii) $\Delta =5$

\item[(b)]  Hvor mange observasjoner må velges ut i alt dersom vi 
isteden bruker optimal utvelging?
\end{itemize}

\item  Ved rapportering av variansen til estimatoren $\tilde{Y}$ som brukes
ved stratifisert utvelging antok vi at variansen ${\sigma}_i^2$ innen hvert 
stratum ble estimert ved 
\[ S_i^2=\frac{1}{n_i}\sum_{j=1}^{n_i}{(Y_{ij}-{\bar{Y}}_i)}^2      \]
\begin{itemize}
\item[(a)]  Er denne estimatoren forventningsrett for ${\sigma}_i^2$?  Hvis
ikke, korriger den slik at den blir forventningsrett.  Hint:  Se Kapittel
8.2.
\item[(b)]  Gi en formel for den estimerte varians til $\tilde{Y}$ hvor vi
har med korreksjonsfaktoren for endelig utvalg og eventuell korreksjon fra
(a).
\end{itemize}

\item  En revisor skal anslå det gjennomsnittlige beløp $\bar{v}$ som
er innestående på $N$ = 1000 konti ut fra et utvalg på av
$n$ = 100 konti som granskes nærmere.  Revisoren ønsker å utnytte
informasjon fra året før da han gjennomgikk alle konti.  Han deler
disse i fire strata etter innestående året før.  Han har også
beregnet standardavviket ${\sigma}_i$ for hvert stratum dette året.
Resultatet var:
\begin{center}
\begin{tabular}{lrrrr}
Stratum nr.:              &      1   &     2     &      3     &   4     \\
Innestående i kr.:    &   0-999  & 1000-4999 &  5000-9999 & 10000-  \\
Antall konti ($N_i$):       &     400  &    300    &      200   &  100    \\
Standardavvik(${\sigma}_i$):&     250  &   1000    &     1500   & 3000
\end{tabular}
\end{center}
\begin{itemize}
\item[(a)]  Det er foreslått å velge ut $n_1=10, n_2=n_3=n_4=30$
 fra de fire strata. Hvorfor?

Anta at observerte gjennomsnitt og tilhørende empiriske standardavvik er
\begin{center}
\begin{tabular}{lrrrr}
Gj.snitt ($\bar{Y}_i$)    &    800   &   3500   &   8000   &   20000 \\
St.avvik ($S_i$)          &    200   &    900   &   1600   &    3500
\end{tabular}
\end{center}
\item[(b)]  Estimer $\bar{v}$ og rapporter resultatet.
\item[(b)]  Bokført verdi er 4.9 millioner kroner.  Er dette urimelig?
\end{itemize}

\item  Anta at revisoren i forrige oppgave isteden valgte $n$ = 100 konti
tilfeldig fra alle $N$ = 1000 konti og at han observerte gjennomsnittlig
innestående lik 5300 kr.  Gjennomsnittlig innestående på de
samme utvalgte konti året før var 4800 kr., mens det totale 
gjennomsnitt året før var 4600 kr.  Gi et alternativt estimat for
$\bar{v}$ på grunnlag av disse opplysningene.

\item  Gitt følgende sammenhørende verdier av $(v,w)$ for en 
populasjon på $N$ = 10 elementer.
\begin{center}
\begin{tabular}{ccccccccccc}
 $v_i$ :   &  2  &  3  &  4  &  5  &  2  &  3  &  2  &  6  &  1  &  4 \\
 $w_i$ :   &  1  &  2  &  4  &  3  &  2  &  4  &  1  &  3  &  1  &  2
\end{tabular}
\end{center}
Det trekkes et tilfeldig utvalg på $n$ = 3 elementer.

\begin{itemize}
\item[(a)]  Beregn variansen til rateestimatoren.
\item[(b)]  Beregn også variansen til gjennomsnittsestimatoren.
\item[(c)]  Hvor stort utvalg må trekkes for at denne skal gi samme 
pålitelighet som rateestimatoren for $n$ = 3.
\end{itemize}

\item  
\begin{itemize}
\item[(a)]  Foreslå en estimator for variansen til rateestimatoren.
\item[(b)] $\star$Forsøk å finne ut om den foreslåtte estimator er 
forventningsrett.
\end{itemize}

\item  En tannlege har gjennomført pensling av tennene til $N$ = 1000 
elever med sikte på å redusere tannråte.  Elevene går til
kontroll en gang hvert halvår og tidligere hadde de gjennomsnittlig
2.2 hull pr. gang.  Et halvår senere er han spent på om penslingen
har gitt resultater, og han noterer seg antall hull hos de $n$ =10 første
barna som kommer til kontroll.  Resultatet var
\begin{center}
\begin{tabular}{lcccccccccc}
 $Y_i$ :   &  2  &  0  &  3  &  1  &  1  &  4  &  0  &  4  &  2  &  3
\end{tabular}
\end{center}
\begin{itemize}
\item[(a)]  Estimer gjennomsnittlig antall hull etter et halvår med
penslede tenner og rapporter resultatet.
\item[(b)]  Er det grunn til å tro at penslingen har gitt resultater?
Tannlegen har opplysninger om antall hull ved siste kontroll for elevene
ovenfor.  Disse var (i samme rekkefølge)

\begin{center}

\begin{tabular}{ccccccccccc}
 $w_i$ :   &  3  &  0  &  4  &  2  &  1  &  5  &  1  &  4  &  3  &  3
\end{tabular}
\end{center}
\item[(c)]  Bruk dette til å gi et alternativt estimat.
\item[(d)]  Kommenter resultatene og også opplegget til undersøkelsen.
\end{itemize}

\item  En skogeier (f.eks. staten) ønsker å telle antall trær i
et større område.  For dette formål er området delt i et
rutenett bestående av $N$ = 100 ruter.  Av disse velges ut $n$ = 10 ruter
og trærne der telles.  Nå foreligger det luftfoto over vedkommende
område og det er mulig ut fra dette å anslå grovt antall trær
i hver rute.

Drøft hvordan denne tilleggsinformasjon kan utnyttes ved estimering av de
totale antall trær i området.

\item  En skogeier ønsker å anslå det totale volum av skogen i et
større område.  For dette formål er området delt i et rutenett
bestående av $N$ ruter.  Av disse velges ut tilfeldig $n$ ruter, fra 
rute nr. $i$ velges tilfeldig $m_i$ trær og gjennomsnittlig volum
$\bar{Y}_i$ av disse noteres (i $m^3$).  Samtidig telles det totale antall
trær $M_i$ i denne ruten.

\begin{itemize}
\item[(a)]  Foreslå en estimator for det totale volum av skogen uttrykt
ved $N$, $m_i$, $M_i$, $\bar{Y}_i$,  $i = 1, 2, \ldots, n.$
\item[(b)]  Estimer skogens totale volum dersom $N$ = 100, $n$ = 8 $m_1 =
m_2 = \cdots = m_8 = 5$ og det ble observert

\begin{center}
\begin{tabular}{lrrrrrrrr}
Rute nr. 1 :       &   1 &   2 &   3 &   4 &   5 &   6 &   7 &   8 \\
Antall trær :  &  30 &  50 &  55 &  35 &  30 &  40 &  65 &  10 \\
Gj.snitt :         & 0.8 & 0.6 & 1.1 & 1.0 & 0.4 & 0.5 & 1.0 & 0.5
\end{tabular}
\end{center}
\item[(c)]  Drøft muligheten for å trekke inn flere opplysninger som
kan gi et mer pålitelig estimat.
\end{itemize}

\item $\star$Anta en situasjon med $k$ strata med $N_i$ elementer i stratum 
nr.$i$, der variansen er ${\sigma}_i^2$,  $i = 1, 2, \ldots, k$.

\begin{itemize}
\item[(a)]  Vis at optimal utvelging svarer til $n_i$ proporsjonal
med $N_i\cdot {\sigma}_i$.
\end{itemize}
Anta at kostnadene ved hver observasjon fra stratum nr. $i$ er $c_i$, $i=1,
2, \ldots, k$.  Det er foreslått å velge $n_i$ proporsjonal med
$N_i\cdot {\sigma}_i/\sqrt{c_i}$.
\begin{itemize}
\item[(b)]  Vis at dette minimerer variansen for gitte kostnader, og også
mini\-merer kostnadene for gitt varians.
Hint:  Ulikheten i Oppgave~\ref*{kap:stokastiske}.40 kan brukes.
\end{itemize}

\item  Ved en undersøkelse er viltbestanden i et område anslått til
2 000 - 3 000 dyr.  Drøft hvordan denne informasjon kan brukes til å
planlegge en undersøkelse med merking og gjenfanging for neste år,
der en ønsker å anslå bestanden med en sikkerhetsmargin på
pluss-minus 200 dyr.

\item  Anta at en bestand av villrein skal anslås.
\begin{itemize}
\item[(a)]  Drøft metoden med merking og gjenfanging i denne sammenheng.
\item[(b)]  Hvilke konsekvenser har det dersom individer dør og fødes
i perioden fra merking til gjenfanging.
\item[(c)]  Foreslå eventuelt andre metoder til å estimere bestanden.
\end{itemize}
\end{enumerate}
\normalsize

 
