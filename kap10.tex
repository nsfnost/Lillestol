\chapter{Kontrollerte eksperimenter}
\label{kap:kontrollerte} % Opprinnelig kapittelnr: 10

\section{Innledning}
De mest gunstige omstendigheter for innsamling av et tallmateriale vil
være et {\em kontrollert eksperiment}.  Med dette menes at data
samles inn ved et eksperiment med veldefinerte eksperimentbetingelser,
som i de fleste tilfeller er valgt av eskperimentator selv.

Et kontrollert eksperiment vil som regel gi langt mer pålitelige
konklusjoner enn et mer eller mindre tilfeldig innsamlet tallmateriale.
Ved en god planlegging kan vi nemlig eliminere en rekke feilkilder.
Et kontrollert eksperiment gir også grunnlag for 
sannsynlighetsbetraktninger og for vurdering av usikkerheten knyttet
til konklusjonene fra analysen av tallmaterialet.  Som regel vil man 
også kunne forankre dataanalysen i modeller som allment kan aksepteres.
En viktig egenskap ved mange kontrollerte eks\-peri\-men\-ter er at de kan 
gjentas, også av andre, under tilnærmet de samme 
eksperimentbetingelser, slik at eventuelle konklusjoner kan etterprøves.

Muligheten for å utføre kontrollerte eksperimenter vil variere 
med det fagområde det er tale om.  De beste mulighetene har vi i 
naturvitenskapene, i første rekke fysikk, biologi og kjemi, samt i
teknikk og medisin.  I et labo\-ra\-to\-rium er det ofte mulig å finne et
eksperimentopplegg (design) og en analysemetode som er skreddersydd for
det problem vi ønsker å belyse.  I individorienterte fag som
pedagogikk og psykologi er også mulighetene gode.  Det samme gjelder
visse områder relatert til økonomisk virksomhet, f.eks. i
produktutvikling, produksjonsplanlegging og i studier av 
konsu\-ment\-ad\-ferd.  I samfunnsvitenskaper som sosiologi og 
samfunnsøkonomi er mulighetene mindre, men her finnes eksempler på
observasjonssituasjoner, gitt navnet {\em kvasieksperimenter}, der en i noen
grad kan rettferdiggjøre bruk av de samme modeller og analysemetoder
som brukes under mere ideelle omstendigheter.  

På en rekke fagfelter har en ingen mulighet  for å influere på
datamaterialet, og er henvist til rene observasjonsstudier.  I slike 
situasjoner vil en rekke statistiske analysemetoder vanskelig kunne
rettferdiggjøres.  Spørsmålet er om det finnes andre 
analysemetoder, eller om det rett og slett er urimelig eller umulig å
trekke generelle konklusjoner utfra observasjonene, slik at en analyse
må begrenses til det rent beskrivende.  Spesielt vanskelig er
studier der observasjonene tas på ulike tidspunkter, f.eks. for å
avdekke effekt av sosiale og politiske tiltak.  Det kan da være 
vanskelig å eliminere muligheten for at eventuelle endringer skyldes
andre forhold.  En annen vanskelighet oppstår når de foreliggende
data er fra en snevrere populasjon enn den vi ønsker å
generalisere til.

Eksempelvis, en av grunnene til at det gikk såpass lang tid før
røking som årsak til lungekreft ble allment akseptert i 
vitenskapelige kretser, var at man lenge bare hadde historiske data å
holde seg til, og en retrospektiv analyse rommet altfor mange muligheter
for feiltolkninger.  

Vi vil illustrere en del poenger vedrørende kontrollerte eksperimenter
ved et eksempel:\\


\begin{eksempel}{Vaksine}
Betrakt en situasjon der barna i et distrikt ofte lider av en bestemt
sjelden sykdom.  En ny og utprøvd vaksine blir introdusert og over
en femårsperiode observeres en betydelig nedgang i det antall barn
som angripes av sykdommen.  Det er da fristende å trekke den 
konklusjon at nedgangen skyldes vaksinen.  Dette kan være noe 
forhastet, idet femårsperioden kan tenkes falle sammen med en generell
bedring i kosthold, hygiene og helsetjeneste. 

 Det hadde vært 
et noe sikrere grunnlag å trekke konklusjoner på, dersom man 
hadde gitt noen barn vaksine og andre ikke, den siste gruppen utgjør 
da en såkalt kontrollgruppe.  Utover dette behandles de to grupper 
likt, eksempelvis tilbys de samme helsetjeneste.  Det beste ville
være om utvelgelsen skjer tilfeldig.  Da sikrer man at de som 
vaksineres ikke utgjør en spesiell gruppe m.h.t. andre kjennetegn,
noe som kan vanskeliggjøre tolkningen etterpå, f.eks. ved at de
som ble vaksinert i utgangspunktet hadde gjennomgående god helse. 
En eventuell betydelig forskjell mellom de to gruppene etter en viss 
periode vil da med rimelighet kunne tilskrives vaksinen og ikke andre
utenforliggende forhold.

I dette eksemplet er det altså metodologiske grunner for å
utføre vaksinasjonsprogrammet i form av et kontrollert eksperiment.
Det kan imidlertid være konkurrerende hensyn å ta, f.eks. rent 
økonomiske eller det moralske ved å ikke gi vaksinen straks til
alle som kunne trenge den.  Vi må da imidlertid ha for øye at
alternative metoder kan medføre at man selv etter lengre tids bruk,
ikke er i stand til å gi en dokumentasjon av vaksinens virkninger.
\end{eksempel}
Det kreves god innsikt i vedkommende fagområde kombinert med
betydelig statistisk kløkt, dersom slike data skal kunne analyseres
på en måte som står for kritikk, men statistisk teori kan
også være til hjelp her.

Vi vil nedenfor se på et par sentrale problemstillinger i 
forbindelse med kontrollerte eksperimenter.


\section{Sammenlignende eksperimenter og randomisering}
I praksis vil en ofte være interessert i å klarlegge virkningen
av en eller flere alternative behandlinger som kan anvendes på en
viss type objekter, f.eks. ulike medikamenter anvendt på pasienter,
ulike læremetoder anvendt på elever, ulike gjødninger anvendt
på planter, ulike dietter ved foring av griser eller ulike 
bearbeidelsesmåter av et industriprodukt.  Vi vil her se på
situasjoner der en ønsker å sammenligne to behandlinger, den ene
kan tenkes å være en ny behandling $B$ som man ennå ikke
fullt ut kjenner virkningen av, den andre kan være den tradisjonelle
behandling $A$ (som muligens består i ikke å behandle i det hele tatt).

  Vi vil tenke oss at et visst antall objekter velges ut og 
gis behandling $B$, og deres responser blir så observert.  Selv om
vi mener å kjenne virkningen av den tradisjonelle behandlingen
svært godt, vil det være hensiktsmessig å inkludere objekter
som gis behandling $A$ i eksperimentet, fordi vi da sikrer at sammenligningen
av de to behandlingene skjer under de samme betingelser.  Vi har dermed
en {\em behandlingsgruppe} ($B$) og en {\em kontrollgruppe} ($A$), og et
eksperiment av denne typen kalles ofte et {\em sammenlignende eksperiment}.
Vi har tidligere antydet at de objekter som skal gis en bestemt behandling
bør velges ut ved loddtrekning, dette kalles gjerne {\em randomisering}.
Flere ulike former for randomisering er tenkelig, vi vil nedenfor studere
to hovedtyper, {\em komplett randomisering og randomisering innen blokker}.

Ved komplett randomisering foreligger et gitt antall objekter som
del\-ta\-kere i eksperimentet, fra disse trekkes et tilfeldig utvalg som gis
behandling $B$, mens de resterende gis behandling $A$.  Ved 
randomisering innen blokker tenker vi oss isteden at objektene er inndelt
i flere grupper, kalt blokker, og at det trekkes et tilfeldig utvalg fra
hver blokk som gis behandling $B$, mens de resterende i hver blokk gis
behandling $A$.  Vi ser at vi med den siste framgangsmåten kan sikre at
visse typer objekter er med både blant de som gis behandling $A$ og $B$.

Den observerte respons i et kontrollert eksperiment kan være av to
typer, enten gruppert respons (f.eks. i kategorier av typen 
suksess-fiasko, ja-nei, høy-lav) eller gradert respons (f.eks.
målinger i meter, kilo, sekunder etc.).

Vi vil illustrere noen typiske problemstillinger for de situasjonene
som er skissert ovenfor med eksempler.  La oss starte med å 
studere gruppert respons:


\section{Gruppert respons}
\begin{eksempel}{Verktøy-randomisert}
Ved produksjon av en bestemt artikkel kan en vanskelig arbeidsoperasjon
enten utføres for hånd (metode 1) eller med et spesialverktøy
(metode 2).  Blant $N$=21 arbeidere velges ut tilfeldig $n$=11 arbeidere
som skal utføre arbeidet etter metode 2, mens de 10 andre skal
bruke metode 1.  Vi er bl.a. interessert i å teste om det er grunn
til å påstå at metode 2 gir bedre resultat enn metode 1.  
For hver deltaker i eksperimentet observeres det om vedkommende lykkes
med arbeidsoperasjonen i første forsøk eller ikke.  Resultatene
var
\begin{center}
\begin{tabular}{l|cc|c} 
           &    Lykkes   &   Lykkes ikke    &    Sum \\ \hline
Metode 1   &      6      &        4         &     10  \\
Metode 2   &      8      &        3         &     11 \\ \hline
Sum        &     14      &        7         &     21 \\ \hline
\end{tabular}
\end{center}
Vi ser at metode 2 kom relativt gunstig ut, idet 40\% av deltakerne
ikke lyktes med metode 1, mens det tilsvarende tall for metode 2 ble
ca. 27\%.  De observerte forskjeller i arbeidsresultatet kan enten
forklares ved at metode 2 virkelig er bedre enn 1, eller ved at de er
likeverdige og at det observerte resultat skyldes tilfeldigheter
(vi antar at metode 2 ikke under noen om\-sten\-dighet er dårligere
enn 1).  La oss som nullhypotese $H_0$ bruke at metodene er
likeverdige.  Som testobservator bruker vi antallet $Y$ blant de som
bruker metode 2 som ikke lykkes.  Det er små verdier av $Y$ som
gir grunnlag for å forkaste nullhypotesen til fordel for metode 2.
Dersom nullhypotesen er riktig er $Y$ hypergeometrisk fordelt
($N$ = 21, $M$ = 7, $n$ = 11).  Dette forklarer vi slik:  Dersom de to 
metodene er likeverdige, vil det være $M$ = 7 som ikke lykkes i
første forsøk uansett hvilken metode de brukte (om de lykkes
eller ikke skyldes altså ikke metoden, men andre forhold).  Den
observerte verdi av $Y$ vil derfor være bestemt ved trekningen,
nemlig ved hvilke arbeidere blant de $M$ = 7 spesielle som 
tilfeldigvis ble trukket ut til å arbeide etter metode 2.

Beregner vi $P$-verdien til det observerte resultat, får vi 

\[ P_{H_0}(Y\leq 3)=\sum_{y=0}^3 \frac{\bino{7}{y} \bino{14}{11-y}}
                                 {\bino{21}{11}}=0.438 \]
Denne er såpass høy at det ikke er noen rimelig grunn til å
påstå at metode 2 virkelig gir bedre resultat enn metode 1.
\end{eksempel}

Vi ser at i dette eksemplet hadde randomiseringen en rolle utover
det å sikre rettferdighet for de to metodene.  Den satte oss også
i stand til å finne den eksakte sannsynlighetsfordelingen til 
testobservatoren $Y$ når nullhypotesen er riktig. 
Analysemetoden som er benyttet i dette eksemplet kalles
{\em Fisher-Irwins test}, og er aktuell i en rekke situasjoner.  La oss
imid\-ler\-tid se på den tilsvarende problemstilling der det ikke er
foretatt noen randomisering.\\

\begin{eksempel}{Verktøy-ikke randomisert}
En bedrift produserer en artikkel på to produksjonssteder 1 og 2.
På det ene stedet (2) er det anskaffet spesialverktøy, og før
man bestemmer seg for å anskaffe dette på det andre 
produksjonsstedet, vil man gjennomføre et eksperiment som omfatter
alle arbeiderne på begge produksjonsstedene, henholdsvis 10 ved
sted 1 og 11 ved sted 2, som alle får arbeide under samme
produksjonsforhold.  Anta, for korthets skyld, at data er de samme som
i Eksempel 2.  Denne situasjonen faller imidlertid utenfor rammen av den 
modell vi brukte i forrige eksempel, fordi det her ikke foreligger
noen randomisering. 

 Det foreliggende problem kan imidlertid studeres
slik:  Vi kan se på produksjonsresultatet for de to 
produksjonsstedene som to binomiske forsøks\-rekker med henholdsvis
$n_1$=10 og $n_2$=11 forsøk i hver og med suksess-sannsynlighet
henholdsvis $p_1$ og $p_2$.  Disse tolkes som sannsynligheten for at
en tilfeldig arbeider på vedkommende arbeidsplass lykkes (se
for øvrig Oppgave~\ref*{kap:sannsynlighetsfordelinger}.16).  Nullhypotesen om at de to produksjonsstedene
holder samme kvalitet kan formuleres som $p_1 = p_2$, mens alternativet
at sted 2 gir bedre kvalitet kan formuleres som $p_2>p_1$.  La $X_1$
og $X_2$ være antall som lykkes ved de to produksjonsstedene.
Antar vi at produksjonsresultatene for de ulike arbeiderne er uavhengige,
vil $X_1$ og $X_2$ være uavhengige stokastiske variable som er
binomisk fordelte henholdsvis ($n_1,p_1$) og ($n_2,p_2$).  

Det kan gis argumenter for at testing av hypotesen om ingen forskjell
kan utføres som i forrige eksempel, dvs. betrakte alle marginalene
som gitte tall.  Som testobservator kan brukes en hvilken som helst av
de fire størrelsene inn i tabellen. Merk at når alle 
marginalene er gitte vil en av disse bestemme de tre andre 
(jfr. Oppgave~\ref*{kap:sannsynlighetsfordelinger}.35).  Testingen foregår da som om den valgte 
testobservator, under nullhypotesen, var hypergeometrisk fordelt 
($N,M,n$) der $M$ og $n$ er de respektive marginaler, eksempelvis er
$X_1$ hypergeometrisk fordelt ($N$=21, $M$=14, $n$=10).
\end{eksempel}

Selv om de to eksemplene ovenfor er beslektet, og vi har foreslått
samme analysemetode for begge, er det en prinsipiell forskjell m.h.t. 
de konklusjoner det er mulig å trekke.  Anta at datamaterialet 
var slik at det var grunnlag for å forkaste hypotesen om samme
kvalitet til fordel for produksjon med spesialverktøy.  I det
første eksemplet vil det, p.g.a. randomiseringen, være grunnlag
for å påstå at spesialverktøy generelt gir bedre
produksjonsresultat, mens det i det andre eksemplet ikke er grunnlag 
for å trekke en så vidtrekkende konklusjon, med mindre vi på
forhånd visste at de to arbeidsplassene var like m.h.t. kvaliteten
av produksjonen uten spesialverktøy.  

En alternativ testmetode basert på normaltilnærming er U-testen som
vi studerte i Kapittel 7.6, som er ekvivalent med en kjikvadrattest som 
beskrevet i Kapittel 9.3.

La oss så se på et eksempel av typen randomiserte blokker.  
For enkelhets skyld betrakter vi en situasjon med bare to objekter
innen hver blokk, slike situasjoner forekommer ofte i praksis og kalles
{\em parvise sammenligninger}.\\


\begin{eksempel}{Konservering}
En bedrift som produserer syltetøy vurderer en ny metode for 
konservering som det påstås gir bedre smak enn den som brukes 
nå.  Bedriften ønsker å finne ut om det er grunnlag for dette, 
og det utføres derfor et kontrollert eksperiment.  Det er mulig at
råstoffet kan variere fra koking til koking, og det synes derfor
rimelig å dele hver koking i to deler hvorav den ene konserveres
etter den nye metode, den andre etter den gamle.  Utvelgelsen bør
skje ved loddtrekning, det sikrer en rettferdig behandling dersom
kvaliteten innen hver koking skulle variere.  Dette gjentas for i alt 
ti kokinger.  I denne situasjon har vi altså ti blokker, hver 
blokk består av to kvanta syltetøy.

  La vår nullhypotese
være at de to metodene er likeverdige, mens alternativet er at
den nye metoden gir bedre produkt.  Når syltetøyet er ferdig for
salg, avgjør bedriftens smaksekspert på grunnlag av en rekke
smaksprøver hvilken av de to typer syltetøy innen hver koking som
hun mener smaker best, anta for enkelhets skyld at uavgjort ikke 
forekommer.  La $X$ være antall ganger det nye syltetøy var
best, store verdier av $X$ gir grunnlag for å forkaste nullhypotesen.
Når nullhypotesen er riktig vil $X$ være binomisk fordelt
($n$=10, $p$=0.5), fordi da er det tilfeldig hvilket av de to produkter
som smakseksperten velger ut som best.  Anta at den observerte verdi
av $X$ er 7.  $P$-verdien til det observerte resultat er da 

\[ P=P_{H_0}(X\geq 7)=0.1719,          \]
som neppe gir grunn til å påstå at det nye produkt er 
bedre enn det gamle.  Den analysemetode vi har brukt her går ofte 
under navnet {\em tegntesten}.
\end{eksempel}

For å undersøke om det nye konserveringsmidlet gir bedre smak
er det foreslått et alternativt eksperiment, nemlig at de to typer
syltetøy blir prøvd ut på et panel bestående av et visst
antall konsumenter.  Kan hende har bedriften erfaring for at kvaliteten
av produktet er svært jevn både innen hver koking og fra 
koking til koking, slik at dette utelukkes som eventuelle feilkilder
ved undersøkelsen.  Imidlertid tenker man seg at dersom det er forskjell
i smak for de to produktene så kan det slå ut ulikt fra person
til person, idet smak og behag er forskjellig.  I denne situasjonen har
vi en blokk bestående av to prøver for hver person, og vi vil,
for hver person, trekke lodd hvilken av de to prøver som vedkommende
får smake først.  Vi unngår da eventuelle feil som skyldes
at personer, i en situasjon hvor de ikke smaker forskjell, systematisk
sier at det første (evt. siste) produktet er best.  Analysen av
resultatene kan foregå på samme måte som ovenfor.  I en
situasjon der vi ikke kan utelukke feilkilder av den typen vi skisserte
først, vil vi kan hende utføre et større eksperiment der hver 
deltaker i pa\-ne\-let får smake flere smaksprøver fra ulike kokinger.
Analysen av et slikt eksperiment blir noe mer komplisert.


\section{Gradert respons-Rangtester}
La oss så se på noen situasjoner hvor den observerte respons
er gradert.\\

\begin{eksempel}{Verktøy-randomisert}
La situasjonen være som beskrevet i Eksempel 2, men anta isteden
at det er forbrukt tid for arbeidsoperasjonen som er av interesse.
Det er påstått at spesialverktøy (metode 2) gjennomgående
forkorter arbeidstiden.  Blant de $N$=21 arbeidere velges ut tilfeldig
$n$=11 som utfører arbeidet etter metode 2, de resterende etter metode 1.
Resultatet ble (i sekunder)
\begin{center}
\begin{tabular}{lccccccccccc}
Metode 1: &  45 & 48 & 61 & 52 & 48 & 63 & 52 & 54 & 50 & 58 &   \\
Metode 2: &  47 & 59 & 43 & 50 & 45 & 45 & 49 & 41 & 47 & 95 & 50
\end{tabular}
\end{center}
For å karakterisere tendensen i dette tallmaterialet kunne vi 
bruke gjennomsnittstidene for de to metodene.  Imidlertid ser vi at
den nest siste observasjon for metode 2 er svært avvikende
\footnote{Ved nærmere undersøkelse viste det seg at observasjonen
skyldtes et arbeidsuhell som ikke hadde noe med bruk av spesialverktøy
å gjøre.} og vil trekke gjennomsnittet opp på et nivå
som ikke er typisk for observasjonene som helhet.  Det synes mer 
rimelig å bruke medianen til å karakterisere materialet.  Den
er for metode 1 lik 52 og for metode 2 lik 47, slik at spesialverktøy
ser ut til å kunne redusere arbeidstiden.  Spørsmålet er
om forskjellen er stor nok til at vi avviser at den skyldes 
tilfeldigheter.  Vi stiller da opp nullhypotesen at de to metodene
er like m.h.t. tidsforbruk og alternativet at metode 2 er raskere.  En
metode å teste denne hypotesen på er å ordne tallmaterialet
i stigende rekkefølge og understreke observasjoner svarende til
metode 2, slik
\begin{center}
\begin{tabular}{ccccccccccc}
 \underline{41} & \underline{43} & \underline{45} & 45 & \underline{45}
  & \underline{47} & \underline{47} & 48 & 48 & \underline{49} 
  &    \underline{50} \\
 50 & \underline{50}& 52 & 52 & 54 & 58 &  \underline{59} &
 61 & 63 & \underline{95} & 
\end{tabular}
\end{center}
Vi tildeler så de 21 observasjonene rang fra 1 til 21 i henhold
til rekkefølgen.  Sum rang for metode 2-observasjonene blir

\[    W = 1 + 2 + 3 + 5 + 6 + 7 + 10 + 11 + 13 + 18 + 21 = 97 \]
Merk at det kan oppstå et problem ved at to eller flere
observasjoner har samme verdi, og dette må løses på en,
for de to metodene, rettferdig måte.

Det er grunn til å forkaste nullhypotesen når $W$ er
tilstrekkelig liten, fordi dette svarer til at metode 2-observasjonene
er gjennomgående små.  For å avgjøre om det observerte
resultat gir grunn til forkasting må vi beregne $P$-verdien

\[ P=P_{H_0}(W\leq 97).         \]
Slike sannsynligheter kan, som følge av randomiseringen, beregnes
eksakt, og det er utarbeidet tabeller som gjør det mulig å
avlese disse direkte.  Vi vil her ikke gå nærmere inn på
teorien for eksakt beregning, ei heller på bruken av slike tabeller,
og må derfor henvise leseren til annen litteratur.  Den skisserte
metode går i litteraturen under navnet {\em Wilcoxons test for 
to utvalg}\index{Wilcoxons test!to utvalg}.  Imidlertid kan det vises at dersom $N$ ikke er altfor liten,
f.eks. minst 15, og utvalget ikke er altfor skjevt m.h.t. antallet i
hver gruppe, så vil sannsynlighetsfordelingen til $W$ kunne 
tilnærmes til normalkurven.  Vi vil da trenge at såframt 
nullhypotesen er riktig (se Oppgave~13)

\[ EW=\frac{n(N+1)}{2} \mbox{\ \ \ } varW=\frac{n(N-n)(N+1)}{12}    \]
Brukes dette i den foreliggende situasjon, får vi $EW$=121 og 
$varW$=201.67, slik at 

\[  P=P_{H_0}(W\leq 97)\approx G(\frac{97-121}{14.2})=G(-1.69)=0.0455  \]
Materialet gir altså rimelig grunn til å påstå at 
metode 2 er raskere (signifikant på 5\% nivå).  I praksis vil
man trenge de eksakte tabellene som er nevnt ovenfor bare for 
spesielt små $N$. \\
\end{eksempel}
Over til en situasjon med parrede sammenligninger:\\

\begin{eksempel}{Verktøy-randomiserte blokker}
La situasjonen være som i forrige eksempel, men anta isteden at 
det deltar $n$=10 arbeidere som utfører arbeidsoperasjonen to
ganger, en med hver metode.  Her utgjør hver arbeider en blokk 
bestående av to forsøk, og for hver arbeider trekkes det lodd
om hvilken metode som skal brukes først.  Det kan ikke utelukkes
at det er betydelige individuelle variasjoner i arbeidstempo fra arbeider
til arbeider.  Med dette eksperimentopplegget blir imidlertid de to
metodene satt opp mot hverandre for hvert individ, og dette burde gi et
sikrere grunnlag for konklusjoner vedrørende forskjeller mellom
metodene.  Det er derfor trolig at opplegget her er å foretrekke 
framfor opplegget i forrige eksempel.  Imidlertid kan det tenkes forhold
der opplegget ikke kan brukes, f.eks. dersom en arbeider som er opplært
til å bruke spesialverktøy ikke lenger er i stand til å 
utføre arbeidsoperasjonen uten spesialverktøy like godt som før.
Anta at dette ikke er tilfellet her.  Resultatet av eksperimentet ble
(i sekunder)
\begin{center}
\begin{tabular}{lrrrrrrrrrr}
Arbeider nr.: &   1  &  2  &  3  &  4  &  5  &  6  &  7  &  8  &  9  &  10 \\
Metode 1:    &  48  & 53  & 52  & 57  & 43  & 83  & 59  & 71  & 40  &  61 \\
Metode 2:    &  45  & 42  & 58  & 50  & 41  & 47  & 53  & 66  & 45  &  53
\end{tabular}
\end{center}
For å karakterisere tendensen i dette tallmaterialet kan vi først
beregne forskjellen i forbrukt tid for hver arbeider.  Denne er
\begin{center}
\begin{tabular}{lrrrrrrrrrr}
Forskjell: &   3  &  11  &  $-6$  &  7  &  2  &  36  &  6  &  5  &  $-5$  &  8
\end{tabular}
\end{center}
Den gjennomsnittlige forskjell er 6.7 sekunder, mens medianforskjellen
er 5.5.  Spørsmålet er om denne forskjell er stor nok til å
se bort fra at den skyldes tilfeldigheter.  La nullhypotese og 
alternativ være som i forrige eksempel.  En måte å teste
hypotesen på, er å ordne forskjellene etter stigende tallverdi
slik (merk den rettferdige behandling av de sammenfallende tall\-ver\-dier)
\begin{center}
\begin{tabular}{cccccccccc}
           2  &  3  &  5  &  $-5$  &  $-6$  &  6  &  7  &  8  &  11  &  36
\end{tabular}
\end{center}
Vi tildeler disse tall rang fra 1 til 10 i henhold til rekkefølgen.
Sum rang for de negative differanser blir

\[         V = 4 + 5 = 9.                \]
Det er grunn til å forkaste nullhypotesen når $V$ er 
tilstrekkelig liten fordi dette betyr få og små negative
differenser, dvs. metode 2-observasjonene er gjennomgående mindre
enn de tilhørende metode 1-observasjonene.  For å avgjøre
om det observerte resultat gir grunn til forkastning må vi beregne
$P$-verdien

\[  P=P_{H_0}(V\leq 9).         \]
Slike sannsynligheter kan, som følge av randomiseringen, beregnes
eksakt, og det er utarbeidet tabeller hvor vi kan avlese disse direkte,
vi viser igjen til annen litteratur.  Den skisserte metode går i
litteraturen under navnet {\em Wilcoxons test for parrede observasjoner}
\index{Wilcoxons test!parrede observasjoner}.
Det kan vises at dersom $n$ ikke er altfor liten, så vil 
sannsynlighetsfordelingen til $V$ kunne tilnærmes med normalkurven.
Vi vil da trenge at såframt nullhypotesen er riktig blir (se Oppgave~14)

\[ EV=\frac{n(n+1)}{4} \mbox{\ \ \ } varV=\frac{n(n+1)(2n+1)}{24}     \]
Brukes dette i den foreliggende situasjon får vi $EV$= 27.5 og
$varV$= 96.25 slik at

\[  P=P_{H_0}(V\leq 9)\approx G(\frac{9-27.5}{9.81})=G(-1.89)=0.029 \]
mens det eksakte tall ifølge tabell er 0.032. \footnote {Bruk av
heltallskorreksjon gir forøvrig enda bedre tilnærming.}
Materialet gir altså rimelig grunn til å påstå at metode
2 er raskere (signifikant på 5\% nivå).
\end{eksempel}

For problemstillingene i de to siste eksemplene har vi skissert enkle
ana\-ly\-se\-me\-toder basert på ranger.  Det finnes også andre metoder.
Mest brukt er kanskje metoder basert på gjennomsnittsbetraktninger
på observasjonene direkte.  Disse forutsetter som regel en såkalt
variansanalysemodell,  og dette er temaet for Kapittel 11.  For eksakt
beregning av $P$-verdier kreves da i tillegg forutsetning om 
normalitet av de opprinnelige observasjonene, noe som ikke ser ut til 
å være tilfellet for materialene i Eksempel 5 og 6.  En vil
kanskje kunne tro at en kaster bort mye informasjon ved å erstatte
de opprinnelige tall men ranger.  Dette er ikke tilfelle, og i tillegg
har slike metoder den gunstige egenskap at eventuelle ekstreme 
observasjoner, som muligens skyldes uhell, ikke får avgjørende
betydning for konklusjonen.  Metoden er hva statistikerne kaller 
{\em robust}.

En kommentar om alternative eksperimentopplegg:  Ved randomiserte 
blokker forsøker en å velge blokker med objekter som er mest
mulig homogene, men gjerne med betydelig variasjon mellom blokkene.
Da får en prøvd de ulike behandlingsmetoder direkte mot hverandre
under mest mulig like forhold.  En vil derfor som regel foretrekke
dette eksperimentopplegg framfor komplett randomisering dersom begge
er mulig å realisere.  Eksempelvis vil en psykolog ved sammenligning
av to læremetoder gjerne arbeide med en rekke par av en-eggede 
tvillinger, men slike er ikke alltid lette å oppdrive, og han må
kan hende være foruten.

Vi har her bare tatt for oss eksperimenter med en faktor med to kate\-gorier:
 metode 1 og metode 2. Eksperimenter med flere faktorer og med flere
kategorier for hver faktor, er diskutert i Kapittel 11 og 15. \\

\noindent
{\bf Merknad:} De rangmetodene som er presentert i dette avsnittet forutsetter
ingen spesiell sannsynlighetsfordeling for den graderte respons, de er
såkalte fordelingsfrie metoder. Metodene er i praksis også brukt i
situasjoner der observasjonene selv er på en rangeringsskala, dvs. er
ordinale (se innledningen i Kapittel 9). Dette er ofte tilfelle i
markedsforskning, der konsumenter bedømmer iht. en vurderingsskala,
for eksempel en 7-delt skala fra Dårlig {(-3)} via Middels (0) til Godt (3).
Metoden er ikke like godt teoretisk forankret i slike situasjoner, og et lite
antall ordinale kategorier leder til mange sammenfallende ranger.
God programvare bør imidlertid holde orden på dette, slik at reelle
sannsynlighetsgarantier kan gis under visse forutsetninger.


\section{Oppgaver}
\small
\begin{enumerate}
\item 
En sykehuslege tror at en bestemt behandling kan forlenge livet til
personer som har hatt hjerteattakk.  Diskuter muligheten for et
forsøksopplegg der eventuelle forskjeller i levetid med rimelighet
kan tilskrives behandlingen og ikke andre utenforliggende forhold.

\item
Det påstås at man ved sprøyting med visse kjemikalier fra 
fly kan øke sjansen for at regnskyer avgir regn.  Diskuter de 
problemer som er knyttet til å kunne avgjøre dette.  Foreslå
et eksperimentopplegg som kan gi et grunnlag for pålitelige
konklusjoner.

\item
Man ønsker å utføre en undersøkelse om bruk av 
selvinstruerende (programmert) lærestoff kontra vanlig 
klasseromsundervisning.  En gruppe elever tilbys de to alternativer og 
får velge selv.  Etter endt undervisning tar begge grupper samme
prøve og resultatene sammenlignes.  Diskuter eventuelle betenkelige
sider ved et slikt forsøksopplegg, foreslå et alternativt
opplegg.  Diskuter også muligheten av opplegg som tar omsyn til
elevenes motivasjon (evt. uvillighet) til å delta etter et bestemt
opplegg, eller overhodet.

\item
La situasjonen være som i forrige oppgave.  Blant 24 elever velges
ut tilfeldig 12 elever som får selvinstruerende undervisning,
resten får vanlig undervisning.  Man ønsker å vite om det
er grunn til å påstå at selvinstruerende undervisning gir
dårligere resultat enn vanlig undervisning.  Resultatet av den
etterfølgende prøve ble
\begin{center}
\begin{tabular}{l|cc|c} 
                     &  Bestått  &  Ikke bestått  &   Sum \\ \hline
Ny undervisning      &      9        &       3            &    12 \\
Vanlig undervisning  &     11        &       1            &    12 \\ \hline
Sum                  &     20        &       4            &    24 \\ \hline
\end{tabular}
\end{center}
Formuler situasjonen som et hypotesetestingsproblem, beregn $P$-verdi
og avgi konklusjon dersom signifikansnivået er valgt lik 5\%.

\item
En bedrift vurderer to lagringsmetoder, den ene er noe mer kostbar enn
den andre.  Et produksjonsparti på 400 artikler deles i to og
hver halvpart lagres etter hver sin metode.  Blant de 200 som ble 
lagret etter den rimeligste metode måtte 15 vrakes etter en viss tid,
mens bare 10 måtte vrakes av de som var lagret etter den kostbare 
metoden.  Gir dette grunnlag for å påstå at den rimelige
metoden er dårligere enn den kostbare metoden?  Hvilken forutsetning 
bygger analysen på?

\item
Et gartneri ønsker å undersøke om en viss type frø har 
økt spireevne dersom det pakkes i en ny type emballasje (1) i 
forhold til vanlig emballasje (2).  Ta sannsynligheten for at et
tilfeldig frø spirer som et uttrykk for spireevnen.  Det tas
$n_1 = n_2$ = 50 frø av hver sort og disse sås under de samme 
forhold.  Av de frø som var lagret i den nye emballasje spirte 45
og av de som ble lagret i den gamle emballasjen spirte 40.  Estimer
forskjellen i spireevne og rapporter resultatet.  Test hypotesen om 
at de to spireevnene er like, beregn $P$-verdien til det observerte
resultat og gi konklusjonen dersom signifikansnivået er valgt lik
5\%.

\item
Man ønsker å undersøke om en ny type solbadolje $A$ er 
bedre enn et konkurrerende produkt $B$.  I alt 12 personer har sagt
seg villige til å delta i et eksperiment.  Hver rygg ``deles i to"
den ene del smøres inn med $A$, den andre del med $B$, hvilken del 
som får $A$ skjer ved loddtrekning.  Etter en lengre periode i
solen observeres grad av solbrenthet på en skala fra 0 til 3.
Anta at resultatet var
\begin{center}
\begin{tabular}{lrrrrrrrrrrrr}
Person nr.:  &   1 & 2 & 3 & 4 & 5 & 6 & 7 & 8 & 9 & 10 & 11 & 12 \\
Med A:       &   2 & 1 & 1 & 2 & 0 & 2 & 1 & 0 & 2 &  1 &  1 &  1 \\
Med B:       &   3 & 2 & 1 & 1 & 1 & 3 & 0 & 1 & 2 &  1 &  3 &  2 \\
Forskjell:   &  $-$&$-$& 0 & + &$-$&$-$& + &$-$& 0 &  0 & $-$& $-$
\end{tabular}
\end{center}
Ved analysen velger en å se bort fra de personer der forskjellen
er null.  La $X$ være antall positive differenser.

\begin{itemize}
\item[(a)] Forklar at dersom produktene i virkeligheten er likeverdige
vil $X$ være binomisk fordelt ($n$=9, $p$=0.5).
\item[(b)] Hvordan kan eventuelle forskjeller mellom $A$ og $B$ forklares?
\item[(c)] Formuler situasjonen som et hypotesetestingsproblem.  Hva
           blir konklusjonen dersom det valgte sinifikansnivå var 5\%?
\end{itemize}

\item
15 skiløpere i Norgeseliten er samlet i treningsleir.  Det er nettopp
mar\-keds\-ført en ny type skismøring ($A$) og langrennstreneren
ønsker å teste denne mot en type smøring ($B$) som han vet
er velegnet for dagens føre.  Treneren velger ut tilfeldig 7
løpere som smører med $A$, mens resten smører med $B$.  Deretter
starter alle løpere i et 15 km langrenn med fellesstart.  Resultatet
ble i minutter og sekunder
\begin{center}
\begin{tabular}{ccccccccc}
A:  &  48.13 & 47.59 & 48.09 & 47.43 & 48.31 & 48.01 & 49.03 &       \\
B:  &  48.06 & 47.45 & 49.27 & 48.36 & 48.54 & 49.14 & 48.20 & 48.26
\end{tabular}
\end{center}
Dersom den nye smøringen er signifikant bedre enn den gamle ønsker
man å gjøre dette kjent blant konkurranseløpere.  Beregn
tilnærmet $P$-verdi.  Hva blir konklusjonen dersom signifikansnivå
5\% er valgt?  Enn 10\%?

\item
Se igjen på problemstillingen i Oppgave~3 og~4.  Anta at hver
prøve blir gradert på en karakterskala fra 0 til 9, hvor 0 og
1 betyr ikke bestått.  Anta at resultatet ble

\begin{center}
\begin{tabular}{lcccccccccccc}
Ny undervisning:     &  1 & 8 & 5 & 6 & 5 & 3 & 1 & 0 & 3 & 2 & 6 & 4 \\
Vanlig undervisning: &  0 & 2 & 5 & 7 & 4 & 4 & 9 & 7 & 5 & 5 & 3 & 6
\end{tabular}
\end{center}
Vil du på dette grunnlag påstå at selvinstruerende 
undervisning gir dårligere resultat.

\item
Det blir påstått at ingrediensen Luriol, når den blir
blandet i bensinen, gir en høyere toppfart for biler.  For å
undersøke dette blir følgende undersøkelse gjennomført:
Ti bilfirmaer har hver stillet til rådighet to biler som begge
er av samme modell og standard.  Det trekkes lodd hvilken av de to
bilene av hver modell som skal få Luriol.  Det blir så
arrangert en fartsprøve.  Resultatet ble i km/t.
\begin{center} \addtolength{\tabcolsep}{-0.2\tabcolsep}
\begin{tabular}{lcccccccccc} 
Modell:     &  1  &  2  &  3  &  4  &  5  &  6  &  7  &  8  &  9  &  10 \\
Uten Luriol:& 136 &  94 & 127 & 125 & 133 & 124 & 127 & 138 & 103 & 113 \\
Med Luriol: & 140 & 108 & 128 & 128 & 126 & 137 & 139 & 147 & 112 & 108
\end{tabular}
\end{center}
Uten en test som tar sikte på å klargjøre om Luriol har
den påståtte effekt.  Vi ønsker signifikansnivå 5\%.

\item
Betrakt igjen problemstillingen fra Oppgave~3, men anta nå at de
24 elevene er gruppert to og to slik at de to innen hvert par står
noenlunde likt i faget før undervisningen tar til.  Fra hvert par
velges tilfeldig en som får selvinstruerende undervisning, den
andre får vanlig undervisning.  Anta at karakterene på
prøven etter undervisningsperioden ble
\begin{center}
\begin{tabular}{lcccccccccccc}
Par nr.:            & 1 & 2 & 3 & 4 & 5 & 6 & 7 & 8 & 9 & 10 & 11 & 12 \\
Ny undervisning:    & 3 & 1 & 4 & 5 & 7 & 3 & 6 & 4 & 5 &  7 &  6 &  4 \\
Vanlig undervisning:& 5 & 3 & 4 & 4 & 7 & 5 & 7 & 5 & 6 &  5 &  8 &  6
\end{tabular}
\end{center}
Vil du på dette grunnlag påstå at selvinstruerende 
undervisning gir dårligere resultat.  Bruk 5\% signifikansnivå.

\item
Her følger en utskrift fra analyse av dataene i Eksempel 5.

\begin{center} \framebox[10cm]{\begin{minipage}{9cm}\rule{0cm}{0.5cm}
\tt
 >> READ 'eks10.5' X Y \\
 >> WILCOXON2 X Y \\
 -- W = 134.0  (expected = 121)\\
 -- P = 0.0968 (twosided adjusted for ties)\\
\end{minipage}} \end{center}
Forklar den tilsynelatende forskjell fra resultatet i teksten.
Reproduser resultatet med din foretrukne programvare (testen kalles også
Mann-Whitney).\\



Ved løsning av de to neste oppgavene trengs formlene:
\begin{center}
 $ 1+2+3+\cdots +k=k(k+1)/2   $ \\
 $ 1^2+2^2+3^2+\cdots +k^2=k(k+1)(2k+1)/6   $ 
\end{center}
\item
 $\star$Betrakt en lotterisituasjon hvor elementene i populasjonen har 
verdiene $1, 2, 3, \ldots, N$, dvs. $v_1 = i$.
\begin{itemize}
\item[(a)] Vis at i dette tilfellet blir 

\[  \bar{v}=\frac{1}{2}(N+1) \mbox{\ \ og \ \ } 
                               {\sigma}^2=\frac{(N+1)(N-1)}{12} \]
\item[(b)] Bruk dette til å vise formlene for $EW$ og $varW$ som
trengs til Wil\-coxons test for to utvalg.
\end{itemize}

\item
 $\star$La $I_1, I_2, \ldots, I_n$ være uavhengige indikatorvariable
slik at

\[  P(I_j=1)=P(I_j=0)=\frac{1}{2}    \]
og la
\[ V=I_1+2I_2+\cdots +nI_n. \]
\begin{itemize}
\item[(a)] Vis at 
\[   EV=\frac{n(n+1)}{4} \mbox{\ \ og \ \ }
                     varV=\frac{n(n+1)(2n+1)}{24} \]
\item[(b)] Vis at testobservatoren $V$ som brukes i Wilcoxons test 
for parrede observasjoner har form som ovenfor, der $I_j$ indikerer
om observasjonen som har fått rang $j$ er negativ eller positiv.
\end{itemize}
\end{enumerate}
\normalsize



