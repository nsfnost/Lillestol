
\chapter*{Forord}
\addcontentsline{toc}{chapter}{Forord}

Denne boken er basert på Jostein Lillestøls lærebok 
``Sannsynlighetsregning og statistikk med anvendelser'' som blant annet ble
brukt i grunnkurs i statistikk ved Norges Handelshøyskole i perioden .... - ...
Kildekoden til boken, tilgjenglig fra \url{https://github.com/nsfnost/Lillestol}, er en del av NOST (Norwegian Open source Statistics Textbooks)
under en \href{https://creativecommons.org/licenses/by-nc-sa/4.0/}{CC~BY-NC-SA~4.0} lisens. Dette innebærer
at boken kan modifiseres (i henhold til lisensen), og tilpasses spesielle kursbehov.
Da det forventes at mange ulike personer i fremtiden vil bidra til læreboken
er forfatteren av boken NOST. Fordi kildekoden er under versjonskontrol vil
det alltid være mulig å spore hvilken deler som kommer fra Lillestøls lærebok.
Det oppfordres til at dette forordet ikke endres og alltid inkluderes i boken.

\section*{Forord til Lillestøl (1997, 5.~utg.)}
Dette er en elementær innføring i sannsynlighetsregning og
statistikk, med utgangspunkt i matematikken fra videregående skole.
Fremstillingen gjør ikke bruk av differensial og integralregning,
og burde være tilgjengelig også for lesere med bare 1MA.
Ved mange universiteter og høyskoler gis i dag innføringskurser i
sannsynlighetsregning og statistikk første studieår, og boken kan
være aktuell som lærebok der.

Boken prøver å oppdra leseren til modelltankegang, dvs. at
praktiske problemer studeres innen en teoriramme, der konklusjoner
bare har gyldighet i relasjon til de forutsetninger som gjøres.
Boken er ingen kokebok for de metoder som oftest anvendes i praksis,
ut fra oppfatningen at det ikke er en metode alene som overføres til nye
situasjoner, men argumentasjon for metoden.

Boken er delt i to deler.  Første del (kapitlene 1-8) gir en teoretisk
innføring i sannsynlighetsregning og statistikk.  Annen del (kapitlene
9-16) omfatter ulike emner for videre studium, der de ulike emner kan leses
uavhengig av hverandre.

Første del utgjør den kjerne av stoff som er ment å gi leseren
et grunnlag for å forstå sannsynlighetsteoretiske og statistiske
argumenter anvendt i praksis, og eventuelt kunne utføre enklere 
argumenter på egen hånd.  Dette stoff danner også utgangspunkt 
for kommunikasjon med fagstatistiker og for videre studier.

Så langt det er råd forsøker vi i første del å gi presise
argumenter for metodene. Selv om teori for kontinuerlige modeller faller
utenfor vår matematiske ramme, kommer vi ikke utenom å gi
en grunnleggende innføring i normalfordelingen og sentrale statistiske
argumenter og metoder knyttet til denne.

I annen del av boken presenteres ulike problemområder der 
statistisk tankegang kommer til anvendelse.  Framstillingen
er til dels mindre syste\-matisk enn første del. Vi peker på muligheter,
men viser til spesiallitteratur for mer detaljert informasjon.

 {\bf Advarsel :} Av plasshensyn er mange av de datamaterialer som er
knyttet til eksempler og oppgaver mindre enn det man i praksis ønsker seg
for å belyse problemstillingene. På den annen side tar teorien
nettopp sikte på å vurdere påliteligheten av konklusjoner
basert på begrenset informasjon.

Ved innlæringen av statistiske metoder og dataanalyse i praksis, vil
en kunne dra nytte av en statistisk programpakke. Her anbefales MINITAB,
men også andre pakker, som STATGRAPHICS, SPSS, SYSTAT, DATA DESK og JMP,
kan tjene formålet. Det typiske regneark (som EXCEL) er ikke like tjenlig.
Jeg har ikke funnet det hensiktsmessig å la en bestemt regneteknologi
prege boken, men gir noen eksempler på typiske (engelske) utskrifter.

Boken bruker følgende systematikk:  Kapitler og avsnitt nummereres 
med tallkombinasjoner, eksempelvis betyr 2.3 tredje avsnitt i Kapittel 2.
Eksempler og oppgaver nummereres fortløpende innen hvert kapittel. 
Ved referanse til eksempel eller oppgave i et annet kapittel brukes
tallkombinasjon, eksempelvis vil Oppgave 4.15 være Oppgave nr. 15 i
Kapittel 4.  En del stoff er $\star$-merket og kan hoppes over uten å miste
sammenhengen.  Det samme er tilfellet med enkelte oppgaver som sprenger
grensene for det stoff som er gjennomgått.  Ved slutten av boka finnes
tre appendiks med terminologi, formler og tabeller.


Et mulig pensum for et ett-semesters kurs i sannsynlighetsregning og
statistikk med 2 vekttall vil kunne hentes fra
kapitlene 1-8 i Del I, $\star$-merket stoff unntatt. Trengs 
beskjæring er ett eller flere av avsnittene 3.6, 5.7, 6.6, 7.6, 7.7, 8.5
og 8.6 nærmest. Pensum for et 2 vekttalls kurs i sannsynlighets\-regning
alene, vil kunne omfatte kapitlene 1-6 i Del I, inkludert $\star$-merket stoff 
og eventuelt supplert med første del av Kapittel 16.

Et 4 vekttalls kurs i sannsynlighetsregning og statistikk vil kunne
 omfatte kapitlene 1-8 i del I, samt 4-5 utvalgte
kapitler fra Del II. Et alternativ er færre kapitler fra Del II, med
fordypning i $\star$-merket stoff fra Del I eller med supplerende opplæring
i bruk av statistisk programvare.

Jeg takker Henrik Dahl for mange inspirerende samtaler om faglige og
pedagogiske spørsmål forut for førsteutgaven og ved flere revisjoner.
Takk går også til Geir Egil Eide.

I 5. utgave er det kommet til ett nytt avsnitt: 8.7 Resampling og en rekke nye
oppgaver. Flere kapitler er reorganisert og forenklet, bl.a. ved at en del
formelle begrunnelser er flyttet til sist i avsnittet og $\star$-merket.
\begin{flushright}
  Bergen, 30 september 1996\\ 
 \ \ \ Jostein Lillestøl \ \ \ \ \ \ 
\end{flushright}
